\subsection{\emph{Bluetooth}}

Destina-se a substituir os cabos de conexão de dispositivos fixos e/ou portáteis mantendo altos níveis de segurança. Sua força fundamental está na capacidade de lidar simultaneamente com dados e transmissão de voz, o que fornece aos usuários uma variedade de soluções inovadoras, tais como fones de ouvido \emph{hands-free} para chamadas de voz, impressão, fax, sincronização com PCs e telefones celulares, entre outros~\cite{bluetoothoverview}.

É uma tecnologia de comunicação de curto alcance. Suas principais características são a robustez, a baixa potência e o baixo custo. Sua especificação define uma estrutura uniforme para que uma ampla gama de dispositivos possam se conectar e se comunicar uns com os outros~\cite{bluetoothoverview}.

Sua estrutura e aceitação global permitem que qualquer dispositivo \emph{Bluetooth} ativado possa se conectar a outros dispositivos \emph{Bluetooth} situados em sua proximidade. Uma vez estabelecida, uma conexão \emph{Bluetooth} permite que aparelhos a uma curta distância se comuniquem sem a utilização de fios, ou seja, por meio de redes ad hoc conhecidas como piconets. Estas são estabelecidas de forma dinâmica e automatica assim que um dispositivo entra ou sai do seu raio de alcance, facilitando o processo de conexão por parte do usuário~\cite{bluetoothoverview}.

Cada dispositivo em uma piconet pode pertencer a várias outras piconets e, simultaneamente, se comunicar com até sete outros dispositivos presentes em uma mesma rede~\cite{bluetoothoverview}.