\subsection{Ontologia}
Uma Ontologia é uma descrição formal de conceitos(classes) em um determinado domínio. Esses conceitos possuem propriedades, que descrevem atributos, características e restrições. De forma semelhante à orientação à objetos, conceitos podem possuir subclasses que especializam a classe superior.

Para construir uma Ontologia deve-se definir as classes do domínio, distribuir as classes de forma hierárquica, definir valores e restrições para suas propriedades e utilizar esses valores nas instâncias das classes da Ontologia. Dessa forma, temos uma base de conhecimento.

As ontologias se tornaram popular na Web por dividir produtos em categorias e características em sites de venda e atualmente são utilizadas para compartilhar informação entre pessoas e agentes inteligentes. Uma antologia provê reutilização de conhecimento e torna explícitas as hipóteses sobre um domínio e análise do domínio.

Uma ontologia, portanto, se torna útil para o objetivo deste trabalho.