\subsection{DSOA - \emph{Device Service Oriented Architecture}}

A DSOA foi definida baseada na arquitetura SOA(\emph{Service Oriented Architecture}). A SOA possui diversas características que podem auxiliar ambientes inteligentes, proporciona a capacidade da reutilização de recursos de software. Entretanto, por ser uma apenas uma arquitetura genérica, não leva em consideração características e limitações de ambientes inteligentes, descrevendo apenas serviços em alto nível e não define, ainda, um modelo de comunicação.

Na DSOA destacam-se cinco conceitos básicos:

\begin{itemize}
	\item Ambiente Inteligente:
		É um ambiente composto por pelo menos dois dispositivos capacidade de computação e conectados por meio de uma rede de comunicação coloborativa com os usuários do ambiente. O provimento de serviços para o usuário vem da interação das aplicações deste ambiente dinâmico. Essa dinamicidade do ambiente ubíquo se dá pelo fato que os usuários entram, saem e se movimentam no ambiente, além disso, trazem consigo seus dispositivos. É importante ressaltar que um ambiente inteligente neste trabalho não tem seu significado mais popular de ambiente com inteligência artificial.
	\item Dispositivo:
		É um aparelho que possua capacidade de comunicação e que abrigar aplicações ou disponibilizar seus recursos do ambiente inteligente para que outras aplicações possam utilizar de seus serviços.
	\item Recurso:
		
	\item Serviço
	\item Aplicação
\end{itemize}
