\subsubsection{\emph{Gator Tech}}

O projeto \emph{Gator Tech Smart House} é o resultado de mais de cinco anos de pesquisa na área de computação pervasiva e móvel. O objetivo do projeto é criar ambientes assistivos como casas que terão conhecimentos sobre si e sobre seus residentes criando um mapeamento entre o mundo físico, monitoramento remoto e serviços de intervenção~\cite{gatorTech}.

Neste projeto foi desenvolvida uma arquitetura para plataforma de sensores orientada à serviços, o Atlas. A plataforma Atlas é uma combinação de nós de \emph{hardware} e \emph{firmware} executado em um \emph{hardware} e um \emph{middleware} executando na rede, que provê serviços em um ambiente. Juntos, esses componentes permitem que qualquer sensor, atuador ou qualquer outro dispositivo sejam integrados e controlados por meio da interface de um determinado dispositivo. Essa abordagem facilita o desenvolvimento de aplicações que utilizam esses dispositivos~\cite{gatorTechLessons}. 

O Atlas é responsável por obter a representação de serviços dos dispositivos conectados e gerenciar os serviços de modo que as aplicações possam obter e utilizar os serviços facilmente. Na implementação do \emph{Gator Tech}, a camada de serviços foi construída sobre o \emph{framework} OSGi que mantém o registro dos serviços de todos os nós conectados. Cada sensor ou atuador é representado no \emph{middleware} Atlas como um pacote de serviços OSGi. Inicialmente esses serviços eram representados em classes Java. Entretanto, a criação desses pacotes não acontecia de forma automática. 

Para cada modelo de sensores e atuadores, era necessário ler a especificação do fabricante, examinar a interface e estudar os protocolos de comunicação. Era necessário ainda, que quem escrevesse o pacote fosse especialista na programação Java e no \emph{framework} OSGi. Com o objetivo de resolver essa complexidade o grupo de pesquisa do Atlas desenvolveu a \emph{Device Description Language}(DDL), mostrada na subseção ~\ref{subsec:ddl}. 