\subsubsection{USB}

O \emph{Universal Serial Bus} (USB) é uma arquitetura de comunicação que adiciona a uma máquina hospedeira a capacidade de se interconectar a uma variedade de dispositivos. Surgiu com o intuito de sanar três das principais dificuldades enfrentadas à época de sua criação~\cite{usbspec}:

\begin{itemize}
	\item Integrar as plataformas das industrias da informática e comunicação. Para tal, a troca de informações entre esses dispositivos deveria acontecer de forma ubíqua e barata;
	\item Facilitar a reconfiguração de dispositivos de um computador, tais como teclados, mouses e \emph{joysticks};
	\item Integrar todas as interfaces existentes à época em um única que pudesse ser utilizada pela maior quantidade possível de dispositivos: telefone, fax, \emph{modem}, adaptadores, secretárias eletrônicas, \emph{scanners}, PDA's, teclados, \emph{mouses}, etc.
\end{itemize}

De acordo com sua especificação, suas máquinas hospedeiras devem:

\begin{itemize}
	\item Fornecer energia aos seus periféricos;
	\item Suportar todas as velocidades definidas (SuperSpeed, Low Speed, Full Speed e High SuperSpeed);
	\item Suportar todos os tipos de fluxo de dados definidos (controle, massa, interrupção e síncrono).
\end{itemize}

Entende-se por máquina hospedeira qualquer dispositivo possuidor dos recursos necessários para realizar esta função. Por exemplo, conectar uma câmera à uma impressora faz sentido, enquanto que conectar um GPS à mesma impressora, não. Assim sendo, cada máquina hospedeira não necessita se comportar exatamente como um computador, mas apenas servir como hospedeiro a certos dispositivos convenientes~\cite{usb3spec}.

Atualmente sua velocidade de comunicação varia entre 1.5 ou 12 megabits por segundo (mbs) e seu protocolo permite configurar dispositivos durante a fase de inicialização ou quando eles são plugados em tempo de execução~\cite{hid}, deixando toda parte pesada desse processo a cargo do hospedeiro~\cite{usb3spec}. Tais dispositivos são divididos em várias classes e/ou subclasses diferentes, cada qual definindo um comportamento e protocolos comuns para dispositivos que oferecem funções similares~\cite{hid}.

Um mesmo dispositivo USB pode pertencer a uma ou múltiplas classes. Por exemplo, um celular pode utilizar atributos da classe HID, Áudio e Comunicação. A partir dessas divisões será possível estabelecer uma hierarquia, a qual poderá ser usada no processo de classificação dos dispositivos presentes no smartspace.

Seguem abaixo as classes existentes juntamente com suas descrições~\cite{usbclasscodes}.

\begin{itemize}
	\item \emph{Audio}: 

	Destina-se a todos os dispositivos ou funções embutidos em aparelhos que são usados para manipular áudio, voz e quaisquer funcionalidades relacionadas. Isso inclui tanto dados de áudio (analógico e digital) quanto a capacidade de diretamente controlar tal ambiente por meio de controles de volume e tom. Não inclui a funcionalidade para operar mecanismos de transporte que estão relacionados com a reprodução de dados de áudio, tais como mecanismos de transporte de fita ou controles de unidade de CD-ROM~\cite{usbaudioclass}.

	Exemplos de tais dispositivos incluem \emph{headsets}, \emph{headphones} e \emph{microphones}~\cite{usbbasicaudioclass}.
	\item \emph{Communications and CDC Control}: 

	Há três classes que compõem a definição de dispositivos de comunicação: a Classe de Comunicação do Dispositivo, a Classe de Interface de Comunicação e a Classe de Interface de Dados. A primeira é uma definição de nível do dispositivo e é utilizada pelo hospedeiro para identificar corretamente um dispositivo de comunicação que pode apresentar diferentes tipos de interfaces. A segunda define um mecanismo de propósito geral que pode ser utilizado para habilitar todos os tipos de serviços de comunicação sobre o Universal Serial Bus (USB). A última define um mecanismo de propósito geral para permitir a transferência em massa quando os dados não atendem aos requisitos para qualquer outra classe~\cite{usbcommunicationclass}.

	Exemplos de tais dispositivos incluem:
	\begin{itemize}
		\item Telecomunicações: 

		Modens analógicos, adaptadores de terminais ISDN, telefones digitais e telefones analógicos;
		\item Dispositivos de rede: 

		Modens ADSL, modens à cabo, adaptadores/hubs \emph{Ethernet} 10BASE-T e \emph{Ethernet} cross-over cabos.
	\end{itemize}
	\item HID (\emph{Human Interface Device}): 

	Destina-se primordialmente aos dispositivos que são usados por humanos para controlar e operacionalizar quaisquer tipos de sistemas computacionais~\cite{hid}. Mais específicamente, possui as seguintes particularidades:
	\begin{itemize}
		\item Deve ser tão compacto quanto possível para evitar desperdício de espaço no dispositivo;
		\item Permitir a aplicação de software para evitar informação desconhecida;
		\item Ser extensível e robusto;
		\item Suportar hierarquias e coleções;
		\item Ser autodescritivo para permitir aplicações de software genéricos.
	\end{itemize}

	Exemplos de tais dispositivos incluem: teclados e dispositivos apontadores, painéis de controle, controles que podem ser encontrados em dispositivos como telefones, controles remotos, jogos ou dispositivos de simulação e aparelhos que não necessitam de interação humana mas que fornecem dados de forma similar aos dipositivos da classe HID~\cite{hid}.
	\item \emph{Physical}: 

	É vista como uma extensão da classe HID para dispostivos que requerem resposta física em "tempo real". Seu foco principal está em dispositivos táteis e na implementação de sistemas com resposta à força. Contúdo, não há exigência de que os membros desta classe gerem esse tipo de efeito~\cite{usbphysicalclass}.

	Exemplos de tais dispositivos incluem: \emph{joysticks} e plataformas de movimento.
	\item \emph{Image}: 

	Destina-se aos dispositivos de captura de imagem, como câmeras digitais ou quaisquer outros aparelhos similares~\cite{usbimageclass}.
	\item \emph{Printer}: 

	Destina-se aos dispositivos de impressão. Estes, anteriormente, eram interfaceados por meio das seguintes tecnologias~\cite{usbprintclass}:
	\begin{itemize}
		\item Porta paralela unidirecional;
		\item Porta paralela bi-direcional;
		\item Porta serial;
		\item Porta SCSI;
		\item \emph{Ethernet} / LAN.
	\end{itemize}

	Atualmente existem outras interfaces mais sofisticadas, mas essas são as mais comuns.

	O USB oferece uma capacidade de processamento muito maior que a porta serial e é comparável em velocidade à porta paralela~\cite{usbprintclass}.
	\item \emph{Mass Storage}: 

	Destina-se aos dispositivos de armazenamento em massa, ou seja, todo aparelho que possa ler ou gravar dados digitais em algum sistema de armazenamento.

	Exemplos de tais dispositivos incluem: pendrives, discos de estado sólido com memória flash não volátil e HD's.
	\item \emph{Hub}: 
	\item \emph{Smart Card}: 

	Destina-se aos dispositivos compostos por cartões de circuitos integrados~\cite{usbsmartcard}.
	Exemplos de tais dispositivos incluem: cartões de crédito, CPF's digitais, etc.
	\item \emph{Content Security}: 

	Destina-se primordialmente em proteger e controlar a distribuição de conteúdo digital. Especifica um conjunto comum de requisições de transporte de dados e descritores necessários para apoiar os vários métodos de segurança de conteúdo~\cite{usbcontentsecurityclass}.
	\item \emph{Video}: 

	Destina-se a todos os dispositivos ou funções que compõem outros aparelhos utilizados para manipular vídeos~\cite{usbvideoclass}.

	Exemplos de tais dispositivos incluem: \emph{webcams}, filmadoras digitais, conversores analógicos de vídeo, sintonizadores de televisão analógica e digital e imagens de câmeras que suportam \emph{streaming} de vídeo~\cite{usbvideoclass}.
	\item \emph{Personal Healthcare}: 

	Destina-se em permitir a interoperabilidade entre aparelhos da saúde pessoal e hospedeiros USB, o que permite uma maior variedade de casos de uso para ambos os dispositivos e hospedeiros~\cite{usbhealthcareclass}.

	Exemplos de tais dispositivos incluem: medidores de glicose e oxímetros de pulso~\cite{usbhealthcareclass}.
	\item \emph{Diagnostic Device}: 
	\item \emph{Wireless Controller}: 
	\item \emph{Miscellaneous}: 
	\item \emph{Application Specific}: 
	\item \emph{Vendor Specific}: 
\end{itemize}
