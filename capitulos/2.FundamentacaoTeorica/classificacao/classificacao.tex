\section{Classificação de Recursos}

Como descrito anteriormente, o \emph{uOS} tem como foco a adaptabilidade de serviços, em que cada serviço pertence a um determinado recurso. Este, por sua vez, possui um identificador (ou nome), que o representa unicamente entre os demais. Esta abordagem, apesar de funcional, é deficiente, pois cada \emph{driver} é visto de maneira isolada dos outros.

Imagine, por exemplo, o caso em que uma aplicação necessita de um serviço de \emph{snapshot} presente em um recurso de câmera. Suponha também que, dentre os diversos dispositivos registrados na base do \emph{middleware}, existam três que possuam o serviço desejado:

<PENSAR NUMA REPRESENTAÇÃO MELHOR>
\begin{itemize}
	\item D1
	\\Recurso: Camera
	\\Serviços:
	\begin{itemize}
		\item \emph{snapshot}
	\end{itemize}
	\item D2
	\\Recurso: CameraVideo
	\\Serviços:
	\begin{itemize}
		\item \emph{snapshot}
		\item \emph{streaming} de vídeo
	\end{itemize}
	\item D3
	\\Recurso: Imagem
	\\Serviços:
	\begin{itemize}
		\item \emph{snapshot}
	\end{itemize}
\end{itemize}

Dito isto, cabe ao \emph{uOS} determinar qual recurso será selecionado à aplicação requisitante. Entretanto, considere que o recurso escolhido (D1) esteja indisponível. Neste caso, o sistema deverá repassar um outro recurso que seja equivalente ao primeiro. Entende-se por equivalente todo recurso que:

\begin{itemize}
	\item Possua o mesmo identificador;
	\item Possua serviços equivalentes. Neste caso, tais serviços devem possuir o mesmo retorno e a mesma assinatura, ou seja, mesmos nome, quantidade, tipo e ordem de parâmetros.
\end{itemize}

Repare que D2 e D3 também possuem o serviço de \emph{snapshot}, mas enquanto D3, por se tratar de um recurso de natureza completamente distinta, não poderia ser utilizado de forma alguma pela aplicação solicitante, D2 poderia sem perda de funcionalidade. Contudo, de acordo com a definição acima, isso não é possível, visto que D1 e D2 são recursos diferentes e, portanto, não são equivalentes. Como dito inicialmente, essa deficiência é causada justamente pelo fato de cada \emph{driver} ser visto de forma isolada e não-relacionada pelo \emph{middleware}. Este trabalho tem como um dos objetivos solucionar este problema.


----------------------------------------- REVER -----------------------------------------
<COLOCAR ISSO NA PROPOSTA DE CLASSIFICAÇÃO>

Desta forma, faz-se necessária uma maneira de classificar tais recursos imersos nos mais variados dispositivos presentes no \emph{smart space} e regidos pelo \emph{middleware}. Essa classificação facilitará o desenvolvimento de novos \emph{drivers} para futuras aplicações, pois tornará possivel a definição de interfaces pré-estabelecidas que representem classes de recursos. Outra vantagem decorrente é a possibilidade de seleção de recursos equivalentes, caso o provedor originalmente selecionado esteja indisponível.

<VER SE CABE FALAR ISSO NO JSON>

Em um ambiente cujos dispositivos podem se movimentar livremente e seus recursos devem ser compartilhados de forma transparente ao usuário, torna-se necessário que cada dispositivo presente na rede possa se comunicar facilmente com os demais, independente de quais tecnologias, padrões ou protocolos tenham sido adotados. É fundamental que essa comunicação seja feita de forma tão rápida e confiável quanto possível, pois toda atividade do \emph{middleware} com seus clientes depende dela, inclusive o processo de classificação dos recursos disponíveis no \emph{smart space}, idéia central deste documento.

----------------------------------------- REVER -----------------------------------------

\subsection{Formas de classificação}

Existem duas formas principais de se realizar uma classificação de recursos:

\begin{itemize}
	\item Fixa: conjunto fixo de classificações não-relacionadas;
	\item Relacionada: estabelece relações entre os tipos básicos pré-definidos e suas especializações.
\end{itemize}

As classificações de forma relacionada possuem a vantagem de reutilizar atributos definidos em outras classes e de poder adicionar novos atributos especializando os tipos relacionados, o que evita redundância na definição das classes. Enquanto as classificações de forma Fixa possuem a vantagem de serem mais simples de serem definidas.

Uma importante propriedade de uma classificação é a hierarquia, ou seja, uma representação em que as categorias da classificação podem ser especializadas formando novas categorias mais específicas, com diferentes atributos. Existem diferentes maneiras conhecidas de se representar uma hierarquia de forma a relacionar categorias que tenham atributos em comum: arquivos no formato XML, hierarquia de classes, ontologia ou simplesmente não utilizar uma hierarquia onde as categorias não tem relações entre si, caso da forma fixa de classificação.

O XML (\emph{Extensible Markup Language}), como seu próprio nome sugere, pode ser estendido e dessa forma, é possível montar uma hierarquia de arquivos XML. O seu uso para descrever um tipo de dispositivo tem a vantagem de ser independente de plataformas de hardware ou software e, devido sua formatação, possui uma boa legibilidade, não necessitando de um pré-processamento por parte de um computador para seu conteúdo se tornar compreensível. Esse formato possui, contudo, a desvantagem de repetir uma grande quantidade de informação o que pode prejudicar a velocidade de comunicação entre dispositivos que o utilizem. Outro fator que deve ser levado em consideração é que o conjunto de protocolos \emph{uP} (\emph{Ubiquitous Protocols}) do \emph{middleware} \emph{uOS} utiliza o formato JSON para troca de mensagens, um mecanismo mais leve, e ainda assim estruturado, para representar informação.

A hierarquia de classes é amplamente utilizada na programação orientada à objetos. Uma classe possui atributos ou propriedades e uma outra classe pode herdar esses atributos e criar uma especialização dessa classe raiz, à esse relacionamento dá-se o nome de herança. Dependendo da linguagem utilizada, uma classe pode herdar métodos e atributos de várias outras classes, formando uma herança múltipla de classes, ou então poderá cada subclasse poderá herdar de apenas uma outra classe, não podendo, entretanto, ocorrer uma herança circular, onde uma classe A herda de uma classe B, que por sua vez herda atributos de uma classe C que especializa a classe A. Uma hierarquia de classes forma, portanto, uma espécie de árvore de classes.

Ontologia é uma descrição formal de conceitos (classes) em um determinado domínio. Esses conceitos possuem propriedades, que descrevem atributos, características e restrições. De forma semelhante à orientação à objetos, conceitos podem possuir subclasses que especializam a classe superior. A Ontologia já permite uma herança múltipla de classes, e novos conceitos podem ser construídos por meio do relacionamento entre classes. Para construir uma Ontologia deve-se definir as classes do domínio, distribuir as classes de forma hierárquica, definir valores e restrições para suas propriedades e utilizar esses valores nas instâncias das classes da Ontologia. 

\begin{comment}
Dessa forma, temos uma base de conhecimento. As Ontologias se tornaram populares na Web por dividir produtos em categorias e em características em sites de venda, e atualmente são utilizadas para compartilhar informação entre pessoas e agentes inteligentes. Uma antologia provê reutilização de conhecimento e torna explícitas as hipóteses sobre um domínio e análise do domínio.
\end{comment}


\subsection{Padrões}
%Apresentar os padrões. Focar nas classificações feitas por eles, pontos fortes e fracos.

Nesta seção, discutir-se-á cinco dos principais protocolos e/ou padrões pesquisados que fazem uso de uma classificação de recursos e utilizam as estratégias supracitadas para representar sua hierarquia de recursos: UPnP, IEEE 1451, DLNA, USB, \emph{Bluetooth} e DDL.

\subsection{UPnP}
\label{subsec:upnp}
O \emph{Universal Plug and Play Protocol} (UPnP) é um protocolo para conectividade entre aplicações inteligentes utilizado em computadores e dispositivos, \emph{wireless} em geral. Foi definido com o objetivo de facilitar a conectividade entre dispositivos em diferentes ambientes, como casas, pequenas empresas ou espaços públicos~\cite{upnpArch}. Não necessita de configurações e fornece uma rede invisível com descoberta automática de dispositivos de diferentes fabricantes e diversas categorias.

Seu nome ``Universal'' advém da não utilização de \emph{drivers} para cada dispositivo. Para tal, utiliza protocolos bem conhecidos como IP, TCP, UDP e HTTP, além do formato XML, utilizado no armazenamento de informações tais como fabricante, lista de dispositivos embarcados, descrição dos serviços com lista de comandos e ações com parâmetros e argumentos, URL's de controle, eventos e apresentação.

Quando um novo aparelho adentra na rede, ele recebe um IP via DHCP ou gera um IP para si (no caso de redes não-gerenciadas), publica seus serviços, disponibiliza uma URL e conhece os serviços de outros dispositivos. A \emph{UPnP Device Architecture}(UDA) divide os dispositivos em duas categorias principais: controlados, também chamados de ``dispositivos'' e pontos de controle~\cite{upnpArch}, estabelecendo uma arquitetura cliente-servidor onde pontos de controle acessam os serviços disponibilizados pelos dispositivos. Após a entrada desse novo aparelho na rede, os pontos de controle ainda não têm muita informação sobre o dispositivo, então os pontos de controle interessados solicitam a descrição do dispositivo por meio da URL disponibilizada anteriormente.

O UPnP estabelece uma classificação utilizando um conjunto de tipos pré-determinados. Quando um fabricante deseja produzir um novo dispositivo, este deve se adequar a estes tipos ou estabelecer um novo tipo próprio. Uma característica interessante dessas classificações é que elas podem ser implementadas por dispositivos com pouca capacidade de computação, embora também possam ser utilizadas por dispositivos mais robustos como computadores, ou seja, são independentes de plataforma. Os tipos presentes no UPnP são:

\begin{itemize}
\item Áudio/Vídeo:

	Essa categoria possui duas principais sub-classificações que foram sendo atualizadas com o surgimento de novas tecnologias: 
	\begin{itemize}
		\item \emph{Media Server}:

			Define um dispositivo genérico que provê conteúdo de áudio e/ou vídeo, como CD e DVD \emph{players}, cameras, rádios, televisões e \emph{set-top boxes}. Embora, seja utilizado por dispositivos com diferentes capacidades de processamento, diferentes conteúdos e em diferentes formatos, o \emph{Media Server} expõe o conteúdo disponível de forma uniforme e consistente por meio de um Diretório de Conteúdo.

		\item \emph{Media Renderer}:

			Define um dispositivo genérico capaz de renderizar conteúdos de áudio e/ou vídeo como MP3 \emph{players} e televisões. Dependendo da implementação de um \emph{Media Renderer}, pode-se utilizar os recursos de auto-falante de uma televisão para consumir um serviço de música de um \emph{Media Server}.
	\end{itemize}
\item Gerenciamento de Dispositivos:

	Essa categoria foi criada para adicionar operações gerenciais à qualquer dispositivo UPnP. Essa gerência inclui funções para configuração de serviços e do próprio dispositivo, diagnóstico e correção de problemas além de gerência do \emph{firmware} e dos \emph{softwares} do dispositivo.

\item Automação Residencial:

	O UPnP define quatro tipos de dispositivos de automação residencial:
	\begin{itemize}
		\item Cortina de Proteção Solar:

			Provê uma sombra por meio de uma cortina. Seu controle pode ser manual, automático ou desabilitado. Sua especificação não contempla configurações a respeito da automação da cortina ou proteções.
		\item Câmera Digital de Segurança:

			Provê controle básico sobre a configuração da câmera, contendo serviços de fotos e vídeos.
		\item Aquecimento, Ventilação e Ar-Condicionado:
			
			Esse dispositivo conta com auxílio de sensores de temperatura e possui a capacidade de saber ou controlar a temperatura do ambiente por meio de ventiladores e ar-condicionados.
		\item Controles de Luz:
			
			São divididos em \emph{Luz binária}, que representa uma lâmpada ou qualquer dispositivo emissor de luz que possa somente estar apagado ou aceso, e em \emph{Luz} cuja intensidade pode ser alterada, 
	\end{itemize}

\item Rede:

	Nessa categoria se enquadram dispositivos relacionados com uma rede, seja um \emph{gateway} ou um ponto de acesso local.
	\begin{itemize}
		\item \emph{Gateway} de Internet:
			
			Essa classificação define um dispositivo de interconexão entre uma \emph{Local Area Network} (LAN) e a \emph{Wide Area Network} (WAN), provendo conectividade com a internet.
			\begin{itemize}
				\item Dispositivo de Conexão WAN: É um dispositivo virtual definido tendo o \emph{Gateway} de Internet como raiz. Funciona como um contêiner para um \emph{link} ou serviços de conexão em uma interface WAN. 
				\item Dispositivo WAN: É um dispositivo virtual definido tendo o \emph{gateway} de internet como raiz. Cada dispositivo WAN é uma instância virtual de uma interface WAN no \emph{gateway} de internet. Múltiplas interfaces físicas WAN para clientes UPnP, existirão distintas instâncias deste dispositivo.
			\end{itemize}

		\item Ponto de acesso WLAN(\emph{Wireless Local Area Network}):
			
			Esse dispositivo implementa os padrões IEEE 802.11 (a,b,g) sem fio para prover uma infraestrutura de rede para casas e pequenas empresas. Essa definição não inclui uso dos pontos de acesso como \emph{hotspots} ou redes de grandes empreendimentos. O ponto de acesso age como uma ponte \emph{Ethernet} que permite ligações de múltiplos nós com a LAN.
	\end{itemize}

\item Impressora:

	Define um dispositivo com capacidade de impressão. Essa especificação não abrange dispositivos que possuem funções de FAX ou \emph{Scanner}, que possui uma especificação própria.

\item Acesso Remoto:

	Essa categoria comporta dispositivos que gerenciam o acesso remoto, ela é dividida entre:
	\begin{itemize}
		\item Agente de Descoberta de Acesso Remoto:

			Possui a função de prover a capacidade de sincronizar a informação sobre a descoberta UPnP entre duas redes remotas.
		
		\item Servidor de Acesso Remoto:
		
			Permite que pontos de controle configurem Servidores de Acesso Remoto.

	\end{itemize}

\item Interface Remota:

	Classifica dispositivos entre servidores e clientes de uma interface remota com o usuário.

\item \emph{Scanner}:

	Representa um dispositivo de \emph{Scanner} com \emph{feeder} opcional. Esse dispositivo possui os serviços de digitalização via \emph{feeder} ou \emph{flatbed} e um serviço para configuração com o painel frontal do dispositivo. Essa categoria não contempla funcionalidades de Fax ou cópias.

\item Telefonia:
	
	Categoria para aparelhos que se relacionam com algum serviço de telefonia, tanto como consumidor quanto como con provedor.
	\begin{itemize}
		\item Servidor de Telefonia:

			Permite que pontos de controle gerenciem chamadas telefônicas, mensagens e presença por meio de outros dispositivos UPnP. 

		\item Cliente de Telefonia:
		
			Permite que pontos de controle possam gerenciar mídias por meio de um de um servidor de telefonia.
				
	\end{itemize}

\item Básico:

	A definição de dispositivos básicos provêem um mecanismo para produtos que não se enquadram em uma classificação adequada do UPnP, possam utilizá-lo. Por esse motivo, essa especificação não possui nenhum serviço definido, mas pode ser utilizada como dispositivo raiz para outras categorias já definidas.
\end{itemize}

A partir de uma classificação padrão, os fabricantes podem, por exemplo, estender e especializar determinada definição adicionando novos serviços, o que sugere a formação de uma hierarquia de dispositivos por meio de arquivos XML. O UPnP, portanto, utiliza uma classificação dispositivos relacionada, o que facilita a criação de novos tipos de dispositivos a partir da classficação base agregando novos serviços. Essa classificação extendida, entretanto, deve passar por um processo de certificação. Um ponto forte nas classificações é o fato de já incluirem a especificação dos serviços que os dispositivos provêem.

\subsubsection{IEEE 1451}
O IEEE 1451 se divide em uma família de padrões que foram criados com os objetivos de permitir a capacidade de comunicação entre transdutores(sensores e atuadores) de forma \emph{plug-and-play} por meio de redes com ou sem fio, facilitar a criação de transdutores com inteligência embarcada, simplificar a configuração e manutenção de sistemas, prover comunicação entre transdutores legados, e por fim, habilitar a implementação de transdutores inteligentes e com uso mínimo de memória~\cite{ieee1451journal}.

Um dos padrões da família, o IEEE 1451.1, define um modelo de informação para \emph{Network Capable Application Processors} (NCAP) que foi estabelecido para especificar um modelo de objetos comum e interfaces de componentes da rede de transdutores. Dessa forma, foi desenvolvido um \emph{framework} orientado a objetos que pode ser estendido para facilitar o desenvolvimento de aplicações. Este modelo foi definido da seguinte forma: Um modelo de dados que especifica a forma e o tipo de comunicação, tanto local quanto remota, por meio das interfaces de objetos 1451.1, um modelo de objetos que especifica tipos de componentes de software usados para definir e implementar sistemas e, por fim, modelos de comunicação que definem a sintaxe e semântica das interfaces de software entre redes de comunicação e objetos de aplicação~\cite{ieeeOO1451}~\cite{ieee1451monitoring}.


O padrão ~\cite{ieee1451standard} especifica cada classe no modelo definindo as interfaces da classe(por meio de assinaturas e operações) e o comportamento da classe (via texto ou máquinas de estado). Abaixo temos a hierarquia de classes definida pelo padrão em que as classes em itálico representam classes abstratas:

\begin{tabbing}
\textit{Root} \= \\
\> \textit{Entity} \= \\
\> \> \textit{Block} \= \\
\> \> \> NCAP BLOCK \\
\> \> \> \textit{Function Block} \\
\> \> \> \textit{Base Transducer Block} \\
\> \> \> \textit{Transducer} \= \textit{Block} \\
\> \> \> \> Dot2 Transducer Block \\
\> \> \> \> Dot3 Transducer Block \\
\> \> \> \> Dot4 \= Transducer Block \\
\> \> \textit{Component} \\
\> \> \> Parameter \\
\> \> \> \> Parameter With Update \\
\> \> \> \> \> \= \textit{Physical Parameter} \\
\> \> \> \> \> \> Scalar \= Parameter \\
\> \> \> \> \> \> \> Scalar Series Parameter \\
\> \> \> \> \> \> Vector Parameter \\
\> \> \> \> \> \> \> Vector Series Parameter \\
\> \> \> \> Time Parameter \\
\> \> \> Action \\
\> \> \> File \\
\> \> \> \> Partitioned File \\
\> \> \> Component Group \\
\> \> \textit{Service} \\
\> \> \> \textit{Base Port} \\
\> \> \> \> \textit{Base Client Port} \\
\> \> \> \> \> Client Port \\
\> \> \> \> \> Asynchronous Client Port \\
\> \> \> \> \textit{Base Publisher Port} \\
\> \> \> \> \> Publisher Port \\
\> \> \> \> \> Self Identifying Publisher Port \\
\> \> \> \> \> \> \> Event Generator Publisher Port \\
\> \> \> Subscriber Port \\
\> \> \> Mutex Service \\
\> \> \> Condition Variable Service \\
\end{tabbing}

É possível observar 3 tipos principais de objetos IEEE 1451.1:

\begin{itemize}
	\item\emph{Block}:

	Especializada em três classes:
		\begin{itemize}
			\item\emph{NCAPBlock}:

				Provê interfaces para comunicações de rede e configurações do sistema. Cada NCAP é modelado para possuir ao menos um processo de software. Cada processo de processo de software possuirá exatamente um \emph{NCAPBlock}.
			\item\emph{BaseTransducerBlock}:

				Provê interfaces entre transdutores e funções.
			\item\emph{FunctionBlock}:

				Provê encapsulamento de funções específicas.
		\end{itemize}
	
	\item\emph{Component}:
	
		Fornecem:
		\begin{itemize}
			\item Informações estruturadas: medidas e arquivos.
			\item Coleções de objetos relacionados com a aplicação.
			\item Ações com estados onde a ação é executada após um período de tempo.
		\end{itemize}
	\item\emph{Service}:
	
		Suportam:
		\begin{itemize}
			\item Comunicação entre objetos de diferentes NCAPs.
			\item Sincronização do sistema.
		\end{itemize}
\end{itemize}

Há ainda as classes não-IEEE 1451.1, que não estão representadas na Figura ~\ref{fig:classHierarchy} e possuem restrições de aplicabilidade na arquitetura IEEE 1451. Este padrão possui a limitação de ter sido criado para sensores e atuadores.



\subsection{DLNA}

A \emph{Digital Living Network Alliance} (DLNA) é uma organização composta pelas principais empresas de eletrônicos de consumo, computação e dispositivos móveis, tais como: Microsoft, Sony, Nokia, Samsung, Cisco entre outros. Foi fundada em 2003 e tem como objetivo fornecer orientações para permitir a interoperabilidade entre dispositivos para completar a convergência da indústria digital, levando, dessa forma, inovação, simplicidade e valor aos consumidores~\cite{dlnaoverview}. Tem como visão facilitar a criação, o gerenciamento e o compartilhamento de conteúdo digital (fotos, músicas e vídeos) entre os dispositivos pertencentes à mesma rede~\cite{dlnahdvideostreaming}. Deve permitir, por exemplo~\cite{dlnaoverview}:

\begin{itemize}
	\item facilmente adquirir, armazenar e acessar música digital a partir de praticamente qualquer lugar da casa;
	\item facilmente gerenciar, visualizar, imprimir e compartilhar fotos digitais;
	\item transportar seu conteúdo favorito a partir de qualquer lugar, mesmo que as pontas envolvidas estejam em movimento;
	\item aproveitar a gravação e reprodução de conteúdo distribuído e multi-usuário.
\end{itemize}

Suas diretrizes destacam casos de uso construídos para redes domésticas, funções adicionais que aumentam a experiência do compartilhamento de conteúdo e doze classes de dispositivos espalhados na categoria ``Rede Doméstica e Dispositivos Móveis''. A classificação de um dispositivo é feita de tal forma que um único aparelho multifuncional pode possuir diversas categorias diferentes~\cite{dlnahdvideostreaming, dlnaclasses}:

\begin{itemize}
	\item \emph{Home Network Devices} (HND)
	\begin{itemize}
		\item \emph{Digital Media Server} (DMS): 

		Esses dispositivos armazenam e disponibilizam conteúdo para \emph{Digital Media Players} (DMP) e \emph{Digital Media Renderes} que estão conectados na rede. Alguns servidores de mídia podem ainda proteger o conteúdo do usuário uma vez que ele tenha sido armazenado. Exemplo: PCs e dispositivos de armazenamento via rede(NAS);
		\item \emph{Digital Media Player} (DMP): 

		Esses dispositivos encontram conteúdo disponibilizados pelos servidores de mídia (DMS) e provêem a capacidade de reprodução e renderização de mídia. Exemplos: TVs, aparelhos de som, \emph{home theaters}, monitores sem fio e vídeo games;
		\item \emph{Digital Media Renderer} (DMR): 

		Esses dispositivos tocam conteúdos recebidos do controlador digital de mídia (DMC) que, por sua vez, irá encontrar conteúdo do servidor de mídia digital (DMS). Exemplos: TVs, receptores de áudio ou vídeo, projetores de vídeo e alto-falantes remotos para música;
		\item \emph{Digital Media Controller} (DMC): 

		Esses dispositivos encontram conteúdo em servidores de mídia digital (DMS) e o tocam em renderizadores de mídia digital (DMR). Exemplos: \emph{tablets}, câmeras digitais com Wi-Fi e PDAs;
		\item \emph{Digital Media Printer} (DMPr): 

		Esses dispositivos provêem serviços de impressão para a rede residencial DLNA. Geralmente, tocadores de mídia digital (DMP) e controladores de mídia digital (DMC) com a capacidade de impressão podem utilizar o DMPr para impressão. Exemplo: impressoras de foto e multifuncionais conectadas na rede DLNA.
	\end{itemize}
	\item \emph{Mobile Handheld Devices} (MHD)
	\begin{itemize}
		\item \emph{Mobile Digital Media Server} (M-DMS): 

		Esses dispositivos sem fio armazenam conteúdo e o tornam disponível para tocadores de mídia digital que tenham acesso à rede sem fio ou com fio (M-DMP), renderizadores de mídia digital (DMR) e impressoras de mídia digital (DMPr). Exemplo: telefones portáteis e tocadores portáteis de música;
		\item \emph{Mobile Digital Media Player} (M-DMP): 

		Esses dispositivos sem fio encontram e tocam conteúdo de um servidor de mídia digital (DMS) ou servidor móvel de mídia digital (M-DMS). Exemplo: telefones portáteis, \emph{tablets} projetados para visualização de conteúdo multimídia;
		\item \emph{Mobile Digital Media Uploader} (M-DMU): 

		Esses dispositivos sem fio enviam conteúdo para um servidor digital de mídia (DMS) ou para um servidor móvel de mídia digital (M-DMS). Exemplo: câmeras digitais, e telefones portáteis;
		\item \emph{Mobile Digital Media Downloader} (M-DMD): 

		Esses dispositivos sem fio encontram e armazenam conteúdo de um servidor de mídia digital (DMS) ou de um servidor móvel de mídia digital (M-DMS). Exemplo: tocadores de música e telefones portáteis;
		\item \emph{Mobile Digital Media Controller} (M-DMC): 

		Esses dispositivos encontram conteúdo em um servidor digital de mídia(DMS) ou em um servidor móvel de mídia digital (M-DMS) e o envia para renderizadores de mídia digital(DMR). Exemplo: telefones portáteis e PDAs.
	\end{itemize}
	\item \emph{Home Infrastructure Devices} (HID)
	\begin{itemize}
		\item \emph{Mobile Network Connectivity Function} (M-NCF): 

		Esses dispositivos provêem uma ligação entre a conectividade de rede de dispositivos móveis portáteis e conectividade de rede residencial;
		\item \emph{Media Interoperability Unit} (MIU): 

		Esses dispositivos provêem transformação de conteúdo entre formatos de mídia necessários para uma rede residencial ou para dispositivos móveis portáteis.
	\end{itemize}
\end{itemize}

\begin{table}
	\caption{Camadas padrões DLNA~\cite{dlnahdvideostreaming}.}
	\begin{center}
		\begin{tabular}{rp{5cm}p{5cm}}
		\hline
		\textbf{Camada} & \textbf{Função definida} & \textbf{Padrões}				\\
		\hline
		Transmissão Protegida & Como um conteúdo comercial está protegido em uma rede doméstica. & DTCP/IP \\
		\hline
		Formatos de Mídia & Como um conteúdo de mídia está codificado e identificado para interoperabilidade. & MPEG2, MPEG4, AVC/H.264, LPCM, MP3, AAC LC, JPEG, XHTML-Print- \\
		\hline
		Transporte de Mídia & Como um conteúdo de mídia é transferido. & HTTP, Quality of Service \\
		\hline
		Gerência de Mídia & Como um conteúdo de mídia é identificado, gerenciado e distribuído. & UPnP AV 1.0, UPnP Print Enhanced 1.0 \\
		\hline
		Descoberta e Controle & Como dispositivos se descobrem e se controlam um ao outro. & UPnP Device Architecture 1.0 \\
		\hline
		Redes IP & Como dispositivos com e sem fio fisicamente se conectam e se comunicam. & IPv4 Protocol Suite \\
		\hline
		Conectividade & Com qual rede os dispositivos se conectam e se comunicam& Wired: Ethernet 802.3, MoCAWireless: Wi-Fi 802.11, Wi-Fi Protected Setup \\
		\hline
		\end{tabular}
	\end{center}
	\label{tab:camadaspadroes_dlna}
\end{table}

\begin{comment}
\begin{table}
	\caption{Classes DLNA~\cite{}.}
	\begin{center}
		\begin{tabular}{llll}
		\hline
		\textbf{Classe} & \textbf{Categoria} & \textbf{Descrição} & \textbf{Exemplo}						\\
		\hline
		\hline
		Mobile Network Connectivity Function (M-NCF) & \multirow{2}{*}{Home Infrastructure Devices} &  & \\
		\hline
		Media Interoperability Unit (MIU) & &  & \\
		\hline
		\end{tabular}
	\end{center}
	\label{tab:classes_dlna}
\end{table}
\end{comment}

\begin{figure}[ht]
\center
\includegraphics[scale=0.5]{imagens/recursosDLNA}
\caption{Representação gráfica dos tipos existentes no DLNA.}
\label{fig:ddlspec}
\end{figure}

Considere como exemplo a Figura~\ref{fig:traditionalProccess}, em que uma pessoa deseja compartilhar um pequeno vídeo com seus amigos a partir de seu celular. A fim de exibir esse vídeo em sua TV widescreen na sua sala de estar, ela necessita, primeiramente, enviá-lo para seu próprio email. Em seguida, deve ligar seu PC, fazer o download do vídeo e salvá-lo em um \emph{pendrive} ou cartão de memória. Posteriormente, deve ligá-lo na TV ou em um receptor digital e, então, utilizar a interface do dispositivo para localizar o vídeo e exibi-lo~\cite{dlnahdvideostreaming}, tornando-se, dessa forma, um processo tedioso e demora que os consumidores raramente farão.

Além de interromper a conversa, esse tipo de processo de transferência de conteúdo exige tempo e atenção em excesso do usuário, mesmo quando ocorre sem problemas. Como resultado, a partilha informal de vídeo em múltiplas telas não é vista, muitas vezes, como uma opção razoável~\cite{dlnahdvideostreaming}.

\begin{figure}[ht]
	\center
	\includegraphics[scale=0.3]{imagens/dlna1}
	\caption{Processo de exibição de um vídeo, que está armazenado em um celular, em uma TV.}
	\label{fig:traditionalProccess}
\end{figure}

Este é o objetivo das organizações com compõem a DLNA: oferecer aos seus consumidores a partilha contínua e sem esforço de conteúdo digital, permitindo o envio de fotos ou vídeos para a TV em uma única etapa (Figura~\ref{fig:dlnaProccess}). O consumidor pode até mesmo congelar o vídeo usando um ``controle remoto'' do menu no telefone, bem como \emph{fast-forward} e reproduzir o vídeo~\cite{dlnahdvideostreaming}.

\begin{figure}[ht]
	\center
	\includegraphics[scale=0.3]{imagens/dlna2}
	\caption{Envio de conteúdo de digital para TV em uma única etapa.}
	\label{fig:dlnaProccess}
\end{figure}

Desta forma, o DLNA pode ser visto como um aglomerado de camadas e seus respectivos padrões abertos (Tabela~\ref{tab:camadaspadroes_dlna}) que definem como uma rede residencial interage em todos os seus níveis, ou seja, além de definir como os diferentes padrões irão interoperar e como os dados serão tratados em cada nível, ele também reduz o número de padrões que um dispositivo deve suportar.

\begin{comment}
\begin{itemize}
       \item Enviar: 

       Transferir vídeos ou imagens capturadas em uma câmera digital ou celular para um computador;
       \item Empurrar: 

       Exibir vídeos ou imagens capturadas em uma câmera digital ou celular diretamente em uma TV sem intermédio de um computador;
       \item Localizar e Reproduzir ou ``Reproduzir em...'': 

       Utiliza um celular para localizar uma música ou vídeo armazenados em um computador, unidade de disco externa ou um dispositivo \emph{Network-Attached Storage} (NAS) e transferi-lo, via \emph{stream} ou não, para reprodução;
       \item Puxar e imprimir ou ``Imprimir em...'': 

       Visualizar na TV uma foto armazenada em um servidor de mídia e imprimi-la utilizando uma impressora em rede.
\end{itemize}

\begin{table}
	\caption{Exemplos de casos de uso~\cite{dlnahdvideostreaming}.}
	\begin{center}
		\begin{tabular}{rl}
		\hline
		\textbf{Casos de Uso} & \textbf{Exemplo}																\\
		\hline
		Enviar & Transferir vídeos ou imagens capturadas em uma câmera digital ou celular para um computador.	\\
		\hline
		Empurrar & Exibir vídeos ou imagens capturadas em uma câmera digital ou celular diretamente em uma TV sem intermédio de um computador. \\
		\hline
		Localizar e Reproduzir ou ``Reproduzir em...'' & Utiliza um celular para localizar uma música ou vídeo armazenados em um computador, unidade de disco externa ou um dispositivo \emph{Network-Attached Storage} (NAS) e transferi-lo, via \emph{stream} ou não, para reprodução. \\
		\hline
		Puxar e imprimir ou ``Imprimir em...'' & Visualizar na TV uma foto armazenada em um servidor de mídia e imprimi-la utilizando uma impressora em rede. \\
		\hline
		\end{tabular}
	\end{center}
	\label{tab:casosdeuso_dlna}
\end{table}
\end{comment}

De volta ao exemplo apresentado no início deste capítulo, observa-se que as televisões citadas classificariam-se como dispositivos DMP e DMR.
\subsubsection{USB}

O \emph{Universal Serial Bus} (USB) é uma arquitetura de comunicação que adiciona a uma máquina hospedeira a capacidade de se interconectar a uma variedade de dispositivos. Surgiu com o intuito de sanar três das principais dificuldades enfrentadas à época de sua criação~\cite{usbspec}:

\begin{itemize}
	\item Integrar as plataformas das industrias da informática e comunicação. Para tal, a troca de informações entre esses dispositivos deveria acontecer de forma ubíqua e barata;
	\item Facilitar a reconfiguração de dispositivos de um computador, tais como teclados, mouses e \emph{joysticks};
	\item Integrar todas as interfaces existentes à época em um única que pudesse ser utilizada pela maior quantidade possível de dispositivos: telefone, fax, \emph{modem}, adaptadores, secretárias eletrônicas, \emph{scanners}, PDA's, teclados, \emph{mouses}, etc.
\end{itemize}

De acordo com sua especificação, suas máquinas hospedeiras devem:

\begin{itemize}
	\item Fornecer energia aos seus periféricos;
	\item Suportar todas as velocidades definidas (SuperSpeed, Low Speed, Full Speed e High SuperSpeed);
	\item Suportar todos os tipos de fluxo de dados definidos (controle, massa, interrupção e síncrono).
\end{itemize}

Entende-se por máquina hospedeira qualquer dispositivo possuidor dos recursos necessários para realizar esta função. Por exemplo, conectar uma câmera à uma impressora faz sentido, enquanto que conectar um GPS à mesma impressora, não. Assim sendo, cada máquina hospedeira não necessita se comportar exatamente como um computador, mas apenas servir como hospedeiro a certos dispositivos convenientes~\cite{usb3spec}.

Atualmente sua velocidade de comunicação varia entre 1.5 ou 12 megabits por segundo (mbs) e seu protocolo permite configurar dispositivos durante a fase de inicialização ou quando eles são plugados em tempo de execução~\cite{hid}, deixando toda parte pesada desse processo a cargo do hospedeiro~\cite{usb3spec}. Tais dispositivos são divididos em várias classes e/ou subclasses diferentes, cada qual definindo um comportamento e protocolos comuns para dispositivos que oferecem funções similares~\cite{hid}.

Um mesmo dispositivo USB pode pertencer a uma ou múltiplas classes. Por exemplo, um celular pode utilizar atributos da classe HID, Áudio e Comunicação. A partir dessas divisões será possível estabelecer uma hierarquia, a qual poderá ser usada no processo de classificação dos dispositivos presentes no smartspace.

Seguem abaixo as classes existentes juntamente com suas descrições~\cite{usbclasscodes}.

\begin{itemize}
	\item \emph{Audio}: 

	Destina-se a todos os dispositivos ou funções embutidos em aparelhos que são usados para manipular áudio, voz e quaisquer funcionalidades relacionadas. Isso inclui tanto dados de áudio (analógico e digital) quanto a capacidade de diretamente controlar tal ambiente por meio de controles de volume e tom. Não inclui a funcionalidade para operar mecanismos de transporte que estão relacionados com a reprodução de dados de áudio, tais como mecanismos de transporte de fita ou controles de unidade de CD-ROM~\cite{usbaudioclass}.

	Exemplos de tais dispositivos incluem \emph{headsets}, \emph{headphones} e \emph{microphones}~\cite{usbbasicaudioclass}.
	\item \emph{Communications and CDC Control}: 

	Há três classes que compõem a definição de dispositivos de comunicação: a Classe de Comunicação do Dispositivo, a Classe de Interface de Comunicação e a Classe de Interface de Dados. A primeira é uma definição de nível do dispositivo e é utilizada pelo hospedeiro para identificar corretamente um dispositivo de comunicação que pode apresentar diferentes tipos de interfaces. A segunda define um mecanismo de propósito geral que pode ser utilizado para habilitar todos os tipos de serviços de comunicação sobre o Universal Serial Bus (USB). A última define um mecanismo de propósito geral para permitir a transferência em massa quando os dados não atendem aos requisitos para qualquer outra classe~\cite{usbcommunicationclass}.

	Exemplos de tais dispositivos incluem:
	\begin{itemize}
		\item Telecomunicações: 

		Modens analógicos, adaptadores de terminais ISDN, telefones digitais e telefones analógicos;
		\item Dispositivos de rede: 

		Modens ADSL, modens à cabo, adaptadores/hubs \emph{Ethernet} 10BASE-T e \emph{Ethernet} cross-over cabos.
	\end{itemize}
	\item HID (\emph{Human Interface Device}): 

	Destina-se primordialmente aos dispositivos que são usados por humanos para controlar e operacionalizar quaisquer tipos de sistemas computacionais~\cite{hid}. Mais específicamente, possui as seguintes particularidades:
	\begin{itemize}
		\item Deve ser tão compacto quanto possível para evitar desperdício de espaço no dispositivo;
		\item Permitir a aplicação de software para evitar informação desconhecida;
		\item Ser extensível e robusto;
		\item Suportar hierarquias e coleções;
		\item Ser autodescritivo para permitir aplicações de software genéricos.
	\end{itemize}

	Exemplos de tais dispositivos incluem: teclados e dispositivos apontadores, painéis de controle, controles que podem ser encontrados em dispositivos como telefones, controles remotos, jogos ou dispositivos de simulação e aparelhos que não necessitam de interação humana mas que fornecem dados de forma similar aos dipositivos da classe HID~\cite{hid}.
	\item \emph{Physical}: 

	É vista como uma extensão da classe HID para dispostivos que requerem resposta física em "tempo real". Seu foco principal está em dispositivos táteis e na implementação de sistemas com resposta à força. Contúdo, não há exigência de que os membros desta classe gerem esse tipo de efeito~\cite{usbphysicalclass}.

	Exemplos de tais dispositivos incluem: \emph{joysticks} e plataformas de movimento.
	\item \emph{Image}: 

	Destina-se aos dispositivos de captura de imagem~\cite{usbimageclass}.

	Exemplos de tais dispositivos incluem: câmeras digitais e aparelhos similares.
	\item \emph{Printer}: 

	Destina-se aos dispositivos de impressão. Estes, anteriormente, eram interfaceados por meio das seguintes tecnologias~\cite{usbprintclass}:
	\begin{itemize}
		\item Porta paralela unidirecional;
		\item Porta paralela bi-direcional;
		\item Porta serial;
		\item Porta SCSI;
		\item \emph{Ethernet} / LAN.
	\end{itemize}

	Atualmente existem outras interfaces mais sofisticadas, mas essas são as mais comuns.

	O USB oferece uma capacidade de processamento muito maior que a porta serial e é comparável em velocidade à porta paralela~\cite{usbprintclass}.
	\item \emph{Mass Storage}: 

	Destina-se aos dispositivos de armazenamento em massa, ou seja, todo aparelho que possa ler ou gravar dados digitais em algum sistema de armazenamento.

	Exemplos de tais dispositivos incluem: pendrives, discos de estado sólido com memória flash não volátil e HD's.
	\item \emph{Hub}: 
	\item \emph{Smart Card}: 

	Destina-se aos dispositivos compostos por cartões de circuitos integrados~\cite{usbsmartcard}.

	Exemplos de tais dispositivos incluem: cartões de crédito, CPF's digitais, etc.
	\item \emph{Content Security}: 

	Destina-se primordialmente em proteger e controlar a distribuição de conteúdo digital. Especifica um conjunto comum de requisições de transporte de dados e descritores necessários para apoiar os vários métodos de segurança de conteúdo~\cite{usbcontentsecurityclass}.
	\item \emph{Video}: 

	Destina-se a todos os dispositivos ou funções que compõem outros aparelhos utilizados para manipular vídeos~\cite{usbvideoclass}.

	Exemplos de tais dispositivos incluem: \emph{webcams}, filmadoras digitais, conversores analógicos de vídeo, sintonizadores de televisão analógica e digital e imagens de câmeras que suportam \emph{streaming} de vídeo~\cite{usbvideoclass}.
	\item \emph{Personal Healthcare}: 

	Destina-se em permitir a interoperabilidade entre aparelhos da saúde pessoal e hospedeiros USB, o que permite uma maior variedade de casos de uso para ambos os dispositivos e hospedeiros~\cite{usbhealthcareclass}.

	Exemplos de tais dispositivos incluem: medidores de glicose e oxímetros de pulso~\cite{usbhealthcareclass}.
	\item \emph{Diagnostic Device}:

	Destina-se aos dispositivos que diagnosticam outros dispositivos~\cite{usbclasscodes}.
	\item \emph{Wireless Controller}: 

	Destina-se aos dispositivos controladores \emph{wireless}~\cite{usbclasscodes}.

	Exemplos de tais dispositivos incluem: aparelhos \emph{bluetooth} e via rádio.
	\item \emph{Miscellaneous}:

	Destina-se à dispositivos diversos~\cite{usbclasscodes}.
	\item \emph{Application Specific}:

	Destina-se aos dispositivos que estejam em conformidade com as especificações da várias classes USB. Define um conjunto utilizável de subclasses e protocolos~\cite{usbclasscodes}.
	\item \emph{Vendor Specific}:

	Destina-se à utilização por parte dos fornecedores, que podem usá-la como bem entenderem~\cite{usbclasscodes}.
\end{itemize}

\subsection{\emph{Bluetooth}}

Destina-se a substituir os cabos de conexão de dispositivos fixos e/ou portáteis mantendo altos níveis de segurança. Sua força fundamental está na capacidade de lidar simultaneamente com dados e transmissão de voz, o que fornece aos usuários uma variedade de soluções inovadoras, tais como fones de ouvido \emph{hands-free} para chamadas de voz, impressão, fax, sincronização com PCs e telefones celulares, entre outros~\cite{bluetoothoverview}.

É uma tecnologia de comunicação de curto alcance. Suas principais características são a robustez, a baixa potência e o baixo custo. Sua especificação define uma estrutura uniforme para que uma ampla gama de dispositivos possam se conectar e se comunicar uns com os outros~\cite{bluetoothoverview}.

Sua estrutura e aceitação global permitem que qualquer dispositivo \emph{bluetooth} ativado possa se conectar a outros dispositivos \emph{bluetooth} situados em sua proximidade. Uma vez estabelecida, uma conexão \emph{bluetooth} permite que aparelhos a uma curta distância se comuniquem sem a utilização de fios, ou seja, por meio de redes ad hoc conhecidas como piconets. Estas são estabelecidas de forma dinâmica e automatica assim que um dispositivo entra ou sai do seu raio de alcance, facilitando o processo de conexão por parte do usuário~\cite{bluetoothoverview}.

Cada dispositivo em uma piconet pode pertencer a várias outras piconets e, simultaneamente, se comunicar com até sete outros dispositivos presentes em uma mesma rede~\cite{bluetoothoverview}.
\subsection{Device Description Language (DDL)}
\label{subsec:ddl}

Provê um esquema capaz de descrever a interface dos dispositivos e um processador da linguagem para converter DDL para pacote de serviços da \emph{Open Services Gateway initiative} (OSGi) que separa as responsabilidades entre fabricantes de dispositivos, integradores de sistemas e programadores de aplicativos~\cite{gatorTechDDL}. Inicialmente esses serviços eram representados em classes Java. Entretanto, a criação desses pacotes não acontecia de forma automática. 

Para cada modelo de sensores e atuadores, era necessário ler a especificação do fabricante, examinar a interface e estudar os protocolos de comunicação. Era necessário ainda, que quem escrevesse o pacote fosse especialista na programação Java e no \emph{framework} OSGi. Com o objetivo de resolver essa complexidade o grupo de pesquisa do Atlas desenvolveu a \emph{Device Description Language}(DDL).

\begin{figure}[ht]
\center
\includegraphics[scale=0.4]{imagens/gatorDDL}
\caption{Caracterização de um dispositivo~\cite{ddlSpec}}
\label{fig:ddlspec}
\end{figure}

Na DDL, um dispositivo é caracterizado como uma entidade com propriedades, mecanismo interno e uma interface, como mostra a Figura~\ref{fig:ddlspec}. As propriedades provêem informações a respeito do dispositivo como seu propósito, suas capacidades, seu fabricante e seus requisitos operacionais. Essas informações são críticas para a integração do sistema e para programadores de serviços. O mecanismo interno é responsável pela operação do dispositivo e é desconhecido do mundo externo. A interface do dispositivo é a ponte entre o hiato do mecanismo interno e do mundo externo. Ela especifica a entrada e saída do dispositivo e provê um guia para aplicações e outros serviços interagirem com o aparelho. A classificação possui flexibilidade para ser extendida e a partir de então estabelecer uma relação hierárquica entre os dispositivos.

Os dispositivos estão classificados em três categorias:
\begin{itemize}
	\item Sensor: apenas provê dados de entrada para o usuário externo.
	\item Atuador: apenas aceita dados de saída do usuário externo.
	\item Dispositivo complexo: provê dados de entrada e aceita dados de saída do usuário externo.
\end{itemize}

Essa classificação, portanto, se apresenta de forma bastante genérica, pois um dispositivo pode se enquadrar como sensor se limitando a fornecer dados de entrada, como atuador, se limitando a receber dados e tomar ações, ou então será classificado como um dispositivo complexo. Logo a categoria de dispositivo complexo poderá ser utilizada por diferentes tipos de recursos, necessitando conhecer as operações definidas no recurso para, a partir dos seviços providos pelo dispositivo, poder inferir que tipo de recurso é aquele. 


\subsection{Estado da Arte}
Nesta seção, apresentaremos alguns projetos de computação ubíqua e mostraremos como esses projetos tratam a classificação de recursos.

\subsubsection{Gaia}
No projeto Gaia foi criado um middleware (Gaia OS) com o objetivo de dar suporte ao desenvolvimento e execução de aplicações em \emph{Active Spaces}, ambientes com sistemas interativos. O middleware, uma abstração de um sistema operacional, foi projetado para ser uma infraestrutura distribuiída que coordena entidades de software e dispositivos heterogêneos em um \emph{smart space}. O Gaia OS expõe serviços para buscar e utilizar recursos presentes no ambiente, ter conhecimento do contexto e provê um \emph{framework} para desenvolver aplicações móveis sensíveis ao contexto, que conheçam os recursos disponíveis, utilizem múltiplos dispositivos e tenham como foco o usuário~\cite{gaia2002}.

\emph{Active Spaces} são espaços físicos como escritórios, salas de conferência, casas, hospitais, campi universitários, cidades que possuem dispositivos integrados ao ambiente. O objetivo desses dispositivos é prover e obter informação sobre usuários do ambiente, os ajudando a realizar tarefas que eles não poderiam sem os dispositivos, ou facilitando tarefas do cotidiano.

\begin{figure}[ht]
\center
\includegraphics[scale=0.8]{imagens/gaia-devices}
\caption{Diagrama de Classes Simplificado~\cite{gaiaDevices}}
\label{fig:gaiaClassDiagram}
\end{figure}

No projeto, foi desenvolvido um \emph{framework} para a interação entre dispositivos heterogêneos. Esse \emph{framework} permite a representação das interfaces dos dispositivos com diferentes níveis de detalhe e especialização. As interfaces são definidas utilizando IDL(\emph{Interface Description Language}), que permite a construção de \emph{drivers} de dispositivos em qualquer linguagem de programação garantindo uma facilidade de integração com diferentes dispositivos.

A figura~\ref{fig:gaiaClassDiagram} mostra o Diagrama de Classes simplificado do projeto Gaia. Podemos observar que a Classe \emph{Device} é especializada em dispositivos de entrada(\emph{InDevice}) e saída(\emph{OutDevice}) de dados. Os dispositivos de entrada são ainda especializados em: Microfone, Câmera, Teclado e Sensor, enquanto os dispositivos de saída são especializados em: Impressora e Monitor, que por sua vez, é especializado em Projetor. Há ainda a interface para dispositivos que são de entrada e saída que é especializada em uma unidade de Fita.

\begin{comment}
http://gaia.cs.uiuc.edu/html/device.htm
http://gaia.cs.uiuc.edu/papers/GaiaSubmitted3.pdf
\end{comment}

\subsection{Amigo}
O projeto Amigo (\emph{Ambient Intelligence for the Networked home environment}) desenvolveu um \emph{middleware} com arquitetura baseada em SOA que integra dinamicamente sistemas heterogêneos para alcançar a interoperabilidade entre serviços e dispositivos. O \emph{middleware} provê a semântica para comunicação e descoberta de dispositivos e serviços disponíveis no ambiente, incluindo dispositivos que utilizam padrões para descoberta, como o UPnP, integrando dispositivos móveis, computadores pessoais, eletrodomésticos e dispositivos de automação residencial~\cite{amigoArch}.

\begin{figure}[ht]
\center
\includegraphics[scale=0.5]{imagens/amigo-interfaces}
\caption{Implementação de um \emph{AmigoService} utilizando serviços UPnP~\cite{amigoCore}}
\label{fig:amigoInterfaces}
\end{figure}

Além de utilizar o UPnP para a descoberta de dispositivos, o Amigo é compatível com as classificações de dispositivos do UPnP. Quando um dispositivo UPnP é encontrado, é criada uma instância de um \emph{UPnPDevice} e o \emph{driver AmigoUPnP} é notificado e executa o método \emph{getServices} do \emph{UPnPDevice} e cria para cada serviço uma instância do \emph{UPnPAmigoService}. A figura~\ref{fig:amigoInterfaces} mostra o relacionamento entre as interfaces UPnP com a implementação de uma interface de um Serviço do Amigo.

Os serviços do Amigo são modelados em uma Ontologia que é utilizada para comparar serviços e decidir se eles são equivalentes. A classe central da Ontologia é o Componente que representa o dispositivo que provê o serviço. Para representar o que o dispositivo requer e provê, foi introduzido o conceito da Capacidade, dividida em Capacidade Requerida e Capacidade Provida. Uma Capacidade possui parâmetros de entrada e saída que também são modelados em classes. As capacidades são então associadas à Conversas suportadas pelo Componente e relacionadas à mensagens que são empregadas na Conversa associada como mostra a figura ~\ref{fig:amigoServiceOntology}.

\begin{figure}[ht]
\center
\includegraphics[scale=0.5]{imagens/amigo-ontology}
\caption{Elementos básicos da Ontologia de Serviços~\cite{amigoCore}}
\label{fig:amigoServiceOntology}
\end{figure}


\begin{comment}
http://www.hitech-projects.com/euprojects/amigo/publications/IST-004182%20Amigo-IP%20short%20project%20description.pdf
http://www.hitech-projects.com/euprojects/amigo/deliverables/Deliverable%20D1.2-VolII_SOTA_v10_final.pdf
http://www.hitech-projects.com/euprojects/amigo/deliverables/Amigo_WP2_D2.1_v10%20final.pdf
http://www.hitech-projects.com/euprojects/amigo/deliverables/Amigo_WP3_D31b_v1.0.pdf
\end{comment}

\subsection{\emph{Gator Tech}}

O projeto \emph{Gator Tech Smart House} é o resultado de mais de cinco anos de pesquisa na área de computação pervasiva e móvel. O objetivo do projeto é criar ambientes assistivos como casas que terão conhecimentos sobre si e sobre seus residentes criando um mapeamento entre o mundo físico, monitoramento remoto e serviços de intervenção~\cite{gatorTech}.

Neste projeto foi desenvolvida uma arquitetura para plataforma de sensores orientada à serviços, o Atlas. A plataforma Atlas é uma combinação de nós de \emph{hardware} e \emph{firmware} executado em um \emph{hardware} e um \emph{middleware} executando na rede, que provê serviços em um ambiente. Juntos, esses componentes permitem que qualquer sensor, atuador ou qualquer outro dispositivo sejam integrados e controlados por meio da interface de um determinado dispositivo. Essa abordagem facilita o desenvolvimento de aplicações que utilizam esses dispositivos~\cite{gatorTechLessons}. 

O Atlas é responsável por obter a representação de serviços dos dispositivos conectados e gerenciar os serviços de modo que as aplicações possam obter e utilizar os serviços facilmente. Na implementação do \emph{Gator Tech}, a camada de serviços foi construída sobre o \emph{framework} OSGi que mantém o registro dos serviços de todos os nós conectados. Cada sensor ou atuador é representado no \emph{middleware} Atlas como um pacote de serviços OSGi. Inicialmente esses serviços eram representados em classes Java. Entretanto, a criação desses pacotes não acontecia de forma automática. 

\begin{figure}[ht]
\center
\includegraphics[scale=0.4]{imagens/gatorDDL}
\caption{Caracterização de um Dispositivo~\cite{ddlSpec}}
\label{fig:ddlspec}
\end{figure}

Para cada modelo de sensores e atuadores, era necessário ler a especificação do fabricante, examinar a interface e estudar os protocolos de comunicação. Era necessário ainda, que quem escrevesse o pacote fosse especialista na programação Java e no \emph{framework} OSGi. Com o objetivo de resolver essa complexidade o grupo de pesquisa do Atlas desenvolveu a \emph{Device Description Language}(DDL). Essa linguagem provê um esquema capaz de descrever a interface dos dispositivos e um processador da linguagem para converter DDL para pacote de serviços OSGi que separa as responsabilidades entre fabricantes de dispositivos, integradores de sistemas e programadores de aplicativos~\cite{gatorTechDDL}.

Na DDL, um dispositivo é caracterizado como uma entidade com propriedades, mecanismo interno e uma interface. Como mostra a figura~\ref{fig:ddlspec}. As propriedades proveem informações a respeito do dispositivo com seu propósito, suas capacidades, fabricante e requisitos operacionais. Essas informações são críticas para a integração do sistema e para programadores de serviços. O mecanismo interno é responsável pela operação do dispositivo e é desconhecido do mundo externo. A interface do dispositivo é a ponte entre o hiato entre o mecanismo interno e o mundo externo. Ela especifica a entrada e saída do dispositivo e provê um guia para aplicações e outros serviços interagirem com o dispositivo.

Os dispositivos foram classificados em três categorias:
\begin{itemize}
	\item Sensor: apenas provê dados de entrada para o usuário externo.
	\item Atuador: apenas aceita dados de saída do usuário externo.
	\item Dispositivo Complexo: provê dados de entrada e aceita dados de saída do usuário externo.
\end{itemize}

Essa classificação genérica tem a desvantagem de ser difícil de ser reutilizada e especializada por diferentes dispositivos, sendo que seus recursos são expostos na forma de serviços por meio da especificação de operações~\cite{ddlSpec}.


\subsection{Comparativo}
%Apresentar um Comparativo entre as estratégias e conclusões que embasem a definição da proposição de vcs no capítulo seguinte.
Nesta seção faremos um comparativo dos padrões mostrados de forma a embasar a classificação de recursos que será proposta no próximo capítulo. A tabela~\ref{tab:comparativo} mostra uma comparação dos padrões quanto: à forma de classificação de recursos, à representação e extensibilidade dessa classificação e se um dispositivo pode fazer parte de mais de uma classe.

\begin{table}
	\caption{Comparativo dos Padrões.}
	\begin{center}
	\resizebox{16cm}{!} {
		\begin{tabular}{ccccccc}
		\hline
							& \textbf{IEEE 1451}	& \textbf{Bluetooth} 	& \textbf{USB}	& \textbf{UPnP} & \textbf{DLNA} & \textbf{DDL}\\
		\hline
		Classificação 		& Relacionada 			& Fixa 					& Relacionada 	& Relacionada 	& Relacionada 	& Fixa \\
		\hline
		Representação 		& IDL 					& Descritor				& Descritor		& XML			& XML 			& XML \\ 
		\hline
		Extensível 			& Não 					& Não 					& Sim 			& Sim 			& Sim 			& Não \\
		\hline
		Múltiplas Classes 	& Sim 					& Não					& Sim 			& Não 			& Sim 			& Não \\
		\hline
		\end{tabular}
	}
	\end{center}
	\label{tab:comparativo}
\end{table}
 
Como pode ser observado, os padrões existentes para a classificação de recursos mostrados neste trabalho são todos estruturados, ou seja, não foi utilizada uma forma dinâmica como uma ontologia para representar essa classificação. A tabela~\ref{tab:comparativo} mostra que a maioria dos padrões faz uso de uma classificação relacionada, pois possui a vantagem de aproveitar propriedades já definidas. Além disso, é comum a representação da classificação em arquivos XML, que embora seja um padrão independente de plataforma, tem a desvantagem de possuir uma relevante redundância de dados na sua formatação. Apesar da classificação de alguns padrões possuir a capacidade de ser extensível, é necessário que a nova classe criada pelos fabricantes de um dispositivo passe por uma homologação. Caso do UPnP. O USB, possui subclasses derivadas na sua especificação, entretanto não permite novas definições fora do padrão. Em geral, são pouco flexíveis. 

A tabela mostra ainda uma limitação da DDL (\emph{Device Description Language}), criada para reduzir a complexidade na definição de dispositivos via classes Java, pois, para cada dispositivo que não seja sensor ou atuador, se faz necessária a criação de uma nova classe para esse dispositivo complexo não relacionada com outras que já tenham sido criadas. Outro fator que impede o uso de subclasses é o fato de na definição do dispositivo, estarem definidos também seus serviços, modelados como operações. Se um dispositivo possuir apenas uma operação que receba parâmetros diferentes, por exemplo, a classe anteriormente definida não poderá ser reutilizada. 

\begin{table}
	\caption{Classes presentes nos padrões.}
	\begin{center}
		\begin{tabular}{|ccccc|}
		\hline
									& \textbf{Bluetooth} 	& \textbf{USB}	& \textbf{UPnP} & \textbf{DLNA}	\\
		\hline
		Áudio						& x						& x				& x 			& x				\\
		\hline
		Vídeo						& x						& x				& x				& x				\\
		\hline
		Servidor de mídia			& x						&				& x 			& x				\\
		\hline
		Tocador de mídia			& x						&				& x				& x				\\
		\hline
		Impressora 					& x						& x				& x				& x				\\
		\hline
		Imagem	 					& x						& x				& x				& x				\\
		\hline
		Scanner						& 						& x				& x				& 				\\
		\hline
		Armazenamento				&						& x				& 				& 				\\	
		\hline
		Telefone/PDAs				& x						& x				& x				& x				\\
		\hline
		Teclado						& x						& x				& 				& x				\\
		\hline
		Apontadores					& x						& x				& 				& x 			\\
		\hline
		Interface Humana		 	& x						& x				&  				&  				\\
		\hline
		Específico do fabricante 	& x 					& x				& x				& x				\\
		\hline								
		\end{tabular}
	\end{center}
	\label{tab:comparativoClasses}
\end{table}

A tabela~\ref{tab:comparativoClasses} mostra os principais dispositivos que possuem alguma classe associada nos padrões estudados. O padrão DDL não foi considerado no comparativo de classes devido sua classificação possuir apenas três categorias: sensores, atuadores e dispositivos complexos. Portanto, todos os dispositivos considerados neste comparativo poderiam se encaixar na última categoria. O padrão IEEE 1451 também não foi utilizado no comparativo, pois foi criado para facilitar a comunicação de transdutores, e suas classes não abrangem os dispositivos da tabela.

