\subsubsection{\emph{Bluetooth}}

Destina-se a substituir os cabos de conexão de dispositivos fixos e/ou portáteis mantendo altos níveis de segurança. Sua força fundamental está na capacidade de lidar simultaneamente com dados e transmissão de voz, o que fornece aos usuários uma variedade de soluções inovadoras, tais como fones de ouvido \emph{hands-free} para chamadas de voz, impressão, fax, sincronização com PCs e telefones celulares, entre outros~\cite{bluetoothoverview}.

É uma tecnologia de comunicação de curto alcance. Suas principais características são a robustez, a baixa potência e o baixo custo. Sua especificação define uma estrutura uniforme para que uma ampla gama de dispositivos possam se conectar e se comunicar uns com os outros~\cite{bluetoothoverview}.

Sua estrutura e aceitação global permitem que qualquer dispositivo \emph{bluetooth} ativado possa se conectar a outros dispositivos \emph{bluetooth} situados em sua proximidade. Para isso, cada dispositivo deve ser capaz de interpretar certos perfis, ou seja, definições de aplicações possíveis que especificam comportamentos gerais. No mínimo, cada perfil contém informações sobre os seguines tópicos~\cite{bluetoothprofiles}:

\begin{itemize}
	\item Dependências em outros perfis;
	\item Formatos de interface de usuário sugeridos;
	\item Partes específicas da pilha de protocolos \emph{bluetooth} utilizadas pelo perfil. Para executar sua tarefa, cada perfil utiliza opções específicas e parâmetros em cada camada da pilha e isso pode incluir, se for o caso, um esboço do registro de serviço exigido.
\end{itemize}

Seguem abaixo os perfis atualmente disponíveis~\cite{bluetoothprofiles}:

\begin{itemize}
	\item \emph{Advanced Audio Distribution Profile} (A2DP): 

	Define como um áudio estéreo pode ser transmitido de uma fonte de mídia para um dispositivo de armazenamento.

	Exemplos de tais dispositivos incluem: fones de ouvido, alto-falantes e adaptadores estéreos e reprodutores de MP3~\cite{bluetoothprofilesA2DP}.
	\item \emph{Audio/Video Remote Control Profile} (AVRCP): 

	Destina-se em fornecer uma interface padrão para controlar TV's, equipamentos de áudio estéreo ou outros dispositivos de áudio e vídeo. Permite que um único controle remoto (ou outro dispositivo) controle todos os equipamentos de áudio e vídeo que o usúario tenha acesso.

	Exemplos de tais dispositivos incluem: PC's, PDA's, celulares, controles remotos, fones de ouvido, reprodutores e gravadores de áudio e vídeo, monitores, TV's, sintonizadores e amplificadores~\cite{bluetoothprofilesAVRCP}.
	\item \emph{Basic Imaging Profile} (BIP): 

	Define como um dispositivo de imagem pode ser controlado remotamente, como um dispositivo de imagem pode imprimir e como um dispositivo de imagem pode transferir imagens para um dispositivo de armazenamento.

	Exemplos de tais dispositivos incluem: câmeras digitais, PC's, celulares, impressoras e PDA's~\cite{bluetoothprofilesBIP}.
	\item \emph{Basic Printing Profile} (BPP): 

	Permite que os dispositivos enviem texto, emails, \emph{v-cards}, imagens ou outras informações para impressoras com base em trabalhos de impressão.
	
	Exemplos de tais dispositivos incluem: impressoras, PC's, celulares e PDA's~\cite{bluetoothprofilesBPP}.
	\item \emph{Common ISDN Access Profile} (CIP): 

	Define como uma sinalização de uma Rede Digital de Serviços Integrados (ISDN, em inglês) pode ser transferida através de uma conexão \emph{bluetooth}.

	Exemplos de tais dispositivos incluem: \emph{laptops}, pontos de acesso, PC's e celulares~\cite{bluetoothprofilesCIP}.
	\item \emph{Cordless Telephony Profile} (CTP): 

	Define como um telefone sem fio pode ser implementado sobre um \emph{link} \emph{bluetooth}.

	Exemplos de tais dispositivos incluem: \emph{laptops}, PC's, telefones sem fios, celulares e PDA's~\cite{bluetoothprofilesCTP}.
	\item \emph{Dial-Up Network Profile} (DUN): 

	Fornece um padrão para acessar a internet e outros serviços \emph{dial-up} via \emph{bluetooth}.

	Exemplos de tais dispositivos incluem: \emph{laptops}, PC's, celulares, PDA's e modems~\cite{bluetoothprofilesDUN}.
	\item \emph{Fax Profile} (FAX): 

	Define como um dispositivo de FAX pode ser usado por um dispositivo de terminal.

	Exemplos de tais dispositivos incluem: \emph{laptops}, PC's, celulares, PDA's e modems~\cite{bluetoothprofilesFAX}.
	\item \emph{File Transfer Profile} (FTP): 

	Define como as pastas e os arquivos em um dispositivo servidor podem ser visitados por um dispositivo cliente.

	Exemplos de tais dispositivos incluem: \emph{laptops}, PC's, celulares e PDA's~\cite{bluetoothprofilesFTP}.
	\item \emph{General Audio/Video Distribution Profile} (GAVDP): 

	Fornece a base para o A2DP e o VDP, que são a base dos sistemas de distribuição projetados para fluxos de vídeo e áudio usando a tecnologia \emph{bluetooth}.

	Exemplos de tais dispositivos incluem: reprodutores de música, fones de ouvido e alto-falantes estéreos, \emph{laptops}, PC's, celulares e PDA's~\cite{bluetoothprofilesGAVDP}.
	\item \emph{Generic Object Profile} (GOEP): 

	Utilizado para transferir um objeto de um dispositivo para outro.

	Exemplos de tais dispositivos incluem: \emph{laptops}, PC's, celulares, PDA's e visualizadores de mídia~\cite{bluetoothprofilesGOEP}.
	\item \emph{Hands-Free Profile} (HFP): 

	Descreve como um dispositivo de \emph{gateway} pode ser usado para fazer e receber chamadas para um dispositivo \emph{hands-free}.

	Exemplos de tais dispositivos incluem: carros, kits automotivos, sistemas GPS, fones de ouvido, celulares e PDA's~\cite{bluetoothprofilesHFP}.
	\item \emph{Hard Copy Cable Replacement Profile} (HCRP): 

	Define como uma impressão baseada em driver é realizada através de uma conexão \emph{bluetooth}.

	Exemplos de tais dispositivos incluem: impressoras, PC's e \emph{laptops}~\cite{bluetoothprofilesHCRP}.
	\item \emph{Headset Profile} (HSP): 

	Descreve como um fone de ouvido deve se comunicar com um dispositivo habilitado para \emph{bluetooth}.

	Exemplos de tais dispositivos incluem: fones de ouvido, celulares, PDA's, PC's e \emph{laptops}~\cite{bluetoothprofilesHSP}.
	\item \emph{Human Interface Device Profile} (HID): 

	Define os protocolos, procedimentos e recursos a serem utilizados por teclados \emph{bluetooth}, \emph{mouses}, dispositivos de jogos e apontandores e dispositivos de monitoramento remoto.

	Exemplos de tais dispositivos incluem: teclados, \emph{mouses}, dispositivos de jogos, \emph{tablets}, PC's, \emph{laptops}, celulares e PDA's~\cite{bluetoothprofilesHID}.
	\item \emph{Intercom Profile} (ICP): 

	Define como dois celulares \emph{bluetooth} em uma mesma rede podem se comunicar diretamente sem usar o telefone público ou rede celular.

	Exemplos de tais dispositivos incluem: celulares, PC's e \emph{laptops}~\cite{bluetoothprofilesICP}.
	\item \emph{Object Push Profile} (OPP): 

	Define as regras dos servidores e clientes \emph{push}.

	Exemplos de tais dispositivos incluem: celulares, PC's e \emph{laptops}~\cite{bluetoothprofilesOPP}.
	\item \emph{Personal Area Networking Profile} (PAN): 

	Descreve como dois ou mais dispositivos \emph{bluetooth} podem formar uma rede ad-hoc e como este mecanismo pode ser usado para acessar uma rede remota através de um ponto de acesso à rede.

	Exemplos de tais dispositivos incluem: celulares, PC's e \emph{laptops}~\cite{bluetoothprofilesPAN}.
	\item \emph{Service Discovery Application Profile} (SDAP): 

	Descreve como um aplicativo deve usar o Protocolo de Descrição de Sessão (SDP, em inglês) para descobrir serviços em um dispositivo remoto.

	Exemplos de tais dispositivos incluem: PC's, \emph{laptops}, celulares, PDA's, impressoras, FAX e fones de ouvido~\cite{bluetoothprofilesSDAP}.
	\item \emph{Serial Port Profile} (SPP): 

	Define a forma de configurar portas seriais virtuais e conectar dois dispositivos \emph{bluetooth}.

	Exemplos de tais dispositivos incluem: PC's, \emph{laptops}~\cite{bluetoothprofilesSPP}.
	\item \emph{Synchronization Profile} (SYNC): 

	Utilizado em conjunto com o GOEP para permitir a sincronização de calendários e informações de endereço entre dispositivos \emph{bluetooth}.

	Exemplos de tais dispositivos incluem: PC's, \emph{laptops}, celulares e PDA's~\cite{bluetoothprofilesSYNC}.
	\item \emph{Video Distribution Profile} (VDP): 

	Define como um dispositivo transmite um vídeo através de uma conexão \emph{bluetooth}.

	Exemplos de tais dispositivos incluem: PC's, reprodutores digitais portáteis, câmeras de vídeo, TV's e monitores~\cite{bluetoothprofilesVDP}.
\end{itemize}

Uma vez estabelecida, uma conexão \emph{bluetooth} permite que aparelhos a uma curta distância se comuniquem sem a utilização de fios, ou seja, por meio de redes ad hoc conhecidas como piconets. Estas são estabelecidas de forma dinâmica e automatica assim que um dispositivo entra ou sai do seu raio de alcance, facilitando o processo de conexão por parte do usuário~\cite{bluetoothoverview}.

Cada dispositivo em uma piconet pode pertencer a várias outras piconets e, simultaneamente, se comunicar com até sete outros dispositivos presentes em uma mesma rede~\cite{bluetoothoverview}.

