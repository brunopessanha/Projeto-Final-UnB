\subsection{USB}
O \emph{Universal Serial Bus} (USB) é uma arquitetura de comunicaçao que adiciona a um computador pessoal (PC) a capacidade de se interconectar a uma variedade de dispositivos. Atualmente sua velocidade de comunicação varia entre 1.5 ou 12 megabits por segundo (mbs) e seu protocolo permite configurar dispositivos durante a fase de inicialização ou quando eles são plugados em tempo de execução. Esses dispositivos são divididos em várias classes diferentes, cada qual definindo um comportamento e protocolos comuns para dispositivos que oferecem funções similares.~\cite{USB}

Tabela com exemplos de classes de dispositivos USB.

Dentre as diversas classes existentes para o protocolo USB, será utilizada a \emph{Human Interface Device} (HID). Tal decisão baseou-se na característica principal desse tipo de classe que, de forma geral, destina-se primordialmente aos dispositivos que são usados por humanos para controlar e operacionalizar quaisquer tipos de sistemas computacionais.~\cite{USB}. Mais específicamente, possui as seguintes particularidades:

\begin{itemize}
\item Deve ser tão compacto quanto possível para evitar desperdício de espaço no dispositivo;
\item Permitir a aplicação de software para evitar informação desconhecida;
\item Ser extensível e robusto;
\item Suportar hierarquias e coleções;
\item Ser autodescritivo para permitir aplicações de software genéricos.
\end{itemize}

Tabela com exemplos de dispositivos HID.