\subsection{USB}

O \emph{Universal Serial Bus} (USB) é uma arquitetura de comunicaçao que adiciona a um computador pessoal (PC) a capacidade de se interconectar a uma variedade de dispositivos. Surgiu com o intuito de sanar três das principais dificuldades enfrentadas à época de sua criação~\cite{usbspec}:

\begin{itemize}
	\item Integrar as plataformas das industrias da informática e comunicação. Para tal, a troca de informações entre esses dispositivos deveria acontecer de forma ubíqua e barata;
	\item Facilitar a reconfiguração de dispositivos de um computador, tais como teclados, mouses e joysticks;
	\item Integrar todas as interfaces existentes à época em um única que pudesse ser utilizada pela maior quantidade possível de dispositivos: telefone, fax, modem, adaptadores, secretárias eletrônicas, \emph{scanners}, PDA's, teclados, \emph{mouses}, etc.
\end{itemize}

Atualmente sua velocidade de comunicação varia entre 1.5 ou 12 megabits por segundo (mbs) e seu protocolo permite configurar dispositivos durante a fase de inicialização ou quando eles são plugados em tempo de execução. Esses dispositivos são divididos em várias classes e/ou subclasses diferentes (veja em~\ref{tab:dispositivos_usb}), cada qual definindo um comportamento e protocolos comuns para dispositivos que oferecem funções similares~\cite{hid}. A partir dessas classificações será possível estabelecer uma hierarquia, a qual poderá ser usada no processo de classificação dos dispositivos presentes no smartspace.

\subsubsection{SuperSpeed}
\subsubsection{Wireless}
\subsubsection{Hi-Speed}
\subsubsection{On-The-Go and Embedded Host}

\begin{table}
	\begin{center}
		\begin{tabular}{cccc}
		\hline
		\textbf{Classes de Dispositivos} \\
		\hline
		Audio & Comunicação & Dispositivo Interface Humana & Físico \\
		Imagem & Impressora & Disco Rígido & Hub \\
		\hline
		\end{tabular}
	\end{center}
	\caption{Exemplos de classes de dispositivos USB.}
	\label{tab:dispositivos_usb}
\end{table}

Dentre as diversas classes existentes para o protocolo USB, será utilizada a \emph{Human Interface Device} (HID). Tal decisão baseou-se na característica principal desse tipo de classe que, de forma geral, destina-se primordialmente aos dispositivos que são usados por humanos para controlar e operacionalizar quaisquer tipos de sistemas computacionais~\cite{hid}. Mais específicamente, possui as seguintes particularidades:

\begin{itemize}
	\item Deve ser tão compacto quanto possível para evitar desperdício de espaço no dispositivo;
	\item Permitir a aplicação de software para evitar informação desconhecida;
	\item Ser extensível e robusto;
	\item Suportar hierarquias e coleções;
	\item Ser autodescritivo para permitir aplicações de software genéricos.
\end{itemize}

\begin{table}
	\begin{center}
		\begin{tabular}{lr}
		\hline
		\textbf{Dispositivo} & \textbf{Exemplo Descritivo} \\
		\hline
		Teclados e dispositivos apontadores & Mouses, trackballs e joysticks \\
		Painéis de controle & Maçanetas, Interruptores, botões e alavancas \\
		Controles que podem ser encontrados \\
		em dispositivos como telefones,\\
		controles remotos, jogos ou dispositivos \\
		de simulação & Luvas virtuais, câmbios, \\
		volantes e pedais \\
		Dispositivos que não necessitam de \\
		interação humana mas que fornecem dados \\
		de forma similar aos dipositivos da \\
		classe HID & Leitores de códigos de barra, \\termômetros ou voltímetros. \\
		\hline
		\end{tabular}
	\end{center}
	\caption{Exemplos de classes de dispositivos USB.}
	\label{tab:dispositivos_USB}
\end{table}