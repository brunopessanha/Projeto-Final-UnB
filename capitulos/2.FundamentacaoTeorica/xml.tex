\subsection{XML}
O XML (\emph{Extensible Markup Language}), como seu próprio nome sugere, pode ser estendido e dessa forma, é possível montar uma hierarquia de arquivos XML. O seu uso para descrever um tipo de dispositivo tem a vantagem de ser independente de plataformas de hardware ou software e, devido sua formatação, possui uma boa legibilidade, não necessitando de um pré-processamento por parte de um computador para seu conteúdo se tornar compreensível. Esse formato possui, contudo, a desvantagem de repetir uma grande quantidade de informação o que pode prejudicar a velocidade de comunicação entre dispositivos que o utilizem.

Outro fator que deve ser levado em consideração é que o conjunto de protocolos \emph{uP} (\emph{Ubiquitous Protocols}) utiliza o formato \emph{JSON} para troca de mensagens, um mecanismo mais leve, e ainda assim estruturado, para representar informação.

