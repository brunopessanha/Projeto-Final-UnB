\chapter{Classificação de Recursos}

Num ambiente em que dispositivos podem se movimentar livremente e cujos serviços devem ser compartilhados de forma transparente ao usuário, torna-se necessário que cada dispositivo presente na rede possa se comunicar facilmente com os demais, independente de quais tecnologias, padrões ou protocolos tenham sido adotados. É fundamental que essa comunicação seja feita de forma tão rápida e confiável quanto possível, pois toda atividade do middleware com seus clientes depende dela, inclusive o processo de classificação dos recursos disponíveis no smartspace, idéia central deste documento.

Uma propriedade de uma classificação é a hierarquia, ou seja, uma representação em que as categorias da classificação podem ser especializadas formando novas categorias mais específicas, com diferentes atributos. Existem diferentes maneiras conhecidas de se representar uma hierarquia de forma a relacionar categorias que tenham atributos em comum: arquivos no formato XML, hierarquia de classes, ontologia ou simplesmente não utilizar uma hierarquia onde as categorias não tem relações entre si.

O XML (\emph{Extensible Markup Language}), como seu próprio nome sugere, pode ser estendido e dessa forma, é possível montar uma hierarquia de arquivos XML. O seu uso para descrever um tipo de dispositivo tem a vantagem de ser independente de plataformas de hardware ou software e, devido sua formatação, possui uma boa legibilidade, não necessitando de um pré-processamento por parte de um computador para seu conteúdo se tornar compreensível. Esse formato possui, contudo, a desvantagem de repetir uma grande quantidade de informação o que pode prejudicar a velocidade de comunicação entre dispositivos que o utilizem. Outro fator que deve ser levado em consideração é que o conjunto de protocolos \emph{uP} (\emph{Ubiquitous Protocols}) do middleware \emph{uOS} utiliza o formato JSON para troca de mensagens, um mecanismo mais leve, e ainda assim estruturado, para representar informação.

A Hierarquia de Classes é amplamente utilizada em Programação Orientada à Objetos. Uma classe possui atributos ou propriedades e uma outra classe pode herdar esses atributos e criar uma especialização dessa classe raís, à esse relacionamento dá-se o nome de Herança. Dependendo da linguagem orientada utilizada, uma Classe pode herdar métodos e atributos de várias outras Classes, formando uma herança múltipla de classes, ou então poderá cada Subclasse poderá herdar de apenas uma outra Classe, não podendo, entretanto, ocorrer uma herança circular, onde uma Classe A herda de uma Classe B, que por sua vez herda atributos de uma Classe C que especializa a Classe A. Uma Hierarquia de Classes forma, portanto, uma espécie de árvore de classes.

Uma Ontologia é uma descrição formal de conceitos (classes) em um determinado domínio. Esses conceitos possuem propriedades, que descrevem atributos, características e restrições. De forma semelhante à orientação à objetos, conceitos podem possuir subclasses que especializam a classe superior. A Ontologia já permite uma herança múltipla de classes, e novos conceitos podem ser construídos por meio do relacionamento entre classes. Para construir uma Ontologia deve-se definir as classes do domínio, distribuir as classes de forma hierárquica, definir valores e restrições para suas propriedades e utilizar esses valores nas instâncias das classes da Ontologia. Dessa forma, temos uma base de conhecimento. As Ontologias se tornaram popular na Web por dividir produtos em categorias e características em sites de venda e atualmente são utilizadas para compartilhar informação entre pessoas e agentes inteligentes. Uma antologia provê reutilização de conhecimento e torna explícitas as hipóteses sobre um domínio e análise do domínio.

\section{Padrões Diversos}

Nesta seção, discutir-se-á quatro dos principais protocolos e padrões pesquisados que fazem uso de uma classificação de recursos e utilizam as estratégias supracitadas para representar sua hierarquia de recursos: UPnP, IEEE 1451, DLNA e USB. 

\subsection{UPnP}
O \emph{Universal Plug and Play Protocol}(UPnP) é uma arquitetura para conectividade entre aplicações inteligentes, em execução em computadores e dispositivos \emph{wireless} em geral. Ela foi definida para facilitar a conectividade entre dispositivos em diferentes ambientes, como casas, pequenas empresas ou espaços públicos~\cite{upnpArch}. A arquitetura foi desenvolvida de forma a não necessitar de configurações e fornecer uma rede invisível com descoberta automática de dispositivos de diferentes fabricantes e diversas categorias. Dessa maneira, quando um novo dispositivo entra nessa rede, ele é identificado por um IP, publica seus serviços e conhece os serviços de outros dispositivos do ambiente. 

O UPnP utiliza o nome "Universal", pois não é necessária a utilização de ~\emph{drivers} para cada dispositivo. Para atingir esse objetivo, essa arquitetura faz uso de protocolos de internet bem conhecidos como IP, TCP, UDP, HTTP além do formato XML, para facilitar a interoperabilidade entre dispositivos diversos. Da mesma forma que os protocolos de internet, os contratos dos serviços dos dispositivos são escritos em XML e enviados por HTTP.

Após a entrada do dispositivo em uma rede, ele recebe um IP via DHCP ou gera um IP para si, no caso de redes não gerenciadas. A \emph{UPnP Device Architecture}(UDA) divide os dispositivos em duas categorias principais: dispositivos controlados ou simplesmente "dispositivos" e pontos de controle~\cite{upnpArch} formando uma espécie de arquitetura cliente(pontos de controle)-servidor(dispositivos). 

\begin{comment}	
Quando um dispositivo entra na rede UPnP, ele comunica seus serviços para os pontos de controle ou, caso seja um ponto de controle, o protocolo de descoberta do UPnP permite que esse ponto de controle procure dispositivos de seu interesse. O segundo passo é o envio da descrição detalhada do dispositivo para os pontos de controle, que de posse dessa descrição pode enviar mensagens de controle para o dispositivo(passo 3). Os pontos de controle podem também assinar um contrato para envio de notificações de evento utilizando algum dos serviços que o dispositivo provê(passo 4). O último passo é de apresentação, caso o dispositivo possua uma URL de apresentação é possível abrir uma página \emph{web} no \emph{browser} e um usuário pode controlar o dispositivo por meio dessa página~\cite{upnpArch}.
\end{comment}


Nosso foco neste trabalho se encontra no segundo passo do processo: a descrição dos dispositivos. Após sua descoberta, os pontos de controle ainda não tem muita informação sobre o dispositivo, então os pontos de controle interessados solicitam a descrição do dispositivo via uma URL que este dispositivo disponibilizou durante primeira etapa.

A descrição de um dispositivo definida no UPnP é feita por meio de um arquivo XML e inclui informações sobre o fabricante, lista de dispositivos embarcados ou serviços e para cada serviço URLs para controle, eventos e apresentação. A descrição de serviços inclui ainda uma lista de comandos ou ações que o dispositivo irá responder e para cada ação existem parâmetros ou argumentos. O estado do serviço é representado por uma lista de variáveis.

O UPnP define uma série de classificações pré-determinadas que os fabricantes podem utilizar na definição de seus dispositivos. Um característica encontrada nessas classificações é a sua leveza de serem implementadas por dispositivos com pouca capacidade de computação, embora também possam ser utilizadas por dispositivos mais robustos como computadores. São elas:

\begin{itemize}
\item Áudio/Vídeo:
	Essa categoria possui duas principais sub-classificações que foram sendo atualizadas com o desenvolvimento de novas teconologias: 
	\begin{itemize}
		\item \emph{Media Server}:

			Define um dispositivo genérico que provê conteúdo de áudio e/ou vídeo, como CD e DVD \emph{players}, cameras, rádios, televisões e \emph{set-top boxes}. Embora, seja utilizado por dispositivos com diferentes capacidades de processamento e conteúdos, o Media Server expõe conteúdo de forma consistente.

		\item \emph{Media Renderer}:

			Define um dispositivo genérico capazes de renderizar conteúdos de áudio e/ou vídeo como MP3 \emph{players} e televisões. Dependendo da implementação de um \emph{Media Renderer}, pode-se utilizar os recursos de auto-falante de uma televisão para consumir um serviço de música de um \emph{Media Server}.
	\end{itemize}
\item Gerenciamento de Dispositivos:
	Essa categoria foi criada para adicionar operações gerenciais à qualquer dispositivo UPnP. Esse gerência inclui funções para configuração de serviços e do próprio dispositivo, diagnóstico e correção de problemas além de gerência do \emph{firmware} e dos \emph{softwares} do dispositivo.

\item Automação Residencial:
	\begin{itemize}
		\item \emph {Cortina de Proteção Solar}:

			Provê uma sombra por meio de uma cortina. Seu controle pode ser manual, automático ou desabilitado. Sua especificação não contempla configurações a respeito da automação da cortina ou proteções.
		\item \emph {Camera Digital de Segurança}:

			Provê controle básico sobre a configuração da câmera, contendo serviços de fotos e vídeos.
		\item \emph {Aquecimento, Ventilação e Ar Condicionado}:
			
			Esse dispositivo conta com auxílio de sensores de temperatura e possui a capacidade de saber ou controlar a temperatura do ambiente por meio de ventiladores e ar-condicionados.
		\item \emph {Controles de Luz}:
			
			São divididos em Luz binária, que representa uma lâmpada ou qualquer dispositivo emissor de luz que possa somente estar apagado ou aceso, e em Luz cuja intensidade pode ser alterada, 
	\end{itemize}

\item Rede:
	\begin{itemize}
		\item \emph{Gateway} de Internet:
			
			Essa classificação define um dispositivo de interconexão entre uma rede residencial local (LAN) e a \emph{Wide Area Network} (WAN), provendo conectividade com a internet.
			\begin{itemize}
				\item Dispositivo de Conexão WAN: É um dispositivo virtual definido tendo o \emph{Gateway} de Internet como raiz. Funciona como um contêiner para um \emph{link} ou serviços de conexão em uma interface WAN. 
				\item Dispositivo WAN: É um dispositivo virtual efinido tendo o \emph{gateway} de internet como raiz. Cada dispositivo WAN é uma instância virtual de uma interface WAN no \emph{gateway} de internet. Múltiplas interfaces físicas WAN para clientes UPnP, existirão distintas instâncias deste dispositivo.
			\end{itemize}

		\item Ponto de acesso WLAN(\emph{Wireless Local Area Netowork}):
			
			Esse dispositivo implementa os padrões IEEE 802.11 (a,b,g) sem fio para prover uma infraestrutura de rede para casas e pequenas empresas. Essa definição não inclui uso dos pontos de acesso como \emph{hotspots} ou redes de grandes empreendimentos. O ponto de acesso age com uma ponte \emph{Ethernet} que permite ligações de múltiplos nós com a LAN.
	\end{itemize}

\item Impressora:
	Define um dispositivo com capacidade de impressão. Essa especificação não abrange dispositivos que possuem funções de FAX ou Scanner, que possui uma especificação própria.

\item Acesso Remoto:
	Essa categoria é dividida entre dois dispositivos:
	\begin{itemize}
		\item Agente de Descoberta de Acesso Remoto:
			Possui a função de prover a capacidade de sincronizar a informaçào sobre a descoberta UPnP entre duas redes remotas.
		
		\item Servidor de Acesso Remoto:
			Permite que pontos de controle configurem Servidores de Acesso Remoto.

	\end{itemize}

\item Interface Remota:
	Classifica dispositivos entre servidores e clientes de uma interface remota com o usuário.

\item \emph{Scanner}:
	Representa um dispositivo de \emph{Scanner} com \emph{feeder} opcional. Esse dispositivo possui os serviços de digitalização via \emph{feeder} ou \emph{flatbed} e um serviço para configuração com o painel frontal do dispositivo. Essa categoria não contempla funcionalidades de Fax ou cópias.

\item Telefonia:
	\begin{itemize}
		\item Servidor de Telefonia:

			Permite que pontos de controle gerenciem chamadas telefônicas, messagens e presença por meio de outros dispositivos UPnP. 

		\item Cliente de Telefonia:
		
			Permite que pontos de controle possam gerênciar mídias por meio de um de um servidor de telefonia.
				
	\end{itemize}

\item Básico:
	A definição de dispositivos básicos provêm um mecanismo para produtos que não se enquadram em uma classificação adequada do UPnP, possam utilizá-lo. Por esse motivo, essa especificação não possui nenhum serviço definido, mas pode ser utilizada como dispositivo raíz para outras categorias já definidas.
\end{itemize}

Os fabricantes podem, a partir de uma classificação padrão, estender e especializar determinada definição adicionando novos serviços, por exemplo, o que sugere uma hierarquia de dispositivos por meio de arquivos XML. Um fator interessante nas classificações de dispositivos UPnP é que sua especificação já traz consigo a especificação de serviços que esses dispositivos provêem. 

\subsection{IEEE 1451}
O IEEE 1451 se divide em uma família de padrões que foram criados com os objetivos de permitir a capacidade de comunicação entre transdutores(sensores e atuadores) de forma \emph{plug-and-play} por meio de redes com ou sem fio, facilitar a criação de transdutores com inteligência embarcada, simplificar a configuração e manutenção de sistemas, prover comunicação entre transdutores legados e por fim habilitar a implementação de transdutores inteligentes e com uso mínimo de memória~\cite{ieee1451journal}.

Um dos padrões da família, o IEEE 1451.1, define um modelo de informação para \emph{Network Capable Application Processors} (NCAP) que foi estabelecido para especificar um modelo de objetos comum e interfaces de componentes da rede de transdutores. Dessa forma, foi desenvolvido um \emph{framework} orientado à objetos que pode ser estendido para facilitar o desenvolvimento de aplicações que foi definido da seguinte forma: Um modelo de dados que especifica a forma e o tipo de comunicação, tanto local quanto remota, por meio das interfaces de objetos 1451.1, um modelo de objetos que especifica tipos de componentes de software usados para definir e implementar sistemas e por fim, modelos de comunicação que definem a sintaxe e semântica das interfaces de software entre redes de comunicação e objetos de aplicação~\cite{ieeeOO1451}.

Citar ~\cite{ieee1451standard}.

\subsection{DLNA}

A \emph{Digital Living Network Alliance} (DLNA) é uma organização composta pelas principais empresas de eletrônicos de consumo, computação e dispositivos móveis. Foi fundada em 2003 e tem como objetivo fornecer orientações para permitir a interoperabilidade entre dispositivos para completar a convergência da indústria digital, levando, dessa forma, inovação, simplicidade e valor para os consumidores~\cite{dlnaoverview}. Tem como visão facilitar a criação, o gerenciamento e o compartilhamento de conteúdo digital (fotos, músicas e vídeos) entre os dispositivos pertencentes à mesma rede~\cite{dlnahdvideostreaming}. Deve permitir, por exemplo~\cite{dlnaoverview}:

\begin{itemize}
	\item Facilmente adquirir, armazenar e acessar música digital a partir de praticamente qualquer lugar da casa;
	\item Facilmente gerenciar, visualizar, imprimir e compartilhar fotos digitais;
	\item Transportar seu conteúdo favorito a partir qualquer lugar, mesmo que as pontas envolvidas estejam em movimento;
	\item Aproveitar a gravação e reprodução de conteúdo distribuído e multi-usuário.
\end{itemize}

Suas diretrizes destacam casos de uso cuidadosamente construídos para redes domésticas (veja a tabela~\ref{tab:casosdeuso_dlna}), funções adicionais que aumentam a experiência do compartilhamento de conteúdo e doze classes de dispositivos espalhados na categoria "Rede Doméstica e Dispositivos Móveis". A classificação de um dispositivo é feita de tal forma que um único aparelho multifuncional pode possuir diversas categorias diferentes~\cite{dlnahdvideostreaming, dlnaclasses}:

\begin{itemize}
	\item Home Network Devices
	\begin{itemize}
		\item \emph{Digital Media Server} (DMS): Esses dispositivos armazenam conteúdo e o torna disponível para \emph{Digital Media Players} (DMP) e \emph{Digital Media Renderes} que estão conectados na rede. Alguns servidores de mídia podem ainda proteger o conteúdo do usuário uma vez que ele tenha sido armazenado. Exemplo: PCs e dispositivos de armazenamento via rede(NAS);
		\item \emph{Digital Media Player} (DMP): Esses dispositivos encontram conteúdo disponibilizados pelos servidores de mídia (DMS) e proveem a capacidade reprodução e renderização de mídia.Exemplos: TVs, aparelhos de som, \emph{home theaters}, monitores sem fio e vídeo games;
		\item \emph{Digital Media Renderer} (DMR): Esses dispositivos tocam conteúdos recebidos do controlador digital de mídia (DMC) que, por sua vez, irá encontrar conteúdo do servidor de mídia digital (DMS). Exemplo: TVs, receptores de áudio ou vídeo, projetores de vídeo e alto-falantes remotos para música;
		\item \emph{Digital Media Controller} (DMC): Esses dispositivos encontram conteúdo em servidores de mídia digital (DMS) e o tocam em renderizadores de mídia digital(DMR). Exemplo: \emph{Tablets}, câmeras digitais com Wi-Fi e PDAs;
		\item \emph{Digital Media Printer} (DMPr): Esses dispositivos proveem serviços de impressão para a rede residencial DLNA. Geralmente, tocadores de mídia digital (DMP) e controladores de mídia digital (DMC) com a capacidade de impressão podem utilizar o DMPr para impressão. Exemplo: Impressoras de foto e multifuncionais conectadas na rede DLNA.
	\end{itemize}
	\item Mobile Handheld Devices
	\begin{itemize}
		\item \emph{Mobile Digital Media Server} (M-DMS): Esses dispositivos sem fio armazenam conteúdo e o tornam disponível para tocadores de mídia digital que tenham acesso à rede sem fio ou com fio (M-DMP), renderizadores de mídia digital (DMR) e impressoras de mídia digital (DMPr). Exemplo: telefones portáteis e tocadores portáteis de música;
		\item \emph{Mobile Digital Media Player} (M-DMP): Esses dispositivos sem fio encontram e tocam conteúdo de um servidor de mídia digital (DMS) ou servidor móvel de mídia digital (M-DMS). Exemplo: telefones portáteis, \emph{tablets} projetados para visualização de conteúdo multimídia;
		\item \emph{Mobile Digital Media Uploader} (M-DMU): Esses dispositivos sem fio enviam conteúdo para um servidor digital de de mídia (DMS) ou para um servidor móvel de mídia digital (M-DMS). Exemplo: câmeras digitais, e telefones portáteis;
		\item \emph{Mobile Digital Media Downloader} (M-DMD): Esses dispositivos sem fio encontram e armazenam conteúdo de um servidor de mídia digital (DMS) ou de um servidor móvel de mídia digital (M-DMS). Exemplo: tocadores de música e telefones portáteis;
		\item \emph{Mobile Digital Media Controller} (M-DMC): Esses dispositivos encontram conteúdo em um servidor digital de mídia(DMS) ou em um servidor móvel de mídia digital (M-DMS) e o envia para renderizadores de mídia digital(DMR). Exemplo: telefones portáteis e PDAs.
	\end{itemize}
	\item Home Infrastructure Devices
	\begin{itemize}
		\item \emph{Mobile Network Connectivity Function} (M-NCF): Esses dispositivos proveem uma ligação entre a conectividade de rede de dispositivos móveis portáteis e conectividade de rede residencial;
		\item \emph{Media Interoperability Unit} (MIU): Esses dispositivos proveem transformação de conteúdo entre formatos de mídia necessários para uma rede residencial ou para dispositivos móveis portáteis.
	\end{itemize}
\end{itemize}

\begin{comment}
\begin{table}
	\begin{center}
		\begin{tabular}{llll}
		\hline
		\textbf{Classe} & \textbf{Categoria} & \textbf{Descrição} & \textbf{Exemplo}						\\
		\hline
		\hline
		Mobile Network Connectivity Function (M-NCF) & \multirow{2}{*}{Home Infrastructure Devices} &  & \\
		\hline
		Media Interoperability Unit (MIU) & &  & \\
		\hline
		\end{tabular}
	\end{center}
	\caption{Classes DLNA~\cite{}.}
	\label{tab:classes_dlna}
\end{table}
\end{comment}

Imagine, por exemplo, uma situação hipotética em que uma pessoa deseja compartilhar um pequeno vídeo com seus amigos a partir de seu celular. A fim de exibir esse vídeo em sua TV widescreen na sua sala de estar, ela necessita, primeiramente, envia-lo para seu próprio email. Em seguida, deve ligar seu PC, fazer o download do vídeo e salvá-lo em um \emph{pendrive} ou cartão de memória. Posteriormente, deve ligá-lo na TV ou em um receptor digital e, então, utilizar a interface do dispositivo para localizar o vídeo e exibi-lo~\cite{dlnahdvideostreaming}.

Além de interromper a conversa, esse tipo de processo de transferência de conteúdo simplesmente consome muito tempo, mesmo quando ocorre sem problemas. Como resultado, a partilha informal de vídeo em múltiplas telas não é visto muitas vezes como uma opção razoável~\cite{dlnahdvideostreaming}.

\begin{figure}[ht]
	\center
	\includegraphics[scale=0.3]{imagens/dlna1}
	\caption{A exibição de um vídeo ou foto de um celular em uma TV envolve tal processo tradicional, tedioso e demorado que os consumidores raramente farão.}
	\label{fig:traditionalProccess}
\end{figure}

Este é o valor da proposição que o DLNA oferece aos seus consumidores: a partilha contínua e sem esforço de conteúdo digital. Produtos projetados seguindo as diretrizes DLNA estabelecidas para compartilhar vídeo em apenas um único passo: a transmissão de uma cópia do vídeo do telefone sem fio para a TV. O consumidor pode até mesmo congelar o vídeo usando um "controle remoto" do menu no telefone, bem como \emph{fast-forward} e reproduzir o vídeo~\cite{dlnahdvideostreaming}.

\begin{figure}[ht]
	\center
	\includegraphics[scale=0.3]{imagens/dlna2}
	\caption{O valor da proposição do DLNA é o compartilhamento transparente e fácil de conteúdo, permitindo que os consumidores enviem uma cópia de um vídeo ou fotos diretamente para a TV em uma única etapa.}
	\label{fig:dlnaProccess}
\end{figure}

Em suma, o DLNA pode ser visto como uma coleção de padrões abertos que definem como uma rede residencial interage em todos os seus níveis, ou seja, além de definir como os diferentes padrões irão interoperar e como os dados serão tratados em cada nível, ele também reduz o número de padrões que um dispositivo deve suportar.

\begin{table}
	\begin{center}
		\begin{tabular}{rl}
		\hline
		\textbf{Casos de Uso} & \textbf{Exemplo}																\\
		\hline
		Enviar & Transferir vídeos ou imagens capturadas em uma câmera digital ou celular para um computador.	\\
		\hline
		Empurrar & Exibir vídeos ou imagens capturadas em uma câmera digital ou celular diretamente em uma TV sem intermédio de um computador. \\
		\hline
		Localizar e Reproduzir ou "Reproduzir em..."" & Utiliza um celular para localizar uma música ou vídeo armazenados em um computador, unidade de disco externa ou um dispositivo \emph{Network-Attached Storage} (NAS) e transferi-lo, via \emph{stream} ou não, para reprodução. \\
		\hline
		Puxar e imprimir ou "Imprimir em..." & Visualizar na TV uma foto armazenada em um servidor de mídia e imprimi-la utilizando uma impressora em rede. \\
		\hline
		\end{tabular}
	\end{center}
	\caption{Exemplos de casos de uso~\cite{dlnahdvideostreaming}.}
	\label{tab:casosdeuso_dlna}
\end{table}

\begin{table}
	\begin{center}
		\begin{tabular}{rll}
		\hline
		\textbf{Camada} & \textbf{Função definida} & \textbf{Padrões}				\\
		\hline
		Transmissão Protegida & Como um conteúdo comercial está protegido em uma rede doméstica. & DTCP/IP \\
		\hline
		Formatos de Mídia & Como um conteúdo de mídia está codificado e identificado para interoperabilidade. & MPEG2, MPEG4, AVC/H.264, LPCM, MP3, AAC LC, JPEG, XHTML-Print- \\
		\hline
		Transporte de Mídia & Como um conteúdo de mídia é transferido. & HTTP, Quality of Service \\
		\hline
		Gerência de Mídia & Como um conteúdo de mídia é identificado, gerenciado e distribuído. & UPnP AV 1.0, UPnP Print Enhanced 1.0 \\
		\hline
		Descoberta e Controle & Como dispositivos se descobrem e se controlam um ao outro. & UPnP Device Architecture 1.0 \\
		\hline
		Redes IP & \multirow{2}{*}{Como dispositivos com e sem fio fisicamente se conectam e se comunicam.} & IPv4 Protocol Suite \\
		Conectividade & & Wired: Ethernet 802.3, MoCAWireless: Wi-Fi 802.11, Wi-Fi Protected Setup \\
		\hline
		\end{tabular}
	\end{center}
	\caption{Camadas padrões DLNA~\cite{dlnahdvideostreaming}.}
	\label{tab:camadaspadroes_dlna}
\end{table}

\subsection{USB}

O \emph{Universal Serial Bus} (USB) é uma arquitetura de comunicaçao que adiciona a um computador pessoal (PC) a capacidade de se interconectar a uma variedade de dispositivos. Surgiu com o intuito de sanar três das principais dificuldades enfrentadas à época de sua criação~\cite{usbspec}:

\begin{itemize}
	\item Integrar as plataformas das industrias da informática e comunicação. Para tal, a troca de informações entre esses dispositivos deveria acontecer de forma ubíqua e barata;
	\item Facilitar a reconfiguração de dispositivos de um computador, tais como teclados, mouses e joysticks;
	\item Integrar todas as interfaces existentes à época em um única que pudesse ser utilizada pela maior quantidade possível de dispositivos: telefone, fax, modem, adaptadores, secretárias eletrônicas, \emph{scanners}, PDA's, teclados, \emph{mouses}, etc.
\end{itemize}

Atualmente sua velocidade de comunicação varia entre 1.5 ou 12 megabits por segundo (mbs) e seu protocolo permite configurar dispositivos durante a fase de inicialização ou quando eles são plugados em tempo de execução. Esses dispositivos são divididos em várias classes e/ou subclasses diferentes (veja em~\ref{tab:dispositivos_usb}), cada qual definindo um comportamento e protocolos comuns para dispositivos que oferecem funções similares~\cite{hid}. A partir dessas classificações será possível estabelecer uma hierarquia, a qual poderá ser usada no processo de classificação dos dispositivos presentes no smartspace.

\begin{table}
	\begin{center}
		\begin{tabular}{cccc}
		\hline
		\textbf{Classes de Dispositivos}																		\\
		\hline
		Audio					&	Comunicação				&	Interface Humana	&	Físico 					\\
		\hline
		Imagem					&	Impressora				&	Disco Rígido		&	Hub 					\\
		\hline
		\emph{Smart Card}		&	Segurança de Conteúdo	&	Vídeo				&	Saúde Pessoal 			\\
		\hline
		Diagnóstico				&	Controlador Wireless	&	Diversos			&	Aplicação Específica	\\
		\hline
		Fabricante Específico	&							&						&							\\
		\hline
		\end{tabular}
	\end{center}
	\caption{Exemplos de classes de dispositivos USB.}
	\label{tab:dispositivos_usb}
\end{table}

\subsubsection{SuperSpeed}
\subsubsection{Wireless}
\subsubsection{Hi-Speed}
\subsubsection{On-The-Go and Embedded Host}

Dentre as diversas classes existentes para o protocolo USB, será utilizada a \emph{Human Interface Device} (HID). Tal decisão baseou-se na característica principal desse tipo de classe que, de forma geral, destina-se primordialmente aos dispositivos que são usados por humanos para controlar e operacionalizar quaisquer tipos de sistemas computacionais~\cite{hid}. Mais específicamente, possui as seguintes particularidades:

\begin{itemize}
	\item Deve ser tão compacto quanto possível para evitar desperdício de espaço no dispositivo;
	\item Permitir a aplicação de software para evitar informação desconhecida;
	\item Ser extensível e robusto;
	\item Suportar hierarquias e coleções;
	\item Ser autodescritivo para permitir aplicações de software genéricos.
\end{itemize}

\begin{table}
	\begin{center}
		\begin{tabular}{lr}
		\hline
		\textbf{Dispositivo} & \textbf{Exemplo Descritivo} \\
		\hline
		Teclados e dispositivos apontadores & Mouses, trackballs e joysticks \\
		Painéis de controle & Maçanetas, Interruptores, botões e alavancas \\
		Controles que podem ser encontrados \\
		em dispositivos como telefones,\\
		controles remotos, jogos ou dispositivos \\
		de simulação & Luvas virtuais, câmbios, \\
		volantes e pedais \\
		Dispositivos que não necessitam de \\
		interação humana mas que fornecem dados \\
		de forma similar aos dipositivos da \\
		classe HID & Leitores de códigos de barra, \\termômetros ou voltímetros. \\
		\hline
		\end{tabular}
	\end{center}
	\caption{Exemplos de classes de dispositivos USB.}
	\label{tab:dispositivos_USB}
\end{table}

\begin{comment}
\section{Hierarquia de Recursos}
Na seção anterior mostramos o estado da arte na classificação de dispositivos e recursos. Vimos que é comum a representação dessa classificação de forma hierárquica. Nesta seção mostraremos as formas conhecidas de representar uma hierarquia de recursos.

\subsection{XML}
O XML(\emph{Extensible Markup Language}), como seu próprio nome sugere, pode ser estendido e dessa forma, é possível montar uma hierarquia de arquivos XML. O seu uso para descrever um tipo de dispositivo tem a vantagem de ser independente de plataformas de hardware ou software e, devido sua formatação, possui uma boa legibilidade, não necessitando de um pré-processamento por parte de um computador para seu conteúdo se tornar compreensível. Esse formato possui, contudo, a desvantagem de repetir uma grande quantidade de informação o que pode prejudicar a velocidade de comunicação entre dispositivos que o utilizem.


\subsection{Hierarquia de Classes}
Uma das formas bem conhecidas e amplamente utilizada para definir um modelo hierárquico é utilizar uma modelagem Orientada à Objetos, ou seja, classes e suas especializações e relacionamentos. Esse é o modelo adotado, pelo padrão IEEE 1451.1, por exemplo, que utiliza um modelo de classes para definir sua hierarquia na classificação de dispositivos.
\subsection{Ontologia}
Uma Ontologia é uma descrição formal de conceitos(classes) em um determinado domínio. Esses conceitos possuem propriedades, que descrevem atributos, características e restrições. De forma semelhante à orientação à objetos, conceitos podem possuir subclasses que especializam a classe superior.

Para construir uma Ontologia deve-se definir as classes do domínio, distribuir as classes de forma hierárquica, definir valores e restrições para suas propriedades e utilizar esses valores nas instâncias das classes da Ontologia. Dessa forma, temos uma base de conhecimento.

As ontologias se tornaram popular na Web por dividir produtos em categorias e características em sites de venda e atualmente são utilizadas para compartilhar informação entre pessoas e agentes inteligentes. Uma antologia provê reutilização de conhecimento e torna explícitas as hipóteses sobre um domínio e análise do domínio.

Uma ontologia, portanto, se torna útil para o objetivo deste trabalho.
\end{comment}