\subsection{UPnP}
O \emph{Universal Plug and Play Protocol}(UPnP) é uma arquitetura para conectividade entre aplicações inteligentes, em execução em computadores e dispositivos \emph{wireless} em geral. Ela foi definida para facilitar a conectividade entre dispositivos em diferentes ambientes, como casas, pequenas empresas ou espaços públicos~\cite{upnpArch}. A arquitetura foi desenvolvida de forma a não necessitar de configurações e fornecer uma rede invisível com descoberta automática de dispositivos de diferentes fabricantes e diversas categorias. Dessa maneira, quando um novo dispositivo entra nessa rede, ele é identificado por um IP, publica seus serviços e conhece os serviços de outros dispositivos do ambiente. 

O UPnP utiliza o nome "Universal", pois não é necessária a utilização de ~\emph{drivers} para cada dispositivo. Para atingir esse objetivo, essa arquitetura faz uso de protocolos de internet bem conhecidos como IP, TCP, UDP, HTTP, além do formato XML, para facilitar a interoperabilidade entre dispositivos diversos. Da mesma forma que os protocolos de internet, os contratos dos serviços dos dispositivos são escritos em XML e enviados por HTTP.

Após a entrada do dispositivo em uma rede, ele recebe um IP via DHCP ou gera um IP para si, no caso de redes não gerenciadas. A \emph{UPnP Device Architecture}(UDA) divide os dispositivos em duas categorias principais: dispositivos controlados ou simplesmente "dispositivos" e pontos de controle~\cite{upnpArch} formando uma espécie de arquitetura cliente(pontos de controle)-servidor(dispositivos). 

\begin{comment}	
Quando um dispositivo entra na rede UPnP, ele comunica seus serviços para os pontos de controle ou, caso seja um ponto de controle, o protocolo de descoberta do UPnP permite que ele procure dispositivos de seu interesse. O segundo passo é o envio da descrição detalhada do dispositivo para os pontos de controle, que de posse dessa descrição podem enviar mensagens de controle para o dispositivo(passo 3). Os pontos de controle podem também assinar um contrato para envio de notificações de evento utilizando algum dos serviços que o dispositivo provê(passo 4). O último passo é de apresentação. Caso o dispositivo possua uma URL de apresentação é possível abrir uma página \emph{web} no \emph{browser} e um usuário pode controlar o dispositivo por meio dessa página~\cite{upnpArch}.
\end{comment}


Nosso foco neste trabalho se encontra no segundo passo do processo: a descrição dos dispositivos. Após sua descoberta, os pontos de controle ainda não têm muita informação sobre o dispositivo, então os pontos de controle interessados solicitam a descrição do dispositivo via uma URL que este dispositivo disponibilizou durante primeira etapa.

A descrição de um dispositivo definida no UPnP é feita por meio de um arquivo XML que inclui informações sobre o fabricante, lista de dispositivos embarcados ou serviços e para cada serviço, URLs para controle, eventos e apresentação. A descrição de serviços inclui ainda uma lista de comandos ou ações que o dispositivo deverá responder e para cada ação existem parâmetros ou argumentos. O estado do serviço é representado por uma lista de variáveis.

O UPnP define uma série de classificações pré-determinadas que os fabricantes podem utilizar na definição de seus dispositivos. Um característica encontrada nessas classificações é a sua leveza de serem implementadas por dispositivos com pouca capacidade de computação, embora também possam ser utilizadas por dispositivos mais robustos como computadores. São elas:

\begin{itemize}
\item Áudio/Vídeo:
	Essa categoria possui duas principais sub-classificações que foram sendo atualizadas com o desenvolvimento de novas tecnologias: 
	\begin{itemize}
		\item \emph{Media Server}:

			Define um dispositivo genérico que provê conteúdo de áudio e/ou vídeo, como CD e DVD \emph{players}, cameras, rádios, televisões e \emph{set-top boxes}. Embora, seja utilizado por dispositivos com diferentes capacidades de processamento e conteúdos, o \emph{Media Server} expõe conteúdo de forma consistente.

		\item \emph{Media Renderer}:

			Define um dispositivo genérico capazes de renderizar conteúdos de áudio e/ou vídeo como MP3 \emph{players} e televisões. Dependendo da implementação de um \emph{Media Renderer}, pode-se utilizar os recursos de auto-falante de uma televisão para consumir um serviço de música de um \emph{Media Server}.
	\end{itemize}
\item Gerenciamento de Dispositivos:
	Essa categoria foi criada para adicionar operações gerenciais à qualquer dispositivo UPnP. Esse gerência inclui funções para configuração de serviços e do próprio dispositivo, diagnóstico e correção de problemas além de gerência do \emph{firmware} e dos \emph{softwares} do dispositivo.

\item Automação Residencial:
	\begin{itemize}
		\item Cortina de Proteção Solar:

			Provê uma sombra por meio de uma cortina. Seu controle pode ser manual, automático ou desabilitado. Sua especificação não contempla configurações a respeito da automação da cortina ou proteções.
		\item Câmera Digital de Segurança:

			Provê controle básico sobre a configuração da câmera, contendo serviços de fotos e vídeos.
		\item Aquecimento, Ventilação e Ar Condicionado:
			
			Esse dispositivo conta com auxílio de sensores de temperatura e possui a capacidade de saber ou controlar a temperatura do ambiente por meio de ventiladores e ar-condicionados.
		\item Controles de Luz:
			
			São divididos em Luz binária, que representa uma lâmpada ou qualquer dispositivo emissor de luz que possa somente estar apagado ou aceso, e em Luz cuja intensidade pode ser alterada, 
	\end{itemize}

\item Rede:
	\begin{itemize}
		\item \emph{Gateway} de Internet:
			
			Essa classificação define um dispositivo de interconexão entre uma rede residencial local (LAN) e a \emph{Wide Area Network} (WAN), provendo conectividade com a internet.
			\begin{itemize}
				\item Dispositivo de Conexão WAN: É um dispositivo virtual definido tendo o \emph{Gateway} de Internet como raiz. Funciona como um contêiner para um \emph{link} ou serviços de conexão em uma interface WAN. 
				\item Dispositivo WAN: É um dispositivo virtual efinido tendo o \emph{gateway} de internet como raiz. Cada dispositivo WAN é uma instância virtual de uma interface WAN no \emph{gateway} de internet. Múltiplas interfaces físicas WAN para clientes UPnP, existirão distintas instâncias deste dispositivo.
			\end{itemize}

		\item Ponto de acesso WLAN(\emph{Wireless Local Area Netowork}):
			
			Esse dispositivo implementa os padrões IEEE 802.11 (a,b,g) sem fio para prover uma infraestrutura de rede para casas e pequenas empresas. Essa definição não inclui uso dos pontos de acesso como \emph{hotspots} ou redes de grandes empreendimentos. O ponto de acesso age com uma ponte \emph{Ethernet} que permite ligações de múltiplos nós com a LAN.
	\end{itemize}

\item Impressora:
	Define um dispositivo com capacidade de impressão. Essa especificação não abrange dispositivos que possuem funções de FAX ou \emph{Scanner}, que possui uma especificação própria.

\item Acesso Remoto:
	Essa categoria é dividida entre dois dispositivos:
	\begin{itemize}
		\item Agente de Descoberta de Acesso Remoto:

			Possui a função de prover a capacidade de sincronizar a informação sobre a descoberta UPnP entre duas redes remotas.
		
		\item Servidor de Acesso Remoto:
		
			Permite que pontos de controle configurem Servidores de Acesso Remoto.

	\end{itemize}

\item Interface Remota:
	Classifica dispositivos entre servidores e clientes de uma interface remota com o usuário.

\item \emph{Scanner}:
	Representa um dispositivo de \emph{Scanner} com \emph{feeder} opcional. Esse dispositivo possui os serviços de digitalização via \emph{feeder} ou \emph{flatbed} e um serviço para configuração com o painel frontal do dispositivo. Essa categoria não contempla funcionalidades de Fax ou cópias.

\item Telefonia:
	\begin{itemize}
		\item Servidor de Telefonia:

			Permite que pontos de controle gerenciem chamadas telefônicas, mensagens e presença por meio de outros dispositivos UPnP. 

		\item Cliente de Telefonia:
		
			Permite que pontos de controle possam gerenciar mídias por meio de um de um servidor de telefonia.
				
	\end{itemize}

\item Básico:
	A definição de dispositivos básicos provêm um mecanismo para produtos que não se enquadram em uma classificação adequada do UPnP, possam utilizá-lo. Por esse motivo, essa especificação não possui nenhum serviço definido, mas pode ser utilizada como dispositivo raiz para outras categorias já definidas.
\end{itemize}

Os fabricantes podem, a partir de uma classificação padrão, estender e especializar determinada definição adicionando novos serviços, por exemplo, o que sugere uma hierarquia de dispositivos por meio de arquivos XML. Um fator interessante nas classificações de dispositivos UPnP é que sua especificação já traz consigo a especificação de serviços que esses dispositivos provêm. 
