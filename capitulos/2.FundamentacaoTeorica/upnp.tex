\subsection{UPnP}
O \emph{Universal Plug and Play Protocol}(UPnP) é uma arquitetura para conectividade entre aplicações inteligentes, em execução em computadores e dispositivos \emph{wireless} em geral. Ela foi definida para facilitar a conectividade entre dispositivos em diferentes ambientes, como casas, pequenas empresas ou espaços públicos~\cite{upnpArch}. A arquitetura foi desenvolvida de forma a não necessitar de configurações e fornecer uma rede invisível com descoberta automática de dispositivos de diferentes fabricantes e diversas categorias. Dessa maneira, quando um novo dispositivo entra nessa rede, ele é identificado por um IP, publica seus serviços e conhece os serviços de outros dispositivos do ambiente. Basicamente o que se espera de uma arquitetura para \emph{smart spaces}. 

O UPnP utiliza o nome "Universal", pois não é necessária a utilização de ~\emph{drivers} para cada dispositivo. Para atingir esse objetivo, essa arquitetura faz uso de protocolos de internet bem conhecidos como IP, TCP, UDP, HTTP além do formato XML, para facilitar a interoperabilidade entre dispositivos diversos. Da mesma forma que os protocolos de internet, os contratos dos serviços dos dispositivos são escritos em XML e enviados por HTTP.

Após a entrada do dispositivo em uma rede, ele recebe um IP via DHCP ou gera um IP para si, no caso de redes não gerenciadas. A \emph{UPnP Device Architecture}(UDA) divide os dispositivos em duas categorias principais: dispositivos controlados ou simplesmente "dispositivos" e pontos de controle~\cite{upnpArch} formando uma espécie de arquitetura cliente(pontos de controle)-servidor(dispositivos). 

Quando um dispositivo entra na rede UPnP, ele comunica seus serviços para os pontos de controle ou, caso seja um ponto de controle, o protocolo de descoberta do UPnP permite que esse ponto de controle procure dispositivos de seu interesse. O segundo passo é o envio da descrição detalhada do dispositivo para os pontos de controle, que de posse dessa descrição pode enviar mensagens de controle para o dispositivo(passo 3). Os pontos de controle podem também assinar um contrato para envio de notificações de evento utilizando algum dos serviços que o dispositivo provê(passo 4). O último passo é de apresentação, caso o dispositivo possua uma URL de apresentação é possível abrir uma página \emph{web} no \emph{browser} e um usuário pode controlar o dispositivo por meio dessa página~\cite{upnpArch}.

Nosso foco neste trabalho se encontra no segundo passo do processo: a descrição dos dispositivos. Após sua descoberta, os pontos de controle ainda não tem muita informação sobre o dispositivo, então os pontos de controle interessados solicitam a descrição do dispositivo via uma URL que este dispositivo disponibilizou durante primeira etapa.

A descrição de um dispositivo definida no UPnP é feita por meio de um arquivo XML e inclui informações sobre o fabricante, lista de dispositivos embarcados ou serviços e para cada serviço URLs para controle, eventos e apresentação. A descrição de serviços inclui ainda uma lista de comandos ou ações que o dispositivo irá responder e para cada ação existem parâmetros ou argumentos. O estado do serviço é representado por uma lista de variáveis.

O UPnP define uma série de classificações pré-determinadas de dispositivos que os fabricantes podem utilizar na definição de seus dispositivos. São elas:

\begin{itemize}
\item Áudio/Vídeo
\item Básico
\item Gerenciador de Dispositivo
\item Automação Residencial
\item Rede
\item Impressora
\item Acesso Remoto
\item Interface Remota
\item \emph{Scanner}
\item Telefonia
\end{itemize}

Os fabricantes podem, a partir de uma classificação padrão, estender e especializar determinada definição adicionando novos serviços, por exemplo, o que sugere uma espécie de hierarquia de dispositivos por meio de arquivos XML.



