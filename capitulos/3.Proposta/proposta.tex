\chapter{Classificação de Recursos na DSOA}

A padronização de recursos é de fundamental importância para \emph{frameworks} que se utilizam da descoberta dinâmica de recursos em determinado momento e contexto. Tal padronização é mais do que uma simples tarefa de nomear recursos, pois uma aplicação cliente precisa saber como interagir com eles por meio de suas interfaces. Uma classificação deve responder as seguintes questões ~\cite{pervasiveComputing} para garantir a semântica entre as partes envolvidas:

\begin{itemize}
	\item O que os clientes esperam de uma instância de um recurso do tipo X?

	\item Quais serviços são apropriados para os recursos do tipo X?

	\item Quais protocolos uma instância do tipo X utiliza?
\end{itemize}

A partir dessas respostas, novos \emph{drivers} poderão ser construídos seguindo-se a definição inicial do recurso com seus serviços apropriados, possibilitando diferentes implementações do mesmo \emph{driver}; além disso, os protocolos são essenciais para definir a forma e o meio de comunicação entre dispositivos. Dessa maneira, a classificação de recursos facilitará o desenvolvimento de novos \emph{drivers} para futuras aplicações, pois possibilitará a definição de interfaces pré-estabelecidas que representem classes de recursos. Outra vantagem decorrente é a possibilidade de seleção de recursos equivalentes, caso o provedor originalmente selecionado esteja indisponível. Além das perguntas acima, um importante fator levado em consideração, foi o fato da DSOA ter sido desenvolvida com o objetivo de atender diversas plataformas abstraindo detalhes como capacidade computacional e linguagem de programação~\cite{buzetoDSOA2010}.

Conforme visto na seção~\ref{sec:comparativo}, foram extraídas as características mais vantajosas dos padrões existentes como o relacionamento entre classes de recursos, as heranças simples e múltiplas e a representação que utilize um padrão leve. A partir delas, construiu-se uma classificação que seja relacionada, extensível, que permita a um recurso pertencer à multiplas classes simultaneamente e cuja representação seja via JSON (\emph{JavaScript Object Notation}). O JSON foi escolhido ante XML, pois possui menor tamanho de mensagens e esse fator pode ser decisivo em um ambiente com diversos dipositivos com capacidades computacionais diferentes e possivelmente reduzidas. Dessa forma a limitação dos dispositivos é minimizada e exclui a necessidade de mecanismos externos para tratamento de mensagens, pois o tratamento do formato JSON não exige muita capacidade de computação. Esta nova classificação é estabelecida por:

\begin{itemize}
	\item Forma de classificação

	Optou-se por uma classificação relacionada, em que classes de recursos podem se relacionar umas com as outras utilizando-se herança. Esta possibilita a extensão, em que interfaces previamente estabelecidas seriam utilizadas na elaboração de novos \emph{drivers}. Obtêm-se, com isso, a possibilidade da utilização de polimorfismo e a vantagem de que cada classe represente exatamente um objeto tão simples quanto possível, almejando-se obter uma alta coesão e um baixo acoplamento.

	\item Representação

	Em um ambiente cujos dispositivos podem se movimentar livremente e seus recursos devem ser compartilhados de forma transparente ao usuário, torna-se necessário que cada dispositivo presente na rede possa se comunicar facilmente com os demais. Atualmente, o \emph{uOS} já utiliza JSON em suas trocas de mensagens devido a sua portabilidade e leveza.

	\begin{comment}
	Tal formato apresenta as seguintes características:
	
		\begin{itemize}
	 		\item Baixo custo computacional~\cite{comparativojson};
	 		\item É auto-descritivo, o que facilita os processos de leitura e escrita por seres-humanos~\cite{json};
	 		\item É estruturado, o que facilita sua criação e análise por computadores~\cite{json};
	 		\item É independente de plataforma, pois utiliza UTF-8 como codificação~\cite{utf8}.
	 	\end{itemize}
	 \end{comment}
\end{itemize}

Além da definição da forma de classificação e da representação utilizada, a classificação de recursos para a DSOA construída também possui um conjunto de tipos básicos e extensíveis, apresentado na seção~\ref{sec:tiposBasicos}, e uma relação de equivalência entre eles, detalhada na seção~\ref{sec:equivalenciaRecursos}. Como era de se esperar, as alterações na DSOA acabam por impactar no protocolo \emph{uP} e no \emph{middleware uOS}, como estes são baseados na arquitetura. Sendo assim, na seção~\ref{sec:impactoUOS} será apresentado o impacto em cada um destes elementos.

\section{Tipos Básicos}
\label{sec:tiposBasicos}

A definição de um conjunto de tipos básicos auxilia na criação de novos dispositivos. Com base nesta classificação é possível adequar-se aos padrões pré-estabelecidos na criação de novos drivers, além de assegurar que as aplicações terão um dicionário comum de elementos raiz a serem utilizados. Com base nos tipos exibidos na tabela~\ref{tab:comparativoClasses} definiu-se os seguintes tipos básicos da DSOA:

\begin{itemize}
	\item Cliente de Áudio
		
		Abrange quaisquer tipos de dispositivos que contenham a capacidade de reproduzir áudio. Exemplo: caixas de som, fones de ouvido;

		\emph{\bf{Identificador do Recurso:}} \emph{AudioOutput}.

		\emph{\bf{Serviços:}}
		\begin{itemize}
			\item \emph{play(stream);}
		\end{itemize}

	\item Servidor de Áudio
		
		Abrange quaisquer tipos de dispositivos que possam operar como servidor de áudio. Exemplo: microfones, tocadores de mp3;

		\emph{\bf{Identificador do Recurso:}} \emph{AudioInput}.

		\emph{\bf{Serviços:}} 
		\begin{itemize}
			\item \emph{sendAudio(stream);}
		\end{itemize}

	\item Video Player
		
		Abrange quaisquer tipos de dispositivos que possuam a capacidade de prover um recurso de renderização de vídeo. Exemplo: TVs, monitores, telas de celular;

		\emph{\bf{Identificador do Recurso:}} \emph{VideoOutput}.

		\emph{\bf{Serviços:}} 
		\begin{itemize}
			\item \emph{play(stream);}
		\end{itemize}

	\item Servidor de Vídeo
		
		Abrange quaisquer tipos de dispositivos que possuam operar como servidor de vídeo. Exemplo: tocadores de DVD, set-top boxes;

		\emph{\bf{Identificador do Recurso:}} \emph{VideoInput}.

		\emph{\bf{Serviços:}} 
		\begin{itemize}
			\item \emph{sendVideo(stream);}
		\end{itemize}

	\item Display de Imagem
		
		Abrange quaisquer tipos de dispositivos que possam prover um recurso de renderização de imagem. Exemplo: impressoras, TVs, monitores;

		\emph{\bf{Identificador do Recurso:}} \emph{ImageDisplay}.

		\emph{\bf{Serviços:}} 
		\begin{itemize}
			\item \emph{display(image);}
		\end{itemize}

	\item Servidor Imagem
		
		Abrange quaisquer tipos de dispositivos que possam operar como servidor de imagem. Exemplo: câmeras digitais, \emph{scanners};

		\emph{\bf{Identificador do Recurso:}} \emph{Image}.

		\emph{\bf{Serviços:}} 
		\begin{itemize}
			\item \emph{sendImage(image);}
		\end{itemize}
	\item Teclado
		
		Abrange quaisquer tipos de dispositivos que possam prover um recurso de teclado. Exemplo: teclados e dispositivos de reconhecimento gestual;

		\emph{\bf{Identificador do Recurso:}} \emph{Keyboard}.

		\emph{\bf{Serviços:}} 
		\begin{itemize}
			\item \emph{charTyped(key);}
			\item \emph{commandTyped(key);}
		\end{itemize}

	\item Apontador
		
		Abrange quaisquer tipos de dispositivos que possuam o recurso de apontadores. Exemplo: \emph{mouses}, telas sensíveis ao toque, \emph{joysticks};

		\emph{\bf{Identificador do Recurso:}} \emph{Pointer}.

		\emph{\bf{Serviços:}} 
		\begin{itemize}
			\item \emph{buttonPressed(button);}
			\item \emph{buttonReleased(button);}
		\end{itemize}

\end{itemize}

Repare que os tipos básicos possuem poucos serviços. Esse fato permite que \emph{drivers} mais especializados possam ser definidos a partir deles, pois os recursos básicos possuem apenas os serviços mais simples que se espera de cada um dos recursos. Por exemplo, o serviço mais simples que um recurso de Display de Imagem, uma televisão, por exemplo, pode ter, é o de mostrar uma imagem. Além disso, as aplicações podem requisitar serviços mais ou menos especializados, dependendo da necessidade. Dessa forma, quanto mais granular for a definição de novos recursos, mais fácil será sua reutilização por outros recursos especializados.

Tal definição é importante, pois é ela que permitirá um agrupamento de recursos em torno de suas funcionalidades, a saber, os seus serviços. Seu valor ficará mais claro com a leitura das próximas seções, nas quais os conceitos de ``Equivalência de recursos'' e ``Árvore de Equivalência'' serão apresentados.

\begin{comment}
Suponha que um usuário, por meio de uma aplicação de seu celular, deseja utilizar os serviços providos de um recurso de imagem. Considere ainda, que existam três instâncias disponíveis deste recurso no \emph{smart space}. Para garantir a equivalência dos recursos devemos garantir que eles possuam os mesmos serviços, e para isso, devemos garantir que os serviços possuam a mesma interface, ou seja, os mesmos tipos e número de parâmetros. Suponha que para o usuário que deseja o recurso simples de imagem seja indiferente a quantidade de \emph{megapixels} da imagem. Para o usuário basta apenas o provimento do serviço que retorne algum dado ou altere o estado de alguma aplicação. Dessa forma, devemos garantir que qualquer uma das três instâncias possam prover esse serviço. Uma câmera com \emph{flash}, por exemplo, para este cenário, seria equivalente a um recurso de câmera sem \emph{flash}.
\end{comment}

\section{Equivalência de Recursos}

Entende-se por equivalência tudo aquilo que tem o mesmo valor.

\begin{quote}
	Lucas estava brincando com seu carrinho, quando uma das rodinhas se quebrou. Entristecido, ele procurou por seu pai, que, ao ver a aflição de seu filho, resolveu ajudar. Recolheu uma garrafa PET que estava por perto, desenroscou sua tampinha e a colocou no lugar deixado pela rodinha estragada. Tamanha foi a felicidade de Lucas ao poder voltar a brincar com o seu carrinho.
\end{quote}

A história acima, apesar de simples, contém informações que podem ser úteis. Observe que os objetos envolvidos, ou seja, a rodinha e a tampinha, diferem na maioria das suas características: material, valor, peso, cor, etc. Apesar de tudo, pelo menos uma das características é igual: ambas possuem formatos cilíndricos. Foi essa única semelhança que permitiu o conserto do carrinho e garantiu a felicidade de Lucas! Ou seja, no contexto apresentado, a rodinha e a tampinha puderam ser trocadas sem perda de valor.

É exatamente esse o sentido ao se afirmar que dois recursos são equivalentes. Um não precisa ser igual ao outro, mas apenas possuir uma característica que, em determinado momento, possibilite a intercambialidade. Seria possível, por exemplo, utilizar um \emph{joystick} como um dispositivo apontador e, a qualquer momento, substitui-lo por um \emph{laser pointer} de forma que os serviços necessários continuassem sendo providos. Repare que tais dispositivos, assim como a rodinha e a tampinha, são bastante diferentes em diversas de suas características, mas ainda assim são capazes de fornecer serviços iguais, como mover uma seta pela tela.

\subsection{Cálculo de Equivalência}

Para garantir que determinado recurso seja equivalente a outro, deve-se considerar o conjunto de serviços presentes em cada recurso envolvido, bem como as suas relações. Admita que a notação $A \implies B$ denote que um recurso A é equivalente a um recurso B e considere as afirmações abaixo:

\begin{enumerate}
	\item $R1 \implies R2$;
	\item $R2 \implies R3$.
\end{enumerate}

A partir delas e, com base na figura~\ref{fig:calculoDeEquivalencia}, tome as seguintes regras:

\begin{figure}[ht]
	\center
	\includegraphics[scale=0.39]{imagens/calculoDeEquivalencia}
	\caption{Cálculo de equivalência: exemplo de recursos equivalentes.}
	\label{fig:calculoDeEquivalencia}
\end{figure}

\begin{itemize}
	\item Consistência de serviços

		Os serviços providos por R2 devem existir em R1. Observe que tal obrigação não impede que R1 contenha outros serviços desconhecidos por R2. Caso todos os serviços em R1 sejam iguais aos serviços em R2, tais recursos são iguais e não equivalentes.
	\item Consistência de interface

		As interfaces dos serviços providos por R2 que também estão em R1 devem ser idênticas. Isso implica em dizer que os nomes dos serviços devem ser iguais, bem como os seus parâmetros.

		Parte-se do princípio que não existirá interesse em camuflar um serviço malicioso ao expor uma interface compatível com a equivalência.
	\item Refência circular

		Não é permitida.
\end{itemize}

\begin{comment}
Para que um recurso R1 seja equivalente a um recurso R2, os serviços de R2 deverão estar presentes no conjunto de serviços de R1, mas R1 poderá ter outros serviços que não estão presentes em R2. Seja S(X), o conjunto de serviços do recurso X, então teríamos que $S(R1) \cap S(R2) = S(R2)$. 

\begin{figure}[ht]
	\center
	\includegraphics[scale=0.6]{imagens/equivalenciaDeRecursos}
	\caption{Exemplo de recursos equivalentes.}
	\label{fig:equivalenciaDeRecursos}
\end{figure}

Suponha uma situação em que $A \implies B$, $B \implies C$ e $C \implies A$, logo teríamos que $A \implies C$, o que seria uma relação circular, o que não faz sentido, pois concluiríamos que todos os recursos são iguais e não equivalentes. Devemos, portanto, garantir que não ocorra essa equivalência circular. Caso algum dispositivo queira registrar um novo \emph{driver} que cause essa inconsistência, devemos impedir que essa inconsistência possa ocorrer, ou então alertar o novo driver da ocorrência deste problema e não realizar seu registro.

\subsubsection{Consistência de Interface}

	A figura~\ref{fig:consistenciaInterface} mostra que o recurso R1 é equivalente ao recurso R3, mas embora possua um serviço de mesmo nome que o recurso R2,  seus parâmetros são diferentes, logo, não pode ser um mesmo serviço e os recursos não serão equivalentes..
	
	Para garantir que os recursos são equivalentes, devemos realizar três validações nas interfaces de cada serviço dos recursos:
	\begin{enumerate}
		\item Recurso:
			
			Os serviços devem pertencer ao mesmo recurso cujas classes padrões foram definidas anteriormente.
		
		\item Identificador do Serviço:

			Os serviços devem possuir os mesmos identificadores.

		\item Parâmetros:

			Os serviços devem possuir os mesmos parâmetros.
	\end{enumerate}

	Parte-se do princípio que não existirá interesse em camuflar um serviço malicioso ao expor uma interface compatível com a equivalência.

\begin{figure}[ht]
	\center
	\includegraphics[scale=0.8]{imagens/consistenciaInterface}
	\caption{Exemplo de inconsistência de interface.}
	\label{fig:consistenciaInterface}
\end{figure}
\end{comment}

\begin{comment}
----------------------------------------- REVER ----------------------------------------- \\
Desta forma, faz-se necessária uma maneira de classificar tais recursos imersos nos mais variados dispositivos presentes no \emph{smart space} e regidos pelo \emph{middleware}. Essa classificação facilitará o desenvolvimento de novos \emph{drivers} para futuras aplicações, pois tornará possivel a definição de interfaces pré-estabelecidas que representem classes de recursos. Outra vantagem decorrente é a possibilidade de seleção de recursos equivalentes, caso o provedor originalmente selecionado esteja indisponível. \\
----------------------------------------- REVER -----------------------------------------

COLOQUEI NO COMEÇO DA PROPOSTA
\end{comment}

\subsection{Árvore de Recursos}

Cada uma dessas classes servirá de raiz para uma árvore de recursos, ou seja, existirão diversas árvores, cada qual com outras classes agrupadas por funcionalidades. Para tornar tal estrutura mais clara, considere a figura~\ref{fig:arvoreDeRecursos}. Observe a hierarquia formada a partir da classe padrão \emph{Pointer}. Ela é responsável por agrupar todas as classes de recursos capazes de prover serviços relacionados aos apontadores existentes, por exemplo, \emph{click} e \emph{scroll}. O mesmo acontece com as demais classes padrões. A estrutura em árvore é importante pois permite uma melhor organização lógica das classes relacionadas, além de permitir uma fácil navegabilidade entre elas.

\begin{figure}[ht]
	\center
	\includegraphics[scale=0.35]{imagens/hierarquiaDeRecursos}
	\caption{Exemplo da estrutura de árvores de recursos.}
	\label{fig:arvoreDeRecursos}
\end{figure}

Uma vez que tal estrutura esteja construída, torna-se simples descobrir todos os recursos equivalentes a quaisquer classes presentes na árvore. Para tal, basta tomar a subárvore cuja raiz é o elemento ao qual deseja-se obter seus equivalentes. Observe a figura~\ref{fig:tutorialDeEquivalencia} e imagine que seja necessário oferecer todos os recursos equivalentes à classe \emph{Clickable}. Ao se colocar tal classe como raiz de sua subárvore, torna-se fácil perceber que as classes \emph{Joystick}, \emph{Trackball}, \emph{Mouse} e \emph{TouchScreen} são suas equivalentes.

\begin{figure}[ht]
	\center
	\includegraphics[scale=0.55]{imagens/tutorialDeEquivalencia}
	\caption{Exemplo de como se encontrar recursos equivalentes.}
	\label{fig:tutorialDeEquivalencia}
\end{figure}
\section{Impacto no UbiquitOS}

A proposta anteriomente definida requer algumas mudanças no \emph{middleware}. Esta seção mostrará os impactos do uso da classificação para a determinação da equivalência de recursos.

\subsection{DSOA - \emph{Device Service Oriented Architecture}}

O conceito do Recurso introduzido anteriormente na subseção ~\ref{subsec:introUos} do capítulo ~\ref{cap:classificacao} será afetado devido a proposta de classificação de recursos. Um dispositivo presente no \emph{smart space}, segundo a DSOA, procurava seus serviços no ambiente por meio do identificador do recurso que provê esses serviços. Com a introdução do conceito de Equivalência de Recursos, o dispositivo irá procurar por uma classe desse recurso. O identificador, em vez de representar um nome do recurso, irá, portanto, identificar que tipo de recurso ele representa, ou seja, recurso de: imagem, áudio, apontador e etc.
\subsection{\emph{uP - Ubiquitous Protocols}}

Conjunto de protocolos desenvolvidos para estabelecer um meio de interação entre os serviços levando em consideração a arquitetura DSOA. Tais protocolos definem o canal de comunicação e a forma de interação entre as entidades do ambiente. As mensagens são transmitidas no formato JSON (\emph{JavaScript Object Notation})~\cite{json}.

\begin{comment}
, que utiliza a codificação UTF-8~\cite{utf8}, que foi escolhido por ser um formato estruturado, leve e independente de plataforma. O JSON foi utilizado ante o XML, pois possui menor tamanho de mensagens e esse fator pode ser decisivo em um ambiente com diversos dipositivos com capacidades computacionais diferentes e possivelmente reduzidas. Dessa forma a limitação dos dispositivos é minimizada e exclui a necessidade de uma rede para tratamento dessas mensagens.
\end{comment}

Cada um dos conceitos apresentados na DSOA possui uma representação no \emph{uP} com seus respectivos atributos. A seguir será mostrado como era o \emph{uP} antes da introdução do conceito de equivalência de recursos.

\begin{itemize}
	\item Dispositivo(\emph{UpDevice}):
	
		Por meio dos seguintes atributos, é possível identificar unicamente o dispositivo no ambiente, e quais são as interfaces rede que o dispositivo possui para realizar alguma comunicação:
		\begin{itemize}
			\item \emph{``name''}: 
			
			Identificador do dispositivo;
			\item \emph{``networks''}: 

			Lista de interfaces de rede do dispositivo. Cada interface é composta pelo tipo de rede e endereço do dispositivo.
		\end{itemize}
	\item \emph{Driver}(\emph{UpDriver}): 

		Representa o conceito do Recurso definido na DSOA. Como um dispositivo pode ter várias instâncias de um recurso, cada instância é identificada unicamente dentro do dispositivo que contém este recurso. O \emph{driver} é composto por:
		\begin{itemize}
			\item \emph{``name''}:

				Identificador do recurso no ambiente;
			\item \emph{``services''}:
				
				Lista de serviços síncronos do recurso;
			\item \emph{``events''}:
				
				Lista de serviços assíncronos do recurso.
		\end{itemize}
	\item Serviço(\emph{UpService}): 

		Representa o conceito de mesmo nome definido na DSOA. Sua interface é composta pelos seguintes atributos:
		\begin{itemize}
			\item \emph{``name''}:

				Identificador do serviço disponível no recurso;
			\item \emph{``parameters''}:
				
				Lista de parâmetros necessários para que o serviço seja executado. Esses parâmetros podem ser de dois tipos:
				\begin{enumerate}
					\item Opcional (\emph{OPTIONAL});
					\item Obrigatório (\emph{MANDATORY}).
				\end{enumerate}
		\end{itemize}
\end{itemize}

\subsubsection{Tipo de mensagens}

Para que possa haver interação entre as entidades no ambiente, o \emph{uP} especifica quatro tipos de mensagens e, caso exista necessidade, podem ser personalizadas:

\begin{itemize}
	\item \emph{Service Call}: 

		Mensagem síncrona do \emph{uP}. Seus parâmetros são definidos por meio de suas propriedades, podendo ser obrigatórias ou opcionais:
		\begin{itemize}
			\item Obrigatórias:
				\begin{itemize}
					\item \emph{\bf{type}}: 

						Representa o tipo da mensagem. No caso desta mensagem, seu valor será \emph{``SERVICE\underline{ }CALL\underline{ }REQUEST''};
					\item \emph{\bf{driver}}: 

						Recurso do serviço requisitado;
					\item \emph{\bf{service}}:

						Nome do serviço requisitado;
					\item \emph{\bf{parameters}}:

						Contém os parâmetros do serviço.
				\end{itemize}
			\item Opcionais:
				\begin{itemize}
					\item \emph{\bf{instanceId}}:

						Representa o identificador único da instância do \emph{driver} do serviço requisitado;
					\item \emph{\bf{serviceType}}:
						
						Representa a forma de transmissão de dados;
					\item \emph{\bf{channelIDs}}:

						Representa os identificadores dos canais criados para a comunicação;
					\item \emph{\bf{channelType}}:

						Representa o tipo da rede utilizada na comunicação.
				\end{itemize}
		\end{itemize}
	\item \emph{Service Response}: 

		Mensagem que carrega dados da resposta dada pela execução de um serviço chamado pelo \emph{Service Call}. Essa mensagem possui as seguintes propriedades:
		\begin{itemize}
			\item \emph{\bf{type}}: 

				Representa o tipo da mensagem. No caso desta mensagem, seu valor será \emph{``SERVICE\underline{ }CALL\underline{ }RESPONSE''};
			\item \emph{\bf{responseData}}: 

				Parâmetro opcional que contem os dados da resposta da execução do serviço.
		\end{itemize}
	\item \emph{Notify}: 

		Similar ao \emph{Service Response}, porém carrega dados de notificações de eventos, ou seja, respostas de uma requisição assíncrona de um serviço. Suas propriedades são:
		\begin{itemize}
			\item \emph{\bf{type}}: 

				Representa o tipo da mensagem. No caso desta mensagem, seu valor será \emph{``NOTIFY''};
			\item \emph{\bf{driver}}: 

				Recurso do serviço requisitado;
			\item \emph{\bf{instanceId}}: 

				Representa o identificador da instância onde ocorreu evento;
			\item \emph{\bf{parameters}}: 

				Parâmetro opcional com informações sobre o evento.
		\end{itemize}
	\item \emph{Encapsulated Message}: 

		Utilizada para permitir mensagens codificadas. Esse tipo de mensagem possui as seguintes propriedades:
		\begin{itemize}
			\item \emph{\bf{type}}: 

				Representa o tipo da mensagem. No caso desta mensagem, seu valor será \emph{``ENCAPSULATED\underline{ }MESSAGE''};
			\item \emph{\bf{securityType}}: 

				Identificador do tipo de codificação utilizada;
			\item \emph{\bf{innerMessage}}: 

				Mensagem codificada.
		\end{itemize}
\end{itemize}

\subsubsection{Protocolos}

Com o objetivo de disponibilizar as características de visibilidade, interação e efeito, foram criados mecanismos de acesso aos recursos. O \emph{uP} é dividido em protocolos básicos, que são utilizados para invocar serviços e protocolos complementares que completam suas  funcionalidas.

\begin{itemize}
	\item Protocolos Básicos: 

		São divididos em SCP (\emph{Service Call Protocol}) e EVP(\emph{Event Protocol}). O primeiro possui arquitetura provedor-consumidor, ou seja, o consumidor requisita um serviço e o provedor retorna esse serviço para ele de forma síncrona. No último, o consumidor se registra em um evento no provedor, que response informando que o registro foi realizado com sucesso, e na ocorrência do evento, o consumidor é notificado pelo provedor, demonstrando uma forma assíncrona de comunicação.
	\item Protocolos Complementares:

		Além da interação entre os serviços provida pelos protocolos básicos, o \emph{uP} tem mecanismos que permitem que as aplicações obtenham informações sobre quais são os serviços disponíveis no ambiente e informações a respeito deles. Os protocolos são divididos em um grupo de protocolos com informações sobre o dispositivo e um grupo de protocolos com informações sobre o ambiente. Esses grupos são mapeados em um \emph{driver} que definem sua interface de comunicação. São protocolos complementares:
	\begin{itemize}
		\item \emph{Device Driver}: 

			Responsável por disponibilizar informações a respeito do dispositivo. Possui os seguintes serviços:
			\begin{itemize}
				\item \emph{ListDrivers}: 

					Provê uma lista de instâncias dos drivers disponíveis do dispositivo. Possui dois parâmetros opcionais:
					\begin{itemize}
						\item \emph{\bf{serviceName}}: 

							Nome do serviço;
						\item \emph{\bf{driverName}}: 

							Identificador do recurso.
					\end{itemize}
				\item \emph{Handshake}: 

					Neste protocolo, dois dispositivos trocam informações entre-si. O dispositivo que invoca esse serviço passa como parâmetro um objeto do tipo \emph{device} e recebe como retorno informações sobre o dispositivo que recebeu a chamada;
				\item \emph{Goodbye}: 

					Responsável por retirar o dispositivo da lista de dispositivos presentes no ambiente;
				\item \emph{Authenticate}: 

					Estabelece um contexto de segurança entre dois dispositivos por meio de um prévio compartilhamento de chaves.
			\end{itemize}
		\item \emph{Register Driver}: 

			Responsável por disponibilizar informações que um dispositivo possui sobre o ambiente.
			\begin{itemize}
				\item \emph{ListDrivers}: 

					Provê uma lista de instâncias dos drivers disponíveis no ambiente, geralmente de dispositivos vizinhos. Possui três parâmetros opcionais:
					\begin{itemize}
						\item \emph{\bf{serviceName}}: 

							Nome do serviço;
						\item \emph{\bf{driverName}}: 

							Identificador do recurso;
						\item \emph{\bf{device}}: 
							
							Nome do dispositivo que contém o recurso.
					\end{itemize}
				\item \emph{Publish}: 

					Responsável pela publicação de uma instância de um driver a ser disponibilizada no ambiente.
				\item \emph{UnPublish}: 

					Responsável por retirar as informações sobre uma instância de um driver.
			\end{itemize}
	\end{itemize}
\end{itemize}

O impacto sofrido na DSOA reflete diretamente no \emph{uP}. A visão do dispositivo no \emph{uOS} (\emph{UpDevice}) não sofrerá alterações, mas para que a equivalência de serviços possa ser garantida, os recursos do dispositivo, cada um representado por um \emph{UpDriver}, deverá informar o recurso a que este é equivalente. Além disso, o identificador do recurso representará a classe deste. O \emph{UpDriver} ficará, portanto, com os seguintes campos:

\begin{itemize}
	\item \emph{``name''}:
		
		Representa a classe do recurso.

	\item \emph{``equivalentDrivers''}:
	
		Classes a que este recurso é equivalente.
	\item \emph{``services''}:

		Lista dos serviços síncronos do recurso.

	\item \emph{``events''}:

		Lista dos serviços assíncronos do recurso;
\end{itemize}

O campo \emph{``equivalentDrivers''} irá carregar a informação sobre a equivalência de um recurso. Entretanto, em um ambiente dinâmico, existe a possibilidade do \emph{uOS} não conhecer um destes recursos equivalentes. O dispositivo, então, ficará responsável por prover um serviço que informe ao \emph{uOS} a interface deste(s) recurso(s) desconhecido(s) e sua(s) equivalência(s) até a raiz conhecida pelo \emph{uOS}, para que o \emph{middleware} possa encontrar registrar este novo driver na árvore de equivalência de recursos. Para isso, deverá ser criado mais um serviço no protocolo \emph{Device Driver} que ficará com os seguintes serviços:

\begin{itemize}
	\item \emph{ListDrivers};
	\item \emph{Handshake};
	\item \emph{Goodbye};
	\item \emph{Authenticate};
	\item \emph{TellEquivalentDrivers}:

		Responsável por informar a interface do(s) recurso(s) desconhecido(s) equivalente(s). Será composto pelos seguintes campos:

		\begin{itemize}
			\item \emph{``driversName''}:

			Lista com o(s) nome(s) do(s) recurso(s).

			\item \emph{``interfaces''}:

			Lista com a(s) interface(s) do(s) recurso(s).
		\end{itemize}
\end{itemize}

São afetados, ainda, dois protocolos básicos, o \emph{Service Call} e o \emph{Notify}, e o serviço \emph{ListDrivers} dos protocolos \emph{Device Driver} e \emph{Register Driver}. Todos esses serviços contém o parâmetro \emph{driver} que passará a representar uma classe dentre as classes de recursos definidads anteriormente e não mais o nome do recurso simplesmente.
\subsection{\emph{uOS}}

Além dos protocolos, que fazem parte do \emph{middleware}, o núcleo do \emph{uOS} ficará responsável por garantir as regras que definirão se dois recursos são equivalentes e, por conseguinte, pelas validações das interfaces dos serviços. Quando um dispostivo entra no ambiente inteligente controlado pelo \emph{uOS}, o \emph{middleware} tenta trocar informações com o dispositivo por meio do serviço ``\emph{handshake}'' do protocolo \emph{DeviceDriver}. Após o ``\emph{handshake}'', o \emph{uOS} busca informações sobre os recursos que aquele dispositivo carrega. Essas informações são obtidas por meio do serviço ``\emph{listDrivers}''. Antes da introdução da equivalência de recursos, o registro de novos dispositivos parava neste segundo passo. 

O \emph{uOS} a partir de então irá, além de registrar informações sobre o recurso, irá válidar se esse recurso é de fato equivalente à outros recursos analisando uma hierarquia de equivalência. Os recursos definidos neste trabalho representam diferentes raízes na árvore de equivalência. Uma vez que um novo \emph{driver} é implementado e um dispositivo dotado deste recurso adentra ao ambiente inteligente controlado pelo \emph{uOS}, este novo \emph{driver} passa a integrar um novo ``galho'' na árvore de equivalência, caso a definição de seus serviços esteja de acordo com as regras de equivalência e consistência de interfaces.


