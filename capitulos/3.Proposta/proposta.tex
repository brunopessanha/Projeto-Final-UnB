\chapter{Classificação Proposta}

Neste capítulo falaremos sobre a Classificação de Dispositivos proposta. Essa classificação deverá ser relacionada, extensível e um dispositivo deverá poder fazer parte de múltiplas classes. A classificação relacionada facilita a implementação de novos \emph{drivers} para o \emph{uOS} que poderão se aproveitar das interfaces já existentes. Ser extensível, pois permite uma relação de especialização entre diferentes classes. A capacidade de permitir que um dispositivo pertença à diferentes classes, garante uma flexibilidade para dispositivos com diversos recursos poderem se encaixar nas classificações padrões sem a necessidade da definição de uma nova classe.


\section{A Classificação}
Levando em consideração os fatores acima apresentados e analisando a tabela~\ref{tab:comparativoClasses}, escolhemos as classes que se fazem presente em todos os padrões estudados para definir quais classes serão utilizadas na classificação de dispositivos para o \emph{uOS}. São elas:

\begin{itemize}
	\item Áudio:
		Abrange qualquer tipo de dispositivos que contenham um recurso de som. Exemplo: TVs, celulares, aparelhos de som.
	\item Vídeo:
		Abrange qualquer tipo de dispositivos que possuam a capacidade de prover um recurso de vídeo. Exemplo: TVs, celulares, projetores.
	\item Imagem:
		Abrange qualquer tipo de dispositivos que possam prover um recurso de imagem. Exemplo: Câmeras digitais, \emph{Scanners}, Impressoras.
	\item Teclado:
		Abrange qualquer tipo de dispositivos que possam prover um recurso de teclado. Exemplo: Teclados, \emph{notebooks}, celulares, \emph{tablets}.
	\item Apontador:
		Abrange qualquer tipo de dispositivos que possuam o recurso de apontadores. Exemplo: \emph{Mouses}, telas sensíveis ao toque, \emph{joysticks}.W
	\item Conectividade:
		Abrange qualquer tipo de dispositivos com recurso de conexão. Exemplo: Modens, Roteadores.
\end{itemize}