\subsection{\emph{uOS}}

Além dos protocolos, que fazem parte do \emph{middleware}, o núcleo do \emph{uOS} ficará responsável por garantir as regras que definirão se dois recursos são equivalentes e, por conseguinte, pelas validações das interfaces dos serviços. Quando um dispostivo entra no ambiente inteligente controlado pelo \emph{uOS}, o \emph{middleware} tenta trocar informações com o dispositivo por meio do serviço ``\emph{handshake}'' do protocolo \emph{DeviceDriver}. Após o ``\emph{handshake}'', o \emph{uOS} busca informações sobre os recursos que aquele dispositivo carrega. Essas informações são obtidas por meio do serviço ``\emph{listDrivers}''. Antes da introdução da equivalência de recursos, o registro de novos dispositivos parava neste segundo passo. 

O \emph{uOS} a partir de então irá, além de registrar informações sobre o recurso, irá válidar se esse recurso é de fato equivalente à outros recursos analisando uma hierarquia de equivalência. Os recursos definidos neste trabalho representam diferentes raízes na árvore de equivalência. Uma vez que um novo \emph{driver} é implementado e um dispositivo dotado deste recurso adentra ao ambiente inteligente controlado pelo \emph{uOS}, este novo \emph{driver} passa a integrar um novo ``galho'' na árvore de equivalência, caso a definição de seus serviços esteja de acordo com as regras de equivalência e consistência de interfaces.