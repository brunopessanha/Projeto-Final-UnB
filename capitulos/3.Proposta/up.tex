\subsection{\emph{uP - Ubiquitous Protocols}}

Conjunto de protocolos desenvolvidos para estabelecer um meio de interação entre os serviços levando em consideração a arquitetura DSOA. Tais protocolos definem o canal de comunicação e a forma de interação entre as entidades do ambiente. As mensagens são transmitidas no formato JSON (\emph{JavaScript Object Notation})~\cite{json}.

\begin{comment}
, que utiliza a codificação UTF-8~\cite{utf8}, que foi escolhido por ser um formato estruturado, leve e independente de plataforma. O JSON foi utilizado ante o XML, pois possui menor tamanho de mensagens e esse fator pode ser decisivo em um ambiente com diversos dipositivos com capacidades computacionais diferentes e possivelmente reduzidas. Dessa forma a limitação dos dispositivos é minimizada e exclui a necessidade de uma rede para tratamento dessas mensagens.
\end{comment}

Cada um dos conceitos apresentados na DSOA possui uma representação no \emph{uP} com seus respectivos atributos. A seguir será mostrado como era o \emph{uP} antes da introdução do conceito de equivalência de recursos.

\begin{itemize}
	\item Dispositivo(\emph{UpDevice}):
	
		Por meio dos seguintes atributos, é possível identificar unicamente o dispositivo no ambiente, e quais são as interfaces rede que o dispositivo possui para realizar alguma comunicação:
		\begin{itemize}
			\item \emph{``name''}: 
			
			Identificador do dispositivo;
			\item \emph{``networks''}: 

			Lista de interfaces de rede do dispositivo. Cada interface é composta pelo tipo de rede e endereço do dispositivo.
		\end{itemize}
	\item \emph{Driver}(\emph{UpDriver}): 

		Representa o conceito do Recurso definido na DSOA. Como um dispositivo pode ter várias instâncias de um recurso, cada instância é identificada unicamente dentro do dispositivo que contém este recurso. O \emph{driver} é composto por:
		\begin{itemize}
			\item \emph{``name''}:

				Identificador do recurso no ambiente;
			\item \emph{``services''}:
				
				Lista de serviços síncronos do recurso;
			\item \emph{``events''}:
				
				Lista de serviços assíncronos do recurso.
		\end{itemize}
	\item Serviço(\emph{UpService}): 

		Representa o conceito de mesmo nome definido na DSOA. Sua interface é composta pelos seguintes atributos:
		\begin{itemize}
			\item \emph{``name''}:

				Identificador do serviço disponível no recurso;
			\item \emph{``parameters''}:
				
				Lista de parâmetros necessários para que o serviço seja executado. Esses parâmetros podem ser de dois tipos:
				\begin{enumerate}
					\item Opcional (\emph{OPTIONAL});
					\item Obrigatório (\emph{MANDATORY}).
				\end{enumerate}
		\end{itemize}
\end{itemize}


\subsubsection{Protocolos}

Com o objetivo de disponibilizar as características de visibilidade, interação e efeito, foram criados mecanismos de acesso aos recursos. O \emph{uP} é dividido em protocolos básicos, que são utilizados para invocar serviços e protocolos complementares que completam suas  funcionalidas.

\begin{itemize}
	\item Protocolos Básicos: 

		São divididos em SCP (\emph{Service Call Protocol}) e EVP(\emph{Event Protocol}). O primeiro possui arquitetura provedor-consumidor, ou seja, o consumidor requisita um serviço e o provedor retorna esse serviço para ele de forma síncrona. No último, o consumidor se registra em um evento no provedor, que response informando que o registro foi realizado com sucesso, e na ocorrência do evento, o consumidor é notificado pelo provedor, demonstrando uma forma assíncrona de comunicação.
	\item Protocolos Complementares:

		Além da interação entre os serviços provida pelos protocolos básicos, o \emph{uP} tem mecanismos que permitem que as aplicações obtenham informações sobre quais são os serviços disponíveis no ambiente e informações a respeito deles. Os protocolos são divididos em um grupo de protocolos com informações sobre o dispositivo e um grupo de protocolos com informações sobre o ambiente. Esses grupos são mapeados em um \emph{driver} que definem sua interface de comunicação. São protocolos complementares:
	\begin{itemize}
		\item \emph{Device Driver}: 

			Responsável por disponibilizar informações a respeito do dispositivo. Possui os seguintes serviços:
			\begin{itemize}
				\item \emph{ListDrivers}: 

					Provê uma lista de instâncias dos drivers disponíveis do dispositivo. Possui dois parâmetros opcionais:
					\begin{itemize}
						\item \emph{\bf{serviceName}}: 

							Nome do serviço;
						\item \emph{\bf{driverName}}: 

							Identificador do recurso.
					\end{itemize}
				\item \emph{Handshake}: 

					Neste protocolo, dois dispositivos trocam informações entre-si. O dispositivo que invoca esse serviço passa como parâmetro um objeto do tipo \emph{device} e recebe como retorno informações sobre o dispositivo que recebeu a chamada;
				\item \emph{Goodbye}: 

					Responsável por retirar o dispositivo da lista de dispositivos presentes no ambiente;
				\item \emph{Authenticate}: 

					Estabelece um contexto de segurança entre dois dispositivos por meio de um prévio compartilhamento de chaves.
			\end{itemize}
		\item \emph{Register Driver}: 

			Responsável por disponibilizar informações que um dispositivo possui sobre o ambiente.
			\begin{itemize}
				\item \emph{ListDrivers}: 

					Provê uma lista de instâncias dos drivers disponíveis no ambiente, geralmente de dispositivos vizinhos. Possui três parâmetros opcionais:
					\begin{itemize}
						\item \emph{\bf{serviceName}}: 

							Nome do serviço;
						\item \emph{\bf{driverName}}: 

							Identificador do recurso;
						\item \emph{\bf{device}}: 
							
							Nome do dispositivo que contém o recurso.
					\end{itemize}
				\item \emph{Publish}: 

					Responsável pela publicação de uma instância de um driver a ser disponibilizada no ambiente.
				\item \emph{UnPublish}: 

					Responsável por retirar as informações sobre uma instância de um driver.
			\end{itemize}
	\end{itemize}
\end{itemize}

O \emph{uP} foi criado a partir dos conceitos da DSOA, dessa forma, o impacto sofrido na DSOA é diretamente refletido sobre ele. A visão do dispositivo no \emph{uOS} (\emph{UpDevice}) não sofrerá alterações, mas para que a equivalência de serviços possa ser garantida, os recursos do dispositivo, cada um representado por um \emph{UpDriver}, deverá informar o recurso a que este é equivalente. Além disso, o identificador do recurso representará a classe deste. O \emph{UpDriver} ficará, portanto, com os seguintes campos:

\begin{itemize}
	\item \emph{``name''}:
		
		Representa a classe do recurso.

	\item \emph{``equivalentDrivers''}:
	
		Classes a que este recurso é equivalente.
	\item \emph{``services''}:

		Lista dos serviços síncronos do recurso.

	\item \emph{``events''}:

		Lista dos serviços assíncronos do recurso;
\end{itemize}

O campo \emph{``equivalentDrivers''} irá carregar a informação sobre a equivalência de um recurso. Entretanto, em um ambiente dinâmico, existe a possibilidade do \emph{uOS} não conhecer um destes recursos equivalentes. O dispositivo, então, ficará responsável por prover um serviço que informe ao \emph{uOS} a interface deste(s) recurso(s) desconhecido(s) e sua(s) equivalência(s) até a raiz conhecida pelo \emph{uOS}, para que o \emph{middleware} possa encontrar e registrar este novo driver na árvore de recursos. Para isso, deverá ser criado mais um serviço no protocolo \emph{Device Driver} que ficará com os seguintes serviços:

\begin{itemize}
	\item \emph{ListDrivers};
	\item \emph{Handshake};
	\item \emph{Goodbye};
	\item \emph{Authenticate};
	\item \emph{TellEquivalentDrivers}:

		Responsável por informar a interface do(s) recurso(s) desconhecido(s) equivalente(s). Será composto pelos seguintes campos:

		\begin{itemize}
			\item \emph{``driversName''}:

			Lista com o(s) nome(s) do(s) recurso(s).

			\item \emph{``interfaces''}:

			Lista com a(s) interface(s) do(s) recurso(s).
		\end{itemize}
\end{itemize}

Serão afetados, ainda, dois protocolos básicos, o \emph{Service Call} e o \emph{Notify}, e o serviço \emph{ListDrivers} dos protocolos \emph{Device Driver} e \emph{Register Driver}. Todos esses serviços contém o parâmetro \emph{driver} que passará a representar uma classe dentre as classes de recursos e não mais o nome do recurso simplesmente.