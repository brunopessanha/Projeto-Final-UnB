\subsection{\emph{uP - Ubiquitous Protocols}}

O impacto sofrido na DSOA reflete diretamente no \emph{uP}. A visão do dispositivo no \emph{uOS} (\emph{UpDevice}) não sofrerá alterações, mas para que a equivalência de serviços possa ser garantida, os recursos do dispositivo, cada um representado por um \emph{UpDriver}, deverá informar o recurso a que este é equivalente. Além disso, o identificador do recurso representará a classe deste. O \emph{UpDriver} ficará, portanto, com os seguintes campos:

\begin{itemize}
	\item \emph{``name''}:
		
		Representa a classe do recurso.

	\item \emph{``equivalentDrivers''}:
	
		Classes a que este recurso é equivalente.
	\item \emph{``services''}:

		Lista dos serviços síncronos do recurso.

	\item \emph{``events''}:

		Lista dos serviços assíncronos do recurso;
\end{itemize}

O campo \emph{``equivalentDrivers''} irá carregar a informação sobre a equivalência de um recurso. Entretanto, em um ambiente dinâmico, existe a possibilidade do \emph{uOS} não conhecer um destes recursos equivalentes. O dispositivo, então, ficará responsável por prover um serviço que informe ao \emph{uOS} a interface deste(s) recurso(s) desconhecido(s) e sua(s) equivalência(s) até a raiz conhecida pelo \emph{uOS}, para que o \emph{middleware} possa encontrar registrar este novo driver na árvore de equivalência de recursos. Para isso, deverá ser criado mais um serviço no protocolo \emph{Device Driver} que ficará com os seguintes serviços:

\begin{itemize}
	\item \emph{ListDrivers};
	\item \emph{Handshake};
	\item \emph{Goodbye};
	\item \emph{Authenticate};
	\item \emph{TellEquivalentDrivers}:

		Responsável por informar a interface do(s) recurso(s) desconhecido(s) equivalente(s). Será composto pelos seguintes campos:

		\begin{itemize}
			\item \emph{``driversName''}:

			Lista com o(s) nome(s) do(s) recurso(s).

			\item \emph{``interfaces''}:

			Lista com a(s) interface(s) do(s) recurso(s).
		\end{itemize}
\end{itemize}

São afetados, ainda, dois protocolos básicos, o \emph{Service Call} e o \emph{Notify}, e o serviço \emph{ListDrivers} dos protocolos \emph{Device Driver}  e \emph{Register Driver}. Todos esses serviços contém o parâmetro \emph{driver} que passará a representar uma classe dentre as classes de recursos definidads anteriormente e não mais o nome do recurso simplesmente.