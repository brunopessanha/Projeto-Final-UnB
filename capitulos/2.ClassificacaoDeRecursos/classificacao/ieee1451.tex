\subsection{IEEE 1451}
O IEEE 1451 se divide em uma família de padrões que foram criados com os objetivos de permitir a capacidade de comunicação entre transdutores (sensores e atuadores) de forma \emph{plug-and-play} por meio de redes com ou sem fio, facilitar a criação de transdutores com inteligência embarcada, simplificar a configuração e manutenção de sistemas, prover comunicação entre transdutores legados, e por fim, habilitar a implementação de transdutores inteligentes e com uso mínimo de memória~\cite{ieee1451journal}.

Um dos padrões da família, o IEEE 1451.1, define um modelo de informação para o que é chamado de\emph{Network Capable Application Processors} (NCAP). Sendo estes compostos  por um modelo de objetos comum e interfaces de componentes da rede de transdutores. Dessa forma, foi desenvolvido um \emph{framework} orientado a objetos que pode ser estendido para facilitar o desenvolvimento de aplicações. Este modelo foi definido da seguinte forma: Um modelo de dados que especifica a forma e o tipo de comunicação, tanto local quanto remota, por meio das interfaces de objetos 1451.1, um modelo de objetos que especifica tipos de componentes de software usados para definir e implementar sistemas e, por fim, modelos de comunicação que definem a sintaxe e semântica das interfaces de software entre redes de comunicação e objetos de aplicação~\cite{ieeeOO1451}~\cite{ieee1451monitoring}.


O padrão~\cite{ieee1451standard} especifica cada classe no modelo definindo as interfaces da classe (por meio de assinaturas e operações) e o comportamento da classe (via texto ou máquinas de estado). A Listagem~\ref{dispositivos1451} apresenta a hierarquia de classes definida pelo padrão:

\lstset{caption={Classes de Dispositivos do Padrão IEEE 1451},label=dispositivos1451}
\begin{lstlisting}[frame=single]
Root
	Entity
		Block
			NCAP BLOCK
			Function Block
			Base Transducer Block
			Transducer Block
				Dot2 Transducer Block
				Dot3 Transducer Block
				Dot4 Transducer Block
		Component
			Parameter
				Parameter With Update
					Physical Parameter
						Scalar Parameter
							Scalar Series Parameter
						Vector Parameter
							Vector Series Parameter
				Time Parameter
			Action
			File
				Partitioned File
			Component Group
		Service
			Base Port
				Base Client Port
					Client Port
					Asynchronous Client Port
				Base Publisher Port
					Publisher Port
					Self Identifying Publisher Port
					Event Generator Publisher Port
			Subscriber Port
			Mutex Service
			Condition Variable Service
\end{lstlisting}

\begin{comment}
\begin{tabbing}
\textit{Root} \= \\
\> \textit{Entity} \= \\
\> \> \textit{Block} \= \\
\> \> \> NCAP BLOCK \\
\> \> \> \textit{Function Block} \\
\> \> \> \textit{Base Transducer Block} \\
\> \> \> \textit{Transducer} \= \textit{Block} \\
\> \> \> \> Dot2 Transducer Block \\
\> \> \> \> Dot3 Transducer Block \\
\> \> \> \> Dot4 \= Transducer Block \\
\> \> \textit{Component} \\
\> \> \> Parameter \\
\> \> \> \> Parameter With Update \\
\> \> \> \> \> \= \textit{Physical Parameter} \\
\> \> \> \> \> \> Scalar \= Parameter \\
\> \> \> \> \> \> \> Scalar Series Parameter \\
\> \> \> \> \> \> Vector Parameter \\
\> \> \> \> \> \> \> Vector Series Parameter \\
\> \> \> \> Time Parameter \\
\> \> \> Action \\
\> \> \> File \\
\> \> \> \> Partitioned File \\
\> \> \> Component Group \\
\> \> \textit{Service} \\
\> \> \> \textit{Base Port} \\
\> \> \> \> \textit{Base Client Port} \\
\> \> \> \> \> Client Port \\
\> \> \> \> \> Asynchronous Client Port \\
\> \> \> \> \textit{Base Publisher Port} \\
\> \> \> \> \> Publisher Port \\
\> \> \> \> \> Self Identifying Publisher Port \\
\> \> \> \> \> \> \> Event Generator Publisher Port \\
\> \> \> Subscriber Port \\
\> \> \> Mutex Service \\
\> \> \> Condition Variable Service \\
\end{tabbing}
\end{comment}

Os três tipos principais de objetos IEEE 1451.1 são:

\begin{itemize}
	\item\emph{Block}:

	Especializada em três classes:
		\begin{itemize}
			\item\emph{NCAPBlock}:

				Provê interfaces para comunicações de rede e configurações do sistema. Cada NCAP é modelado para possuir ao menos um processo de software. Cada processo de processo de software possuirá exatamente um \emph{NCAPBlock}.
			\item\emph{BaseTransducerBlock}:

				Provê interfaces entre transdutores e funções.
			\item\emph{FunctionBlock}:

				Provê encapsulamento de funções específicas.
		\end{itemize}
	
	\item\emph{Component}:
	
		Fornecem:
		\begin{itemize}
			\item Informações estruturadas: medidas e arquivos.
			\item Coleções de objetos relacionados com a aplicação.
			\item Ações com estados onde a ação é executada após um período de tempo.
		\end{itemize}
	\item\emph{Service}:
	
		Suportam:
		\begin{itemize}
			\item Comunicação entre objetos de diferentes NCAPs.
			\item Sincronização do sistema.
		\end{itemize}
\end{itemize}

Há ainda as classes não que fazem parte do padrão IEEE 1451.1, que não estão representadas na hierarquia acima e possuem restrições de aplicabilidade na arquitetura IEEE 1451. Este padrão possui a limitação de ter sido criado para sensores e atuadores, e, portanto, embora possua um relacionamento entre as classes, elas não podem ser estendidas. Como forma de representação, esse padrão utiliza a \emph{Interface Description Language} (IDL).


