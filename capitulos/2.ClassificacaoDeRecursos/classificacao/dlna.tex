\subsection{DLNA}

A \emph{Digital Living Network Alliance} (DLNA) é uma organização composta pelas principais empresas de eletrônicos de consumo, computação e dispositivos móveis, tais como: Microsoft, Sony, Nokia, Samsung, Cisco entre outros. Foi fundada em 2003 e tem como objetivo fornecer orientações para permitir a interoperabilidade entre dispositivos para completar a convergência da indústria digital, levando, dessa forma, inovação, simplicidade e valor aos consumidores~\cite{dlnaoverview}. Tem como visão facilitar a criação, o gerenciamento e o compartilhamento de conteúdo digital (fotos, músicas e vídeos) entre os dispositivos pertencentes à mesma rede~\cite{dlnahdvideostreaming}. Deve permitir, por exemplo~\cite{dlnaoverview}:

\begin{itemize}
	\item facilmente adquirir, armazenar e acessar música digital a partir de praticamente qualquer lugar da casa;
	\item facilmente gerenciar, visualizar, imprimir e compartilhar fotos digitais;
	\item transportar seu conteúdo favorito a partir de qualquer lugar, mesmo que as pontas envolvidas estejam em movimento;
	\item aproveitar a gravação e reprodução de conteúdo distribuído e multi-usuário.
\end{itemize}

Suas diretrizes destacam casos de uso construídos para redes domésticas, funções adicionais que aumentam a experiência do compartilhamento de conteúdo e doze classes de dispositivos espalhados na categoria ``Rede Doméstica e Dispositivos Móveis''. A classificação de um dispositivo é feita de tal forma que um único aparelho multifuncional pode possuir diversas categorias diferentes~\cite{dlnahdvideostreaming, dlnaclasses}:

\begin{itemize}
	\item \emph{Home Network Devices} (HND)
	\begin{itemize}
		\item \emph{Digital Media Server} (DMS): 

		Esses dispositivos armazenam e disponibilizam conteúdo para \emph{Digital Media Players} (DMP) e \emph{Digital Media Renderes} que estão conectados na rede. Alguns servidores de mídia podem ainda proteger o conteúdo do usuário uma vez que ele tenha sido armazenado. Exemplo: PCs e dispositivos de armazenamento via rede(NAS);
		\item \emph{Digital Media Player} (DMP): 

		Esses dispositivos encontram conteúdo disponibilizados pelos servidores de mídia (DMS) e provêem a capacidade de reprodução e renderização de mídia. Exemplos: TVs, aparelhos de som, \emph{home theaters}, monitores sem fio e vídeo games;
		\item \emph{Digital Media Renderer} (DMR): 

		Esses dispositivos tocam conteúdos recebidos do controlador digital de mídia (DMC) que, por sua vez, irá encontrar conteúdo do servidor de mídia digital (DMS). Exemplos: TVs, receptores de áudio ou vídeo, projetores de vídeo e alto-falantes remotos para música;
		\item \emph{Digital Media Controller} (DMC): 

		Esses dispositivos encontram conteúdo em servidores de mídia digital (DMS) e o tocam em renderizadores de mídia digital (DMR). Exemplos: \emph{tablets}, câmeras digitais com Wi-Fi e PDAs;
		\item \emph{Digital Media Printer} (DMPr): 

		Esses dispositivos provêem serviços de impressão para a rede residencial DLNA. Geralmente, tocadores de mídia digital (DMP) e controladores de mídia digital (DMC) com a capacidade de impressão podem utilizar o DMPr para impressão. Exemplo: impressoras de foto e multifuncionais conectadas na rede DLNA.
	\end{itemize}
	\item \emph{Mobile Handheld Devices} (MHD)
	\begin{itemize}
		\item \emph{Mobile Digital Media Server} (M-DMS): 

		Esses dispositivos sem fio armazenam conteúdo e o tornam disponível para tocadores de mídia digital que tenham acesso à rede sem fio ou com fio (M-DMP), renderizadores de mídia digital (DMR) e impressoras de mídia digital (DMPr). Exemplo: telefones portáteis e tocadores portáteis de música;
		\item \emph{Mobile Digital Media Player} (M-DMP): 

		Esses dispositivos sem fio encontram e tocam conteúdo de um servidor de mídia digital (DMS) ou servidor móvel de mídia digital (M-DMS). Exemplo: telefones portáteis, \emph{tablets} projetados para visualização de conteúdo multimídia;
		\item \emph{Mobile Digital Media Uploader} (M-DMU): 

		Esses dispositivos sem fio enviam conteúdo para um servidor digital de mídia (DMS) ou para um servidor móvel de mídia digital (M-DMS). Exemplo: câmeras digitais, e telefones portáteis;
		\item \emph{Mobile Digital Media Downloader} (M-DMD): 

		Esses dispositivos sem fio encontram e armazenam conteúdo de um servidor de mídia digital (DMS) ou de um servidor móvel de mídia digital (M-DMS). Exemplo: tocadores de música e telefones portáteis;
		\item \emph{Mobile Digital Media Controller} (M-DMC): 

		Esses dispositivos encontram conteúdo em um servidor digital de mídia(DMS) ou em um servidor móvel de mídia digital (M-DMS) e o envia para renderizadores de mídia digital(DMR). Exemplo: telefones portáteis e PDAs.
	\end{itemize}
	\item \emph{Home Infrastructure Devices} (HID)
	\begin{itemize}
		\item \emph{Mobile Network Connectivity Function} (M-NCF): 

		Esses dispositivos provêem uma ligação entre a conectividade de rede de dispositivos móveis portáteis e conectividade de rede residencial;
		\item \emph{Media Interoperability Unit} (MIU): 

		Esses dispositivos provêem transformação de conteúdo entre formatos de mídia necessários para uma rede residencial ou para dispositivos móveis portáteis.
	\end{itemize}
\end{itemize}

\begin{table}
	\caption{Camadas padrões DLNA~\cite{dlnahdvideostreaming}.}
	\begin{center}
		\begin{tabular}{rp{5cm}p{5cm}}
		\hline
		\textbf{Camada} & \textbf{Função definida} & \textbf{Padrões}				\\
		\hline
		Transmissão Protegida & Como um conteúdo comercial está protegido em uma rede doméstica. & DTCP/IP \\
		\hline
		Formatos de Mídia & Como um conteúdo de mídia está codificado e identificado para interoperabilidade. & MPEG2, MPEG4, AVC/H.264, LPCM, MP3, AAC LC, JPEG, XHTML-Print- \\
		\hline
		Transporte de Mídia & Como um conteúdo de mídia é transferido. & HTTP, Quality of Service \\
		\hline
		Gerência de Mídia & Como um conteúdo de mídia é identificado, gerenciado e distribuído. & UPnP AV 1.0, UPnP Print Enhanced 1.0 \\
		\hline
		Descoberta e Controle & Como dispositivos se descobrem e se controlam um ao outro. & UPnP Device Architecture 1.0 \\
		\hline
		Redes IP & Como dispositivos com e sem fio fisicamente se conectam e se comunicam. & IPv4 Protocol Suite \\
		\hline
		Conectividade & Com qual rede os dispositivos se conectam e se comunicam& Wired: Ethernet 802.3, MoCAWireless: Wi-Fi 802.11, Wi-Fi Protected Setup \\
		\hline
		\end{tabular}
	\end{center}
	\label{tab:camadaspadroes_dlna}
\end{table}

\begin{comment}
\begin{table}
	\caption{Classes DLNA~\cite{}.}
	\begin{center}
		\begin{tabular}{llll}
		\hline
		\textbf{Classe} & \textbf{Categoria} & \textbf{Descrição} & \textbf{Exemplo}						\\
		\hline
		\hline
		Mobile Network Connectivity Function (M-NCF) & \multirow{2}{*}{Home Infrastructure Devices} &  & \\
		\hline
		Media Interoperability Unit (MIU) & &  & \\
		\hline
		\end{tabular}
	\end{center}
	\label{tab:classes_dlna}
\end{table}
\end{comment}

\begin{figure}[ht]
\center
\includegraphics[scale=0.5]{imagens/recursosDLNA}
\caption{Representação gráfica dos tipos existentes no DLNA.}
\label{fig:ddlspec}
\end{figure}

Considere como exemplo a Figura~\ref{fig:traditionalProccess}, em que uma pessoa deseja compartilhar um pequeno vídeo com seus amigos a partir de seu celular. A fim de exibir esse vídeo em sua TV widescreen na sua sala de estar, ela necessita, primeiramente, enviá-lo para seu próprio email. Em seguida, deve ligar seu PC, fazer o download do vídeo e salvá-lo em um \emph{pendrive} ou cartão de memória. Posteriormente, deve ligá-lo na TV ou em um receptor digital e, então, utilizar a interface do dispositivo para localizar o vídeo e exibi-lo~\cite{dlnahdvideostreaming}, tornando-se, dessa forma, um processo tedioso e demora que os consumidores raramente farão.

Além de interromper a conversa, esse tipo de processo de transferência de conteúdo exige tempo e atenção em excesso do usuário, mesmo quando ocorre sem problemas. Como resultado, a partilha informal de vídeo em múltiplas telas não é vista, muitas vezes, como uma opção razoável~\cite{dlnahdvideostreaming}.

\begin{figure}[ht]
	\center
	\includegraphics[scale=0.3]{imagens/dlna1}
	\caption{Processo de exibição de um vídeo, que está armazenado em um celular, em uma TV.}
	\label{fig:traditionalProccess}
\end{figure}

Este é o objetivo das organizações com compõem a DLNA: oferecer aos seus consumidores a partilha contínua e sem esforço de conteúdo digital, permitindo o envio de fotos ou vídeos para a TV em uma única etapa (Figura~\ref{fig:dlnaProccess}). O consumidor pode até mesmo congelar o vídeo usando um ``controle remoto'' do menu no telefone, bem como \emph{fast-forward} e reproduzir o vídeo~\cite{dlnahdvideostreaming}.

\begin{figure}[ht]
	\center
	\includegraphics[scale=0.3]{imagens/dlna2}
	\caption{Envio de conteúdo de digital para TV em uma única etapa.}
	\label{fig:dlnaProccess}
\end{figure}

Desta forma, o DLNA pode ser visto como um aglomerado de camadas e seus respectivos padrões abertos (Tabela~\ref{tab:camadaspadroes_dlna}) que definem como uma rede residencial interage em todos os seus níveis, ou seja, além de definir como os diferentes padrões irão interoperar e como os dados serão tratados em cada nível, ele também reduz o número de padrões que um dispositivo deve suportar.

\begin{comment}
\begin{itemize}
       \item Enviar: 

       Transferir vídeos ou imagens capturadas em uma câmera digital ou celular para um computador;
       \item Empurrar: 

       Exibir vídeos ou imagens capturadas em uma câmera digital ou celular diretamente em uma TV sem intermédio de um computador;
       \item Localizar e Reproduzir ou ``Reproduzir em...'': 

       Utiliza um celular para localizar uma música ou vídeo armazenados em um computador, unidade de disco externa ou um dispositivo \emph{Network-Attached Storage} (NAS) e transferi-lo, via \emph{stream} ou não, para reprodução;
       \item Puxar e imprimir ou ``Imprimir em...'': 

       Visualizar na TV uma foto armazenada em um servidor de mídia e imprimi-la utilizando uma impressora em rede.
\end{itemize}

\begin{table}
	\caption{Exemplos de casos de uso~\cite{dlnahdvideostreaming}.}
	\begin{center}
		\begin{tabular}{rl}
		\hline
		\textbf{Casos de Uso} & \textbf{Exemplo}																\\
		\hline
		Enviar & Transferir vídeos ou imagens capturadas em uma câmera digital ou celular para um computador.	\\
		\hline
		Empurrar & Exibir vídeos ou imagens capturadas em uma câmera digital ou celular diretamente em uma TV sem intermédio de um computador. \\
		\hline
		Localizar e Reproduzir ou ``Reproduzir em...'' & Utiliza um celular para localizar uma música ou vídeo armazenados em um computador, unidade de disco externa ou um dispositivo \emph{Network-Attached Storage} (NAS) e transferi-lo, via \emph{stream} ou não, para reprodução. \\
		\hline
		Puxar e imprimir ou ``Imprimir em...'' & Visualizar na TV uma foto armazenada em um servidor de mídia e imprimi-la utilizando uma impressora em rede. \\
		\hline
		\end{tabular}
	\end{center}
	\label{tab:casosdeuso_dlna}
\end{table}
\end{comment}

De volta ao exemplo apresentado no início deste capítulo, observa-se que as televisões citadas classificariam-se como dispositivos DMP e DMR.