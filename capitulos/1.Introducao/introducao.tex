\chapter{Introdução}

A computação ubíqua caracteriza-se por representar ambientes computacionais responsáveis por realizar determinadas tarefas predeterminadas de tal forma que certas premissas sejam obedecidas. É necessário que os serviços e dispositivos desse tipo de sistema trabalhem harmonicamente a fim de evitar, sempre que possível, toda e qualquer intervenção humana. A esta característica dá-se o nome de invisibilidade ~\cite{gomes2007, weiser1993, weiser1999}. Faz-se necessário, inclusive, que tais sistemas sejam pró-ativos ~\cite{gomes2007, buzeto2010} e consigam determinar, com a ajuda de informações de contexto previamente coletadas, quais as melhores decisões a serem tomadas em determinados instantes. Deve-se considerar, ainda, a mobilidade ~\cite{gomes2007, buzeto2010, weiser1999} dos dispositivos presentes e regidos dentro do ambiente em questão, a saber, o smartspace. Em um ambiente inteligente se faz necessária a adaptabilidade de serviços de forma transparente para os clientes, ou seja, caso um serviço não esteja mais sendo provido por determinado dispositivo, o smartspace deve identificar esta falha e procurar um outro servidor ~\cite{gomes2007, passarinho2008, paranhos2009}.

O objetivo deste trabalho é definir uma classificação de recursos e a partir desta classificação, prover uma hierarquia de recursos extensível para a arquitetura DSOA do middleware de Computação Ubíqua uOS. Essa hierarquia irá facilitar a escolha de determinado serviço provido por algum recurso por parte do uOS ou de alguma aplicação cliente. Recurso, segundo a definição na proposta da DSOA, é um grupo de funcionalidades logicamente relacionadas que deverão ser acessíveis por meio de interfaces pré-definidas ~\cite{buzeto2010}. Por exemplo: uma tela sensível ao toque é um recurso e possui as funcionalidades de prover uma imagem e de dispositivo apontador.

\section{Organização do Trabalho}
Este trabalho está organizado da seguinte maneira.


