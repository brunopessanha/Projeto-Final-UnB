\chapter{Implementação e Testes}

Como visto no capítulo ~\ref{cap:proposta}, a classificação de recursos, além de facilitar o desenvolvimento de novos \emph{drivers}, possibilita a escolha de recursos equivalentes, adiciona flexibilidade ao \emph{smart space} etc. Para demonstrar essas vantagens, será mostrado na seção X um estudo de caso sobre a classificaçào de recursos na DSOA.

\section{O Protótipo}
Com o objetivo de se utilizar da flexibilidade adicionada pela classificação de recursos foram construídas duas aplicações para \emph{smart phones} Android. A primeira aplicação implementa um recurso equivalente ao recurso \emph{Pointer}, a interface \emph{Clickable} mostrada na listagem Y. A segunda aplicação implementa um recurso equivalente ao recurso \emph{Clickable} e \emph{Scrollable}, a interface \emph{MouseDriver} mostrada na listagem Z.