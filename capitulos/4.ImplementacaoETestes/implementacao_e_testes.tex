\chapter{Implementação e Testes}

Como visto no capítulo~\ref{cap:proposta}, a classificação de recursos, além de facilitar o desenvolvimento de novos \emph{drivers}, possibilita a escolha de recursos equivalentes e adiciona flexibilidade ao \emph{smart space}. Para demonstrar essas vantagens, será mostrado na seção X um estudo de caso sobre a classificaçào de recursos na DSOA.

\section{O Protótipo}
A fim de se demonstrar a flexibilidade adicionada pela classificação de recursos, foram construídas duas aplicações para \emph{smart phones} Android. A primeira implementa um recurso do tipo \emph{Clickable}, que por sua vez é equivalente ao tipo básico \emph{Pointer}. A segunda implementa um recurso do tipo \emph{MouseDriver}, que por sua vez é equivalente aos recursos \emph{Clickable} e \emph{Scrollable}.