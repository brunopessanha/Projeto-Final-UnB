\chapter{Implementação e Testes}

Como visto no capítulo~\ref{cap:proposta}, a classificação de recursos, além de facilitar o desenvolvimento de novos \emph{drivers}, possibilita a escolha de recursos equivalentes e adiciona flexibilidade ao \emph{smart space}. Para demonstrar essas vantagens, será mostrado na seção~\ref{sec:osPrototipos} um estudo de caso sobre a classificaçào de recursos na DSOA.

\section{Hydra}
\label{sec:Hydra}

Desenvolvida com a capacidade de sondar o \emph{smart space} em busca de recursos e serviços, a Hydra é uma aplicação capaz de oferecer acesso remoto, distribuido e desagregado dos recursos de entrada e saída dos dispositivos. Sua função é permitir que determinado dispositivo possa disponibilizar seus recursos para utilização por parte de outros aparelhos presentes no ambiente. Seu processo de comunicação com os dispositivos é intermediado pelo uOs, que abstrai a complexidade do processo. Dessa forma, faz-se necessária a utilização de \emph{drivers} que mapeiem os recursos existentes no ambiente de tal forma a manter a compatibilidade com a DSOA.

\begin{comment}
A Hydra consiste em uma aplicação construída com o objetivo de explorar a forma como o ambiente é decomposto em recursos e serviços para possibilitar uma forma desagregada de acesso a recursos de entrada e saída dos dispositivos. Seu objetivo é permitir que um determinado dispositivo tenha seus recursos de entrada e saída redirecionados para outros dispositivos, repassando a estes o controle de sua operação.

A Hydra, trabalhando sobre o middleware uOS, percebe o ambiente como um conjunto
de recursos presentes. A comunicação entre a Hydra os recursos ocorre com intermédio do
uOS, abstraindo tanto da Hydra quando dos drivers o outro ponto de comunicação. Para
os drivers, não importa para quem estão sendo prestados os serviços, e cabe à aplicação
a escolha entre os recursos do ambiente.

Por trabalhar com os conceitos da DSOA (Seção 3.1), a Hydra vê os recursos do
ambiente de forma transparente, e pode fazer uso de qualquer recurso compatível com as
interfaces por ele esperadas e permite que outros dispositivos utilizem os mesmos recursos
do ambiente.

A aplicação Hydra se propõe a reconhecer determinados drivers e redirecionar seus
serviços corretamente.
\end{comment}

\section{Os protótipos}
\label{sec:osPrototipos}

A fim de se demonstrar a flexibilidade adicionada pela classificação de recursos, foram construídas duas aplicações para \emph{smart phones} Android que utilizam da tela sensível ao toque para implementar um recurso de \emph{mouse} integradas com a \emph{Hydra}. Uma das aplicações implementa um recurso do tipo \emph{Clickable}, equivalente ao tipo básico \emph{Pointer}. A outra implementa um recurso equivalente aos recursos \emph{Clickable} e \emph{Scrollable}, ambos equivalentes ao tipo básico \emph{Pointer}. Dessa forma, obteve-se dois recursos de \emph{Mouse}, sendo que um possui apenas serviços de \emph{click} além da movimentação do ponteiro do mouse e o outro possui ambos os serviços e inclui ainda o serviço de \emph{scroll}. 

O recurso \emph{Clickable} possui a seguinte interface:

\begin{itemize}
	
	\item Nome do Recurso: \emph{Clickable};

	\item Serviços:
		
		\begin{itemize}
			
			\item ``\emph{registerListener}'': Serviço síncrono que recebe como parâmetro a ``eventKey'' do serviço assíncrono ao qual deseja ser notificado.

			\item ``\emph{unregisterListener}'': Serviço síncrono que recebe como parâmetro a ``eventKey'' do serviço assíncrono ao qual não deseja mais ser notificado.

			\item ``\emph{move}'': Serviço assíncrono herdado do recurso equivalente \emph{Pointer}. Possui dois parâmetros definidos da seguinte forma:

				\begin{itemize}
					\item ``\emph{axisX}'': parâmetro inteiro obrigatório que informa em quantos \emph{pixels} no eixo X a posição do ponteiro do \emph{mouse} foi alterada.

					\item ``\emph{axisY}'': parâmetro inteiro obrigatório que informa em quantos \emph{pixels} no eixo Y a posição do ponteiro do \emph{mouse} foi alterada.
				\end{itemize}
			
			\item ``\emph{buttonPressed}'': Serviço assíncrono que informa que um botão foi pressionado. Possui um parâmetro definido da seguinte forma:

				\begin{itemize}
					\item ``\emph{button}'': parâmetro inteiro obrigatório que informa o código do botão pressionado.
				\end{itemize}
			
			\item ``\emph{buttonReleased}'': Serviço assíncrono que informa que um botão que fora pressionado foi liberado. Possui um parâmetro definido da seguinte forma:

				\begin{itemize}
					\item ``\emph{button}'': parâmetro inteiro obrigatório que informa o código do botão que foi liberado após ter sido pressionado.
				\end{itemize}

		\end{itemize}
\end{itemize}

Já o recurso \emph{Scrollable} possui a seguinte interface:

\begin{itemize}
	
	\item Nome do Recurso: \emph{Scrollable};

	\item Serviços:
		
		\begin{itemize}
			
			\item ``\emph{registerListener}'': Serviço síncrono que recebe como parâmetro a ``eventKey'' do serviço assíncrono ao qual deseja ser notificado.

			\item ``\emph{unregisterListener}'': Serviço síncrono que recebe como parâmetro a ``eventKey'' do serviço assíncrono ao qual não deseja mais ser notificado.

			\item ``\emph{move}'': Serviço assíncrono herdado do recurso equivalente \emph{Pointer} Possui dois parâmetros definidos da seguinte forma:

				\begin{itemize}
					\item ``\emph{axisX}'': parâmetro inteiro obrigatório que informa em quantos \emph{pixels} no eixo X a posição do ponteiro do \emph{mouse} foi alterada.

					\item ``\emph{axisY}'': parâmetro inteiro obrigatório que informa em quantos \emph{pixels} no eixo Y a posição do ponteiro do \emph{mouse} foi alterada.
				\end{itemize}
			
			\item ``\emph{scroll}'': Serviço assíncrono que informa que um botão foi pressionado. Possui um parâmetro definido da seguinte forma:

				\begin{itemize}
					\item ``\emph{distance}'': parâmetro inteiro obrigatório que informa a quantidade de unidades que uma barra de rolagem foi movimentada.
				\end{itemize}

		\end{itemize}
\end{itemize}

Para implementação dos dois \emph{drivers} foi utilizada a versão JSE do \emph{uOS}, a listagem~\ref{notificacoesDeEventos} mostra como é feito o envio de notificação de eventos de \emph{move}, \emph{click} e \emph{scroll}.

\lstset{caption={Notificações de eventos},label=notificacoesDeEventos, language=Java}
\begin{lstlisting}[frame=single]
	@Override
	public void move(int axisX, int axisY) {
		Notify notify = new Notify(MOVE_EVENT)
			.addParameter(AXIS_X, String.valueOf(axisX))
			.addParameter(AXIS_Y, String.valueOf(axisY));
		
        sendEvent(notify);
	}
    
	@Override
	public void scroll(int distance) {
		Notify notify = new Notify(SCROLL_EVENT)
			.addParameter(DISTANCE, String.valueOf(distance));
	
	    sendEvent(notify);	
	}

	@Override
	public void buttonPressed(int button) {
		raiseButtonEvent(button, BUTTON_PRESSED_EVENT);
	}

	@Override
	public void buttonReleased(int button) {
		raiseButtonEvent(button, BUTTON_RELEASED_EVENT);
	}

	private void raiseButtonEvent(Integer button, String event) {
		
		Notify notify = new Notify(event)
			.addParameter(Mouse.BUTTON, button.toString());

        sendEvent(notify);
	}

	private void sendEvent(Notify notify) {
		
		notify.setDriver(Mouse.DRIVER_NAME);
	    notify.setInstanceId(instanceId);
		
		for (int i = 0 ; i < listenerDevices.size(); i++){
	        UpNetworkInterface uni = 
	        		(UpNetworkInterface) listenerDevices.get(i);
	        UpDevice device = new UpDevice("Anonymous");
	        device.addNetworkInterface(uni.getNetworkAddress(), 
	        		uni.getNetType());
	        try {
	            this.gateway.sendEventNotify(notify, device);
	        } catch (NotifyException e) {
	        	Log.e("MOUSE DRIVER", e.getMessage());
	        }
	    }
	}
\end{lstlisting}

Do lado da aplicação \emph{Hydra}, alteramos a classe \emph{MouseEventsListener} para utilizar as novas interfaces. A listagem~\ref{registroNotificacoes} mostra como é feito cada registro de notificação a um evento. A listagem mostra ainda que a \emph{Hydra} se registra pra serviços de \emph{click}, \emph{move}, \emph{scroll}, mas a listagem~\ref{listDrivers} mostra que a aplicação procura apenas por recursos \emph{Clickable}. Ou seja, para cumprir sua função e funcionar de cliente para um recurso de \emph{mouse}, a \emph{Hydra} precisa apenas dos serviços de \emph{click} e \emph{move} e por isso procura apenas pelo recurso que provê esses serviços, embora também tente subscrever às notificaçòes de eventos de \emph{scroll}, ela funciona normalmente sem este serviço.
 

\lstset{caption={Registro de Notificações de eventos},label=registroNotificacoes, language=Java}
\begin{lstlisting}[frame=single]
	public void registerForDriver(DriverData driverData)
		throws NotifyException {
		gateway.registerForEvent(this, driverData.getDevice(),
				driverData.getDriver().getName(),
				driverData.getInstanceID(), 
				Pointer.MOVE_EVENT);
		gateway.registerForEvent(this, driverData.getDevice(),
				driverData.getDriver().getName(),
				driverData.getInstanceID(), 
				Clickable.BUTTON_PRESSED_EVENT);
		gateway.registerForEvent(this, driverData.getDevice(),
				driverData.getDriver().getName(),
				driverData.getInstanceID(), 
				Clickable.BUTTON_RELEASED_EVENT);
		gateway.registerForEvent(this, driverData.getDevice(),
				driverData.getDriver().getName(),
				driverData.getInstanceID(), 
				Scrollable.SCROLL_EVENT);
	}
\end{lstlisting}

\lstset{caption={Busca por drivers \emph{Clickable}},label=listDrivers, language=Java}
\begin{lstlisting}[frame=single]
	public List<DriverData> getMouseDriversList() {
		return gateway.getDriverManager().listDrivers(Clickable.DRIVER_NAME, null);
	}
\end{lstlisting}
