\chapter{Implementação}

A classificação de recursos (capítulo~\ref{cap:proposta}), além de facilitar o desenvolvimento de novos \emph{drivers}, possibilita a escolha de recursos equivalentes adicionando flexibilidade ao \emph{smart space}. Ao se modificar a DSOA no que se refere a forma que os recursos são encontrados e identificados no ambiente, as partes que implementam esta arquitetura também devem ser ajustadas para se manterem compatíveis. Desta forma, tanto o conjunto de protocolos uP como o middleware uOS foram adequados para comportar a classificação de recursos apresentada.

Os recursos no \emph{uOS} são os meios para se utilizar os serviços disponíveis. Inicialmente, ao se necessitar de um determinado serviço era necessário explicitar exatamente qual o recurso desejado e, caso este estivesse disponível no ambiente, poderia ser cedido para utilização por um dispositivo. Caso contrário, a aplicação requisitante não poderia utilizar o serviço pretendido, mesmo que houvesse algum outro recurso capaz de provê-lo. Entretanto, com a adição da classificação de recursos, tornou-se possível solicitar por algum recurso existente no ambiente e, caso aquele recurso esteja indisponível, outros recursos considerados equivalentes ao primeiro poderão oferecer, de forma transparente ao usuário, o serviço necessário.

\section{\emph{uOS - Ubiquitous OS}}

O \emph{uOS} é um \emph{middleware} cujo propósito é fornecer uma infra-estrutura de software utilizando conceitos de computação ubíqua com foco na adaptabilidade de serviços. Utilizando a arquiteura \emph{DSOA}, o \emph{uOS} é responsável por dar suporte para o desenvolvimento de \emph{drivers} e aplicações. Além disso, o \emph{uOS} utiliza o conjunto de protocolos \emph{uP} para realizar sua comunicação.

\begin{figure}[ht]
	\center
	\includegraphics[scale=0.4]{imagens/ecossistemaUbiquitos}
	\caption{Ecossistema do \emph{uOS}.}
	\label{fig:ecossistemaUbiquitos}
\end{figure}

A figura~\ref{fig:ecossistemaUbiquitos} mostra o ecossistema do \emph{middleware}, onde os \emph{plugins}, abstrações para a rede de comunicação, se destacam como um componente dentro do \emph{uOS}. O \emph{uOS}, por sua vez, utiliza os \emph{plugins} para interfacear a comunicação entre aplicações, por meio de \emph{drivers}. No \emph{uOS} destacam-se três camadas: Rede, Conectividade e Adaptabilidade. A camada de Adaptabilidade é responsável pela coordenação de interaçòes feitas por meio do \emph{middleware}. É nessa camada que se encontram o \emph{Device Manager}, responsável pela gerenciamento de dispositivos presentes no ambiente, ou seja, o \emph{Device Manager} é notificado por um radar sobre a entrada ou saída de dispositivos e o \emph{Driver Manager}, responsável por determinar o \emph{driver} de recurso que irá tratar uma chamada recebida pela \emph{Message Engine}. A Figura TODO mostra a arquitetura interna do \emph{uOS}.

No \emph{Driver Manager} foi adicionada uma estrutura em forma de floresta, onde a raiz de cada árvore representa um recurso novo, ou seja, que não é equivalente a nenhum outro recurso. Essa estrutura é inicializada juntamente com o \emph{Driver Manager} com cada tipo básico definidos na Seção~\ref{sec:tiposBasicos} compondo uma nova raiz da estrutura de árvores, definida na Subseção~\ref{subsec:arvoreDeRecursos}. O \emph{Driver Manager} é responsável por manter as relações de equivalência dos recursos, mantendo a consistência de serviços e interfaces e impedindo que ocorra uma referência circular entre recursos equivalentes. Quando um novo recurso é registrado no \emph{middleware}, o \emph{Driver Manager} realiza validações em seu \emph{driver} segundo as relações de quivalência e em seguida o \emph{driver} é adicionado na árvore de equivalência correta. Além de manter essa estrutura, este módulo é responsável por popular a Ontologia do \emph{uOS}. O novo \emph{driver} é adicionado à uma Ontologia por meio da \emph{uOS-Context API}, um conjunto de interfaces que permite a manipulação estrutural e semântica da Ontologia de Contexto do \emph{uOS}~\cite{ozakisbcup2011}. A Figura~\ref{fig:ontologiaUOS} mostra a parte da ontologia que contém as classes dos recursos \emph{Pointer} e \emph{Keyboard}.

\begin{figure}[ht]
	\center
	\includegraphics[scale=0.55]{imagens/ontologia}
	\caption{Exemplo de recursos da Ontologia do \emph{uOS}.}
	\label{fig:ontologiaUOS}
\end{figure}

O processo de registro do dispositivo no \emph{middleware} foi modificado. Quando um novo recurso a ser registrado é equivalente a pelo menos um recurso desconhecido, o \emph{Driver Manager} não irá registrá-lo em sua árvore de equivalência enquanto não conhecê-lo. Neste caso, o módulo \emph{Device Manager} é responsável por descobrir esse(s) recurso(s). Após conhecer a interface dos recursos, este módulo irá então registrar cada recurso desconhecido para então registrar o novo recurso na árvore de equivalência.

Para que o \emph{Device Manager} possa descobrir a interface de um determinado recurso desconhecido ele deve procurar o dispositivo que possui este recurso. Dessa forma, o \emph{Device Driver} foi acrescido do serviço \emph{``tellEquivalentDrivers''} que informa a interface de todos os recursos necessários para que o registro no \emph{uOS} possa ocorrer.
\section{Estudo de caso}
\label{sec:estudoDeCaso}

A fim de se demonstrar a corretude da proposta, construiu-se duas aplicações para \emph{smart phones} Android (figura~\ref{fig:printscreen_android}) que utilizam a tela sensível ao toque para implementar um recurso de \emph{mouse}. A primeira implementa um recurso do tipo \emph{Clickable}, equivalente ao tipo básico \emph{Pointer}, enquanto que a segunda implementa um recurso equivalente aos recursos \emph{Clickable} e \emph{Scrollable}, sendo que esta última também é equivalente ao tipo básico \emph{Pointer}. Portanto, obteve-se dois recursos de \emph{Mouse}, de forma que um possui apenas o serviço de \emph{click}, além da movimentação do ponteiro do mouse e o outro possui ambos os serviços e inclui ainda o serviço de \emph{scroll}.

Ambas a aplicações são integradas com a \emph{Hydra}, que é uma aplicação desenvolvida com a capacidade de sondar o \emph{smart space} em busca de recursos e serviços e redireciona-los de forma a oferecer acesso remoto, distribuido e desagregado dos recursos de entrada e saída dos dispositivos. Sua função é permitir que determinado dispositivo disponibilize seus recursos para utilização por parte de outros aparelhos presentes no ambiente, repassando a estes o controle da operação do recurso requisitado. Seu processo de comunicação com os dispositivos é intermediado pelo \emph{uOS}, que abstrai a complexidade do processo. Dessa forma, faz-se necessária a utilização de \emph{drivers} que mapeiem os recursos existentes no ambiente de tal forma a manter a compatibilidade com a DSOA~\cite{lucas2011}. Seguem abaixo as interfaces dos \emph{drivers} utilizados.

Recurso \emph{Clickable}:

\begin{itemize}
	
	\item Nome do Recurso: \emph{Clickable};

	\item Serviços:
		
		\begin{itemize}
			
			\item ``\emph{registerListener}'': Serviço síncrono que recebe como parâmetro a ``eventKey'' do serviço assíncrono ao qual deseja ser notificado.

			\item ``\emph{unregisterListener}'': Serviço síncrono que recebe como parâmetro a ``eventKey'' do serviço assíncrono ao qual não deseja mais ser notificado.

			\item ``\emph{move}'': Serviço assíncrono herdado do recurso equivalente \emph{Pointer}. Possui dois parâmetros definidos da seguinte forma:

				\begin{itemize}
					\item ``\emph{axisX}'': parâmetro inteiro obrigatório que informa em quantos \emph{pixels} no eixo X a posição do ponteiro do \emph{mouse} foi alterada.

					\item ``\emph{axisY}'': parâmetro inteiro obrigatório que informa em quantos \emph{pixels} no eixo Y a posição do ponteiro do \emph{mouse} foi alterada.
				\end{itemize}
			
			\item ``\emph{buttonPressed}'': Serviço assíncrono que informa que um botão foi pressionado. Possui um parâmetro definido da seguinte forma:

				\begin{itemize}
					\item ``\emph{button}'': parâmetro inteiro obrigatório que informa o código do botão pressionado.
				\end{itemize}
			
			\item ``\emph{buttonReleased}'': Serviço assíncrono que informa que um botão que fora pressionado foi liberado. Possui um parâmetro definido da seguinte forma:

				\begin{itemize}
					\item ``\emph{button}'': parâmetro inteiro obrigatório que informa o código do botão que foi liberado após ter sido pressionado.
				\end{itemize}

		\end{itemize}
\end{itemize}

Recurso \emph{Scrollable}:

\begin{itemize}
	
	\item Nome do Recurso: \emph{Scrollable};

	\item Serviços:
		
		\begin{itemize}
			
			\item ``\emph{registerListener}'': Serviço síncrono que recebe como parâmetro a ``eventKey'' do serviço assíncrono ao qual deseja ser notificado.

			\item ``\emph{unregisterListener}'': Serviço síncrono que recebe como parâmetro a ``eventKey'' do serviço assíncrono ao qual não deseja mais ser notificado.

			\item ``\emph{move}'': Serviço assíncrono herdado do recurso equivalente \emph{Pointer} Possui dois parâmetros definidos da seguinte forma:

				\begin{itemize}
					\item ``\emph{axisX}'': parâmetro inteiro obrigatório que informa em quantos \emph{pixels} no eixo X a posição do ponteiro do \emph{mouse} foi alterada.

					\item ``\emph{axisY}'': parâmetro inteiro obrigatório que informa em quantos \emph{pixels} no eixo Y a posição do ponteiro do \emph{mouse} foi alterada.
				\end{itemize}
			
			\item ``\emph{scroll}'': Serviço assíncrono que informa que um botão foi pressionado. Possui um parâmetro definido da seguinte forma:

				\begin{itemize}
					\item ``\emph{distance}'': parâmetro inteiro obrigatório que informa a quantidade de unidades que uma barra de rolagem foi movimentada.
				\end{itemize}

		\end{itemize}
\end{itemize}

\begin{comment}
Para implementação dos dois \emph{drivers} foi utilizada a versão JSE do \emph{uOS}, a listagem~\ref{notificacoesDeEventos} mostra como é feito o envio de notificação de eventos de \emph{move}, \emph{click} e \emph{scroll}.

\lstset{caption={Notificações de eventos},label=notificacoesDeEventos, language=Java}
\begin{lstlisting}[frame=single]
	@Override
	public void move(int axisX, int axisY) {
		Notify notify = new Notify(MOVE_EVENT)
			.addParameter(AXIS_X, String.valueOf(axisX))
			.addParameter(AXIS_Y, String.valueOf(axisY));
		
        sendEvent(notify);
	}
    
	@Override
	public void scroll(int distance) {
		Notify notify = new Notify(SCROLL_EVENT)
			.addParameter(DISTANCE, String.valueOf(distance));
	
	    sendEvent(notify);	
	}

	@Override
	public void buttonPressed(int button) {
		raiseButtonEvent(button, BUTTON_PRESSED_EVENT);
	}

	@Override
	public void buttonReleased(int button) {
		raiseButtonEvent(button, BUTTON_RELEASED_EVENT);
	}

	private void raiseButtonEvent(Integer button, String event) {
		
		Notify notify = new Notify(event)
			.addParameter(Mouse.BUTTON, button.toString());

        sendEvent(notify);
	}

	private void sendEvent(Notify notify) {
		
		notify.setDriver(Mouse.DRIVER_NAME);
	    notify.setInstanceId(instanceId);
		
		for (int i = 0 ; i < listenerDevices.size(); i++){
	        UpNetworkInterface uni = 
	        		(UpNetworkInterface) listenerDevices.get(i);
	        UpDevice device = new UpDevice("Anonymous");
	        device.addNetworkInterface(uni.getNetworkAddress(), 
	        		uni.getNetType());
	        try {
	            this.gateway.sendEventNotify(notify, device);
	        } catch (NotifyException e) {
	        	Log.e("MOUSE DRIVER", e.getMessage());
	        }
	    }
	}
\end{lstlisting}
\end{comment}

Do lado da aplicação \emph{Hydra}, alterou-se a classe \emph{MouseEventsListener} para utilizar as novas interfaces. Tal classe se registra pra os serviços de \emph{click}, \emph{move} e \emph{scroll}, mas busca apenas por recursos \emph{Clickable}. Ou seja, para cumprir sua função e assumir papel de cliente para um recurso de \emph{mouse}, a \emph{Hydra} precisa apenas dos serviços de \emph{click} e \emph{move} e por isso procura somente pelo recurso que provê esses serviços, embora também tente utilizar as notificaçòes de eventos de \emph{scroll}, ela funciona normalmente sem este serviço.
 
\begin{comment}
\lstset{caption={Registro de Notificações de eventos},label=registroNotificacoes, language=Java}
\begin{lstlisting}[frame=single]
	public void registerForDriver(DriverData driverData)
		throws NotifyException {
		gateway.registerForEvent(this, driverData.getDevice(),
				driverData.getDriver().getName(),
				driverData.getInstanceID(), 
				Pointer.MOVE_EVENT);
		gateway.registerForEvent(this, driverData.getDevice(),
				driverData.getDriver().getName(),
				driverData.getInstanceID(), 
				Clickable.BUTTON_PRESSED_EVENT);
		gateway.registerForEvent(this, driverData.getDevice(),
				driverData.getDriver().getName(),
				driverData.getInstanceID(), 
				Clickable.BUTTON_RELEASED_EVENT);
		gateway.registerForEvent(this, driverData.getDevice(),
				driverData.getDriver().getName(),
				driverData.getInstanceID(), 
				Scrollable.SCROLL_EVENT);
	}
\end{lstlisting}

\lstset{caption={Busca por drivers \emph{Clickable}},label=listDrivers, language=Java}
\begin{lstlisting}[frame=single]
	public List<DriverData> getMouseDriversList() {
		return gateway.getDriverManager().
			listDrivers(Clickable.DRIVER_NAME, null);
	}
\end{lstlisting}
\end{comment}

\subsection{Os protótipos}

A Figura~\ref{fig:printscreen_symbian} mostra o protótipo desenvolvido para Symbian sendo executado em um celular Nokia N95. No centro da tela é mostrado o fator de movimentação para cada movimento no \emph{joystick} do celular, ou seja, quantas unidades do movimento real do cursor do \emph{mouse} representa um movimento no \emph{joystick} do celular. Esse fator de movimentação foi criado, pois a tela do celular é muito menor que outros dispostivos de saída de vídeo. Uma grande limitação do dispositivo utilizado é que seu \emph{joystick} se movimenta apenas em uma direção a cada vez que o botão é pressionado, sempre na vertical ou na horizontal, não permetindo que \emph{mouse} se movimente em diferentes ângulos.

O fator de movimentação não foi implementado na aplicação para Android, pois foi utilizado um dispositivo com tela sensível ao toque que torna mais natural a movimentação do cursor do \emph{mouse} assemelhando-se com o uso de um \emph{touchpad} de \emph{notebook}. Ao se utilizar a tela sensível ao toque obteve-se uma grande vantagem em relação à implementação anterior, pois é possível movimentar o cursor do \emph{mouse} em qualquer direção. A figura~\ref{fig:printscreen_android} mostra a tela do protótipo utilizado.

\begin{figure}[h]
	\centering
	\begin{minipage}[t]{0.30\linewidth}
		\includegraphics[width=\linewidth]{imagens/printscreen_n95}
		\caption{Printscreen do protótipo desenvolvido para Symbian.}
		\label{fig:printscreen_symbian}
	\end{minipage}
	\hfill
	\begin{minipage}[t]{0.30\linewidth}
		\includegraphics[width=\linewidth]{imagens/printscreen_android}
		\caption{Printscreen do protótipo desenvolvido para Android. O círculo verde representa a posição do dedo no momento da captura da imagem.}
		\label{fig:printscreen_android}
	\end{minipage}
\end{figure}
