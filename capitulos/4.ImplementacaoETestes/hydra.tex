\section{Hydra}
\label{sec:Hydra}

Desenvolvida com a capacidade de sondar o \emph{smart space} em busca de recursos e serviços, a Hydra é uma aplicação capaz de oferecer acesso remoto, distribuido e desagregado dos recursos de entrada e saída dos dispositivos. Sua função é permitir que determinado dispositivo possa disponibilizar seus recursos para utilização por parte de outros aparelhos presentes no ambiente. Seu processo de comunicação com os dispositivos é intermediado pelo uOs, que abstrai a complexidade do processo. Dessa forma, faz-se necessária a utilização de \emph{drivers} que mapeiem os recursos existentes no ambiente de tal forma a manter a compatibilidade com a DSOA~\cite{buzeto2010, lucas2011}.

\begin{comment}
A Hydra consiste em uma aplicação construída com o objetivo de explorar a forma como o ambiente é decomposto em recursos e serviços para possibilitar uma forma desagregada de acesso a recursos de entrada e saída dos dispositivos. Seu objetivo é permitir que um determinado dispositivo tenha seus recursos de entrada e saída redirecionados para outros dispositivos, repassando a estes o controle de sua operação.

A Hydra, trabalhando sobre o middleware uOS, percebe o ambiente como um conjunto
de recursos presentes. A comunicação entre a Hydra os recursos ocorre com intermédio do
uOS, abstraindo tanto da Hydra quando dos drivers o outro ponto de comunicação. Para
os drivers, não importa para quem estão sendo prestados os serviços, e cabe à aplicação
a escolha entre os recursos do ambiente.

Por trabalhar com os conceitos da DSOA (Seção 3.1), a Hydra vê os recursos do
ambiente de forma transparente, e pode fazer uso de qualquer recurso compatível com as
interfaces por ele esperadas e permite que outros dispositivos utilizem os mesmos recursos
do ambiente.

A aplicação Hydra se propõe a reconhecer determinados drivers e redirecionar seus
serviços corretamente.
\end{comment}