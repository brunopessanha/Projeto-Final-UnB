\chapter{Introdução}

A computação ubíqua caracteriza-se por representar ambientes computacionais responsáveis por realizar determinadas tarefas predeterminadas de tal forma que certas premissas sejam obedecidas. É necessário que os serviços e dispositivos desse tipo de sistema trabalhem harmonicamente a fim de evitar, sempre que possível, toda e qualquer intervenção humana. A esta característica dá-se o nome de invisibilidade~\cite{gomes2007}~\cite{buzeto2010}~\cite{weiser1993}~\cite{weiser1999}. Faz-se necessário, inclusive, que tais sistemas sejam pró-ativos~\cite{gomes2007}~\cite{buzeto2010} e consigam determinar, com a ajuda de informações de contexto previamente coletadas, quais as melhores decisões a serem tomadas em determinados instantes. Deve-se considerar, ainda, a mobilidade~\cite{gomes2007}~\cite{buzeto2010}~\cite{weiser1999} dos dispositivos presentes e regidos dentro do ambiente em questão, a saber, o smartspace. Em um ambiente inteligente se faz necessária a adaptabilidade de serviços de forma transparente para os clientes, ou seja, caso um serviço não esteja mais sendo provido por determinado dispositivo, o smartspace deve identificar esta falha e procurar um outro servidor~\cite{gomes2007}~\cite{passarinho2008}~\cite{paranhos2009}.


