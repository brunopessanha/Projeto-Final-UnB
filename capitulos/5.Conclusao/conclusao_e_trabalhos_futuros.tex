\chapter{Conclusão e trabalhos futuros}

A quantidade de dispositivos inteligentes disponíveis para consumo têm aumentado consideravelmente com o passar dos anos. Inúmeros aparelhos têm se tornado essenciais no dia-a-dia das pessoas, e a tendência é que essa dependência aumente gradativamente. Diminutos eletrônicos conectados à internet têm reduzido as distâncias e possibilitado novos tipos de interações nunca antes imaginadas. Atualmente, utilizando um único \emph{smartphone}, é possivel acessar o seu \emph{site} preferido, consultar as ações da bolsa de valores e a previsão do tempo, mudar o canal da sua televisão, ligar/desligar a luz do seu quarto, utilizar o GPS, dentre várias outras tarefas.
É nesse ambiente cada vez mais ubíquo que a classificação de recursos se insere: ambientes com dispositivos heterogêneos (móveis ou não) interagindo entre si, trocando informações ou utilizando capacidades características de cada aparelho.

\section{Contribuições}
\section{Trabalhos futuros}


\begin{enumerate}
	\item Possibilidade de criação de mecanismo de failover que possibilita que a adaptabilidade de serviços seja transparente ao usuário. Atualmente quando um recurso se torna indisponível, o usuário deve selecionar qual seria um outro recurso equivalente que pudesse continuar provando aquele serviço. 
	\item Inclusão de inteligência artifical para escolha automática de melhor recurso disponível.
		- Usando dados de QOS, confiança etc para a escolha.
	\item Utilizar a ontologia em vez da árvore de equivalência para determinar a equivalência entre recursos.
	\item Hydra redirecionar consulta feita por mouse utilizando a ontologia
\end{enumerate}