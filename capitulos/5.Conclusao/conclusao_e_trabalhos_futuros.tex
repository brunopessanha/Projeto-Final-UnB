\chapter{Conclusão e trabalhos futuros}

A quantidade de dispositivos inteligentes disponíveis para consumo têm aumentado consideravelmente com o passar dos anos. Inúmeros aparelhos têm se tornado essenciais no dia-a-dia das pessoas, e a tendência é que essa dependência aumente gradativamente. Diminutos eletrônicos conectados à internet têm reduzido as distâncias e possibilitado novos tipos de interações nunca antes imaginadas. Atualmente, utilizando um único \emph{smartphone}, é possivel acessar o seu \emph{site} preferido, consultar as ações da bolsa de valores e a previsão do tempo, mudar o canal da sua televisão, ligar a luz do seu quarto, obter rotas urbanas, dentre várias outras tarefas.
\begin{comment}
É nesse ambiente cada vez mais ubíquo que a classificação de recursos se insere: ambientes com dispositivos heterogêneos (móveis ou não) interagindo entre si, trocando informações ou utilizando capacidades características de cada aparelho.
\end{comment}

Com recursos cada vez mais capazes e variados, faz-se necessária uma padronização dos mesmos para se atingir o objetivo da ubicomp. É nessa tarefa que a DSOA se concentra, porém apenas com a classificação de recursos este acesso será flexível para abranger essa heterogeneidade de opções.

Como as principais contribuições deste trabalho, pode-se listar:

\begin{itemize}
	\item Possibilidade do usuário saber quais são os recursos equivalentes disponíveis no ambiente, coloborando para melhorar a adaptabilidade de serviços do \emph{uOS}. Antes deste trabalho os recursos não tinham relação entre sí, dificultando a escolha por parte do usuário e impossibilitando que o \emph{middleware} pudesse tomar esta decisão sem consultar o usuário;
	\item Definição das interfaces de recursos básicos que auxiliam o desenvolvimento de novos drivers. Além disso, essa classificação possui a capacidade de ser extendida com a criação de novas interfaces de novos recursos que não se encaixem em nenhuma das classificações;
	\item Alimentação da ontologia com os recursos que estiveram no ambiente.
\end{itemize}

Tais contribuições possibilitam o desenvolvimento de novas funcionalidades que serão detalhadas na seção~\ref{sec:trabalhosFuturos}.

\section{Trabalhos futuros}
\label{sec:trabalhosFuturos}

No decorrer deste trabalho, algumas funcionalidades e melhorias foram percebidas. A partir deste trabalho será possível criar novas funcionalidades que irão tornar o \emph{uOS} uma plataforma mais completa. A que mais se destaca é a possibilidade de escolha automática do melhor recurso disponível, tornando essa tarefa independente da ação do usuário.  Essa escolha automática poderá se utilizar de heurísticas e regras de produção pra definir qual será o melhor recurso equivalente que irá substituir determinado serviço. A escolha poderá ser realizada baseando-se em dados 
de QoS, confiança, segurança, etc. Dessa forma, o usuário não precisará se preocupar em definir qual será o melhor recurso disponível no ambiente para que então possa tomar uma decisão.

Atualmente, caso um serviço se torne indisponível, caberá ao usuário escolher um novo recurso provedor daquele serviço.
Entretanto, a equivalência de recursos permite a criação de um mecanismo de recuperação de falhas que possibilita que a adaptabilidade de serviços ocorra de forma transparente ao usuário.

Como o \emph{uOS} alimenta uma ontologia sempre que descobre novos recursos no ambiente, ao se utilizar a ontologia para determinar a equivalência entre recursos é possível agregar relações adicionadas pelas aplicações para descobrir novas equivalências em tempo de execução, expandindo o conhecimento sobre o ambiente.