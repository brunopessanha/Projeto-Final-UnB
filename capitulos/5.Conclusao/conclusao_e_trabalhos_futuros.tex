\chapter{Conclusão e trabalhos futuros}

A quantidade de dispositivos inteligentes disponíveis para consumo têm aumentado consideravelmente com o passar dos anos. Inúmeros aparelhos têm se tornado essenciais no dia-a-dia das pessoas, e a tendência é que essa dependência aumente gradativamente. Diminutos eletrônicos conectados à internet têm reduzido as distâncias e possibilitado novos tipos de interações nunca antes imaginadas. Atualmente, utilizando um único \emph{smartphone}, é possivel acessar o seu \emph{site} preferido, consultar as ações da bolsa de valores e a previsão do tempo, mudar o canal da sua televisão, ligar a luz do seu quarto, obter rotas urbanas, dentre várias outras tarefas.
\begin{comment}
É nesse ambiente cada vez mais ubíquo que a classificação de recursos se insere: ambientes com dispositivos heterogêneos (móveis ou não) interagindo entre si, trocando informações ou utilizando capacidades características de cada aparelho.
\end{comment}

Com recursos cada vez mais capazes e variados, faz-se necessária uma padronização dos mesmos para se atingir o objetivo da ubicomp. É nessa tarefa que a DSOA se concentra, porém apenas com a classificação de recursos este acesso será flexível para abranger essa heterogeneidade de opções.

Com a operacionalização da classificação de recursos, tornou-se possível saber quais são os recursos equivalentes disponíveis no ambiente, colaborando para melhorar a adaptabilidade de serviços do \emph{uOS}. Antes deste trabalho os recursos não tinham relação entre sí, dificultando a escolha por parte do usuário e impossibilitando que o \emph{middleware} pudesse tomar esta decisão sem consulta-lo. Assim, caso alguém precise de um recurso de \emph{mouse}, todo recurso equivalente será encontrado de forma transparente, o que amplia a gama de recursos a disposição do usuário, levando a um melhor aproveitamento das capacidades do ambiente.

A definição das interfaces de recursos básicos servirá de guia para o desenvolvimento de novos \emph{drivers}. Isso fará com que a construção de novos serviços seja feita de forma mais coesa e que esses serviços sejam reutilizáveis e compativeis entre si. Isso é possível, pois a definição de \emph{drivers} de novos recursos é feita por meio de novas interfaces que estendem as interfaces já definidas.

A ontologia do \emph{uOS} é alimentada com os recursos que passam pelo ambiente. Com a ontologia contendo informações sobre os recursos é possível que as aplicações estendam a base de conhecimento sobre o ambiente e extraiam inteligência e relações mais aprimoradas sobre ele.

A aplicação \emph{Hydra} foi adaptada para utilizar a equivalência de recursos para encontrar recursos no ambiente, o que expandiu a capacidade desta aplicação de reconhecer recursos do ambiente.

Tais contribuições possibilitam o desenvolvimento de novas funcionalidades que serão detalhadas na seção~\ref{sec:trabalhosFuturos}.

\section{Trabalhos futuros}


\begin{enumerate}
	\item Possibilidade de criação de mecanismo de failover que possibilita que a adaptabilidade de serviços seja transparente ao usuário. Atualmente quando um recurso se torna indisponível, o usuário deve selecionar qual seria um outro recurso equivalente que pudesse continuar provando aquele serviço. 
	\item Inclusão de inteligência artifical para escolha automática de melhor recurso disponível.
		- Usando dados de QOS, confiança etc para a escolha.
	\item Utilizar a ontologia em vez da árvore de equivalência para determinar a equivalência entre recursos.
	\item Hydra redirecionar consulta feita por mouse utilizando a ontologia
\end{enumerate}