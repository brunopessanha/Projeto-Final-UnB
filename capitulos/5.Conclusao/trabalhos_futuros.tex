\section{Trabalhos futuros}
\label{sec:trabalhosFuturos}

Com a conclusão deste trabalho, tornou-se possível adicionar novas funcionalidades que poderão tornar o \emph{middleware uOS} uma plataforma mais robusta. Pode-se, por exemplo, otimizar o processo da adaptabilidade de serviços, desenvolvendo um mecanismo de escolha automática de recursos de forma a retornar o melhor recurso disponível para cada ocasião. Para tal, cada algoritmo de seleção poderá valer-se de heurísticas, regras de produção, ontologias, QoS, níveis de confiança ou segurança, etc. Com isso, aproximar-se-á o \emph{uOS} de um sistema cada vez mais ubíquo (exigindo menos interação com o usuário) e tolerante a falhas, pois caso um serviço torne-se indisponível, um novo recurso provedor poderá ser automaticamente selecionado para substituição.

Outro fator a ser observado, é que a definição de tipos básicos possibilita a criação de uma biblioteca de \emph{drivers}, que devem ser utilizados no desenvolvimento de novos tipos. Para tal, basta estender um tipo apropriado e adicionar os serviços necessários para o seu correto funcionamento. Dessa forma, obtêm-se uma biblioteca expansível em que cada \emph{driver} desenvolvido pode ser aproveitado na criação de \emph{drivers} subsequentes.

Por fim, pode-se ainda substituir a árvore de recursos pela ontologia do \emph{uOs}, a qual permitirá utilizar diversas informações adicionadas em tempo de execução pelas aplicações para descobrir novas relações de  similaridade, expandindo o conhecimento sobre o ambiente.