\section{Trabalhos futuros}
\label{sec:trabalhosFuturos}

No decorrer deste trabalho, algumas funcionalidades e melhorias foram percebidas. A partir deste trabalho será possível criar novas funcionalidades que irão tornar o \emph{uOS} uma plataforma mais completa. A que mais se destaca é a possibilidade de escolha automática do melhor recurso disponível, tornando essa tarefa independente da ação do usuário.  Essa escolha automática poderá se utilizar de heurísticas e regras de produção pra definir qual será o melhor recurso equivalente que irá substituir determinado serviço. A escolha poderá ser realizada baseando-se em dados 
de QoS, confiança, segurança, etc. Dessa forma, o usuário não precisará se preocupar em definir qual será o melhor recurso disponível no ambiente para que então possa tomar uma decisão.

Atualmente, caso um serviço se torne indisponível, caberá ao usuário escolher um novo recurso provedor daquele serviço.
Entretanto, a equivalência de recursos permite a criação de um mecanismo de recuperação de falhas que possibilita que a adaptabilidade de serviços ocorra de forma transparente ao usuário.

Como o \emph{uOS} alimenta uma ontologia sempre que descobre novos recursos no ambiente, ao se utilizar a ontologia para determinar a equivalência entre recursos é possível agregar relações adicionadas pelas aplicações para descobrir novas equivalências em tempo de execução, expandindo o conhecimento sobre o ambiente.