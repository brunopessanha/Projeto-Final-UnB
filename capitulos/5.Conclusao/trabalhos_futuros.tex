\section{Trabalhos futuros}

No decorrer deste trabalho, algumas funcionalidades e melhorias foram percebidas, contudo nem todas puderam ser implementadas. Algumas por falta de tempo hábil, outras porque parte das ferramentas necessárias ainda não estavam finalizadas; é o caso da ontologia, que foi finalizada na fase final da implementação da árvore de recursos.

Abaixo, seguem as funcionalidades e melhorias levantadas:

\begin{enumerate}
	\item Criação de um mecanismo de \emph{failover}\footnote{Procedimento que permite a troca automática de recursos equivalentes, ou seja, sem a necessidade de interação com o usuário.} que possibilita que a adaptabilidade de serviços seja transparente ao usuário. Atualmente quando um recurso se torna indisponível, o usuário deve selecionar um outro recurso equivalente que possa continuar provendo o mesmo serviço;
	\item Inclusão de inteligência artifical que permita a escolha automática do melhor recurso disponível com base nos dados de QoS, confiança, segurança, etc;
	\item Utilizar a ontologia em vez da árvore de recursos para determinar equivalências;
	\item Permitir a Hydra redirecionar, utilizando a ontologia, uma consulta feita por mouse.
\end{enumerate}